\begin{resumopage}
\item[Charles Baudelaire] (Paris, 1821---\textit{id}. 1867), escritor francês, é
 hoje reverenciado como um dos paradigmas máximos da modernidade.
 Dono de uma imagética pujante e original, Baudelaire foi também um
 influente crítico de arte e um tradutor de grande envergadura. Alma
 inquieta e conturbada, via com desconfiança a era do progresso,
 entrevendo na modernidade uma morbidez oculta que sua sensibilidade
 extremada não tolerava. Em 1857, a publicação de \textit{As flores do mal}, sua
 obra-prima, ofende a moral burguesa e lhe vale um processo no qual é
 obrigado a pagar uma multa considerável, além de ter de retirar sete poemas do livro.
 Alguns dos sonetos ali encerrados já prefiguravam o simbolismo e o decadentismo, correntes que começavam a
 tomar corpo. Em \textit{Os paraísos artificiais} (1860), explora o potencial
 criador sob o efeito do ópio e do haxixe. Como tradutor, verte
 muitos dos contos e ensaios de  Edgar Allan Poe para o francês, tendo influído assim decisivamente
 para o futuro reconhecimento desse autor, que 
 exerceu influência em sua obra também. Solitário, doente e sem recursos,
 morre em 1867.
\item[Escritos sobre arte] reúne quatro textos da produção crítica de Baudelaire: 
``Da essência do riso'' (\textit{Le Portefeuille}, 1855), ``Alguns caricaturistas estrangeiros'' 
(\textit{Le Présent, 1857}), ``A arte filosófica'' (original encontrado entre papéis 
de Baudelaire e publicado postumamente em \textit{Arte romântica}) e 
``A obra e a vida de Eugène Delacroix'' (\textit{L'Opinion Nationale}, 1863).
Produzidos para periódicos, sem o objetivo de serem posteriormente coligidos em livro,
estes ensaios apresentam um conjunto de reflexões estéticas incomuns para o período, 
como o riso na caricatura, a definição da arte filosófica e a recuperação de autores pouco valorizados. 
\item[Plínio Augusto Coêlho] fundou em 1984 a Novos Tempos Editora, em
Brasília, dedicada à publicação de obras libertárias. A partir de
1989, transfere-se para São Paulo, onde cria a Editora Imaginário,
mantendo a mesma linha de publicações. É idealizador
e co-fundador do \textsc{iel} (Instituto de Estudos Libertários).
\item[Dirceu Villa] é poeta, tradutor e mestre em letras pela Universidade
de São Paulo. Autor do livro de poemas \textit{Descort} (Hedra, 2003), traduziu
\textit{Lustra}, de Ezra Pound (inédito) e colabora em diversos veículos de imprensa.

\end{resumopage}

