\documentclass[10pt]{hedrabook}
\usepackage[brazilian]{babel}
\usepackage{ucs}
\usepackage[utf8x]{inputenc}
\usepackage[svn,center,cam,a5]{hedracrop}
\usepackage{hedrabolsolayout,hedraextra}
\usepackage{hedratoc}
\usepackage[protrusion=true,expansion]{microtype}
\usepackage{comment,lipsum,footmisc}
\usepackage[minionint,mathlf]{MinionPro}
%\usepackage{gmverse}
%\usepackage{hedraebook}		   % hyperlinks: todos os \part têm que estar em CAIXA BAIXA
\usepackage{makeidx,hedraindex}  % cria índice
\makeindex			   % ...  índice

\begin{document}
\SVN $Id: BAUDELAIRE.tex 9570 2011-08-17 18:18:01Z oliveira $


\selectlanguage{brazilian}
\title{Escritos sobre arte} % príncipe
\author{Baudelaire} %Maquiavel
\begingroup\pagestyle{empty}
%\begin{blackpages}
	\maketitle
	\begin{techpage}{5cm}
		\vspace{-1.5cm}
		\putline{Copyright}{Hedra 2007}
		\putline{Tradução$^\copyright$}{Plínio Augusto Coêlho} 
		\putline{Primeira edição}{Editora Imaginário, 1998}
		\putline{Corpo editorial}{
			André Fernandes,\\
			Bruno Costa,
			Caio Gagliardi,\\
			Fábio Mantegari,
			Iuri Pereira,\\
			Jorge Sallum,
			Oliver Tolle,\\ 
 		        Ricardo Martins Valle,\\
			Ricardo Musse
		}
		\ \\

		\putline{Dados}{\begin{footnotesize}       
		Dados Internacionais de Catalogação na Publicação (CIP)\\
	        \vspace{0.5ex}\hrule\vspace{1ex} 
		Baudelaire, Charles (organização e tradução Plínio Augusto Coêlho) 
                                -- São Paulo : Hedra : 2008 .   
				\vspace{1ex}
				\setlength{\parindent}{3ex}
				Bibliografia.\\
				\textsc{isbn} 97-8857-715-085-4 \\
                                \vspace{1ex}
				1. Arte – Filosofia 2. Arte – Humor, sátira, etc. 
				3. Crítica da arte I. Título \\
				\vspace{1ex}
				\noindent 08-2744  \hspace{\stretch{1}} \textsc{cdd}-701.18
				\vspace{1ex}\hrule\vspace{1ex}
				Índice para catálogo sistemático:\\
				1. Arte : Crítica de arte 701.18
				\linebreak
				\end{footnotesize}
                                }
		\direitos
		\dadoseditora
		\depositolegal
	\end{techpage}
	\begin{frontispiciopage}{4cm}{}	
	\putline{Organização e tradução}{\textsc{Plínio Augusto Coêlho}} 
	\putline{\hspace{12ex}Introdução}{\textsc{Dirceu Villa}} %{Introdução|Organização|...}
%		\putline{Tradução}{\textsc{Ipse Lorem}}   %{Introdução|Organização|...}
	\end{frontispiciopage}
\begin{resumopage}
\item[Charles Baudelaire] (Paris, 1821---\textit{id}. 1867), escritor francês, é
 hoje reverenciado como um dos paradigmas máximos da modernidade.
 Dono de uma imagética pujante e original, Baudelaire foi também um
 influente crítico de arte e um tradutor de grande envergadura. Alma
 inquieta e conturbada, via com desconfiança a era do progresso,
 entrevendo na modernidade uma morbidez oculta que sua sensibilidade
 extremada não tolerava. Em 1857, a publicação de \textit{As flores do mal}, sua
 obra-prima, ofende a moral burguesa e lhe vale um processo no qual é
 obrigado a pagar uma multa considerável, além de ter de retirar sete poemas do livro.
 Alguns dos sonetos ali encerrados já prefiguravam o simbolismo e o decadentismo, correntes que começavam a
 tomar corpo. Em \textit{Os paraísos artificiais} (1860), explora o potencial
 criador sob o efeito do ópio e do haxixe. Como tradutor, verte
 muitos dos contos e ensaios de  Edgar Allan Poe para o francês, tendo influído assim decisivamente
 para o futuro reconhecimento desse autor, que 
 exerceu influência em sua obra também. Solitário, doente e sem recursos,
 morre em 1867.
\item[Escritos sobre arte] reúne quatro textos da produção crítica de Baudelaire: 
``Da essência do riso'' (\textit{Le Portefeuille}, 1855), ``Alguns caricaturistas estrangeiros'' 
(\textit{Le Présent, 1857}), ``A arte filosófica'' (original encontrado entre papéis 
de Baudelaire e publicado postumamente em \textit{Arte romântica}) e 
``A obra e a vida de Eugène Delacroix'' (\textit{L'Opinion Nationale}, 1863).
Produzidos para periódicos, sem o objetivo de serem posteriormente coligidos em livro,
estes ensaios apresentam um conjunto de reflexões estéticas incomuns para o período, 
como o riso na caricatura, a definição da arte filosófica e a recuperação de autores pouco valorizados. 
\item[Plínio Augusto Coêlho] fundou em 1984 a Novos Tempos Editora, em
Brasília, dedicada à publicação de obras libertárias. A partir de
1989, transfere-se para São Paulo, onde cria a Editora Imaginário,
mantendo a mesma linha de publicações. É idealizador
e co-fundador do \textsc{iel} (Instituto de Estudos Libertários).
\item[Dirceu Villa] é poeta, tradutor e mestre em letras pela Universidade
de São Paulo. Autor do livro de poemas \textit{Descort} (Hedra, 2003), traduziu
\textit{Lustra}, de Ezra Pound (inédito) e colabora em diversos veículos de imprensa.

\end{resumopage}



% Coloca página preta vazia à direita se texto preencher apenas uma página
\ifodd\thepage\paginabranca\fi

%\end{blackpages}
\endgroup
\setcounter{tocdepth}{0}     % amplitude da presença das partes no índice
\setcounter{secnumdepth}{-2} % amplitude da numeração das partes

\hedratoc

% Favor não alterar o segundo parâmetro (\baselineskip). 
% Para acertar entrelinhas, usar comando \linespread
\fontsize{10.5pt}{\baselineskip}\selectfont
\baselineskip=12.6pt   % Parâmetro válido apenas para corpo 10.5, para outros, acrescer 20% do valor do corpo%%


\chapter[Introdução, por Dirceu Villa]{introdução}

\section{o ensaísmo, grosso modo}

Não é do objetivo nem do escopo desta apresentação discorrer sobre o
ensaio, mas ela seria manca se não apresentasse uma ou duas ideias
nucleares sobre o assunto. Assim, o ponto inicial da nossa conversa
recua até a França do século \textsc{xvi}, e percebemos que depois de o grande	\index{França}
\index{Montaigne, Michel de}Michel de Montaigne (1533---1592) nomear o gênero e fornecer o
\textit{modus faciendi} já em nível de completa excelência em seus
\textit{Ensaios}, muitos autores adotaram a maneira de escrever
pensando com a liberdade erudita, como que de uma conversa educada,
propiciada pela forma. Nas palavras célebres de Montaigne, “leitor, sou
eu mesmo a matéria deste livro”, e nele serão encontrados “alguns
traços do meu caráter e das minhas ideias”.\footnote{ “Do autor ao
leitor”, em Michel Montaigne, \textit{Ensaios} (trad.
Sérgio Milliet), São Paulo, Abril Cultural, \textit{Os
pensadores}, vol.~\textsc{xi}, 1972, p.~11. Essa longa tradição 
reafirma"-se hoje,
nesses mesmos termos, nos ensaios brilhantes, peculiares e muito
pessoais de um grande poeta como Derek Walcott em \textit{What the \index{Walcott@Walcott, Derek|nn}
twilight says: essays}, por exemplo.}

Não se pode afirmar que Montaigne, em 1580, soubesse que a mão da
reflexão estética caberia tão bem na luva do ensaio. Os tratados, tendo
um compromisso mais científico e um andamento mais moroso, não
permitiam a ductibilidade característica de algo que, mesmo ainda tendo
suas regras escritas de composição, era aberta (e cada vez mais) a uma
\textit{ars combinandi}, às variáveis que se intensificariam desde a
erosão do terreno da retórica no século \textsc{xviii}. \index{Baudelaire, Charles}Baudelaire nos diz que
sua intenção não era escrever um tratado, mas sim “participar ao leitor
algumas reflexões que me ocorrem com frequência”, e um ou dois
parágrafos adiante iria se separar ainda mais do ponto de vista
acadêmico, atacando diretamente o que seriam pressupostos acadêmicos
sobre o tema, e se queixando de “certos professores tidos como sérios,
charlatães da seriedade, cadáveres pedantescos saídos dos frios
hipogeus do Instituto”.

Quem nada tinha com isso --- e foi traduzido e admirado por \index{Baudelaire, Charles}Baudelaire
--- era o estadunidense \index{Poe, Edgar Allan}Edgar Allan Poe. 
O pensamento artístico de Poe, de orgulhosa autodeliberação,
fugindo dos estereótipos inspiradores do romantismo e propondo, como
diríamos com \index{Mallarmé, Stéphane}Mallarmé, “quase uma arte”, serviu a sustentar e
desenvolver as próprias convicções de \index{Baudelaire, Charles}Baudelaire não apenas como
artista, mas artista que pensa sobre sua arte, como lemos em sua obra
de ensaios. Os ensaios literários de \index{Poe, Edgar Allan}Poe já apresentam precisamente o
formato que conhecemos, ou tome"-se o exemplo de “The poetic principle”
(``O princípio poético''), no qual comenta seu conceito de poesia, além de
poemas de \index{Byron, Lord}Byron, \indice{Shelley} e outros, imprimindo"-os no corpo do texto para
mais fácil inteligibilidade. E nos dá, é claro, sua definição de
poesia:

\begin{hedraquote}
Sucintamente, eu definiria a Poesia das palavras como
\textit{A Criação Rítmica da Beleza}. Seu único árbitro é o Gosto. Com
o Intelecto ou com a Consciência tem suas únicas relações colaterais. A
não ser que, incidentalmente, não se ocupa nem do Dever nem da Verdade.\footnote{ ``\textit{I 
would define, in brief, the Poetry of words as \textit{The Rhythmical
Creation of Beauty.} Its sole arbiter is Taste. With the Intellect or
with the Conscience it has only collateral relations. Unless
incidentally, it has no concern whatever either with Duty or with
Truth}'',
em Edgar Allan Poe, \textit{The works of the late Edgar
Allan Poe} (ed. Rufus Wilmot Griswold), New York, 1850,  vol. \textsc{iii}, p.~8. 
Tradução do introdutor, bem como as demais citações sem indicação.}
\end{hedraquote}

A base de \index{Poe, Edgar Allan}Poe nos artigos e textos de maior fôlego de \index{Baudelaire, Charles}Baudelaire deve ser levada em consideração, e principalmente uma abordagem menos
derivada de um suposto rigor tratadístico: há nos ensaios de \index{Baudelaire, Charles}Baudelaire
certa leveza e humor que terá tomado também da leitura de \index{Diderot, Denis}Diderot,
outra referência importante. Ao menos, percebe"-se que \index{Baudelaire, Charles}Baudelaire terá
apreciado o estilo de Diderot que lemos já na primeira frase de “Dos
autores e dos críticos”: “Os viajantes falam de uma espécie de homens
selvagens, que sopram no passante agulhas envenenadas. É a imagem dos
nossos críticos”.\footnote{ Denis Diderot, “Dos autores e dos críticos”
(trad. e notas J.~Guinsburg), em \textit{Os
pensadores}, vol.~\textsc{xxii}, São Paulo, Abril Cultural, 1973, pp.~491---496.}

Um autor como \index{Coleridge, Samuel Taylor}Samuel Taylor Coleridge (1772---1834), por exemplo, já
dava exemplos de como o texto crítico de poeta funcionaria. A
\textit{Biographia literaria} (1817) é ainda resultado de um tipo de
concepção “minha vida, minhas ideias”, mas é possível perceber nela a
conjunção do ponto de vista de artista da palavra com a estrutura
fragmentária que se dedica a escrutinar um assunto, partilhando
impressões e formulando um modo específico de abordar a literatura e o
pensamento. \index{Coleridge, Samuel Taylor}Coleridge é um autor muito intelectual e seu matiz é
filosófico.

Assim também com a obra intelectual muito notável de outro poeta
do período, o italiano \index{Leopardi, Giacomo}Giacomo Leopardi (1798---1837), que tende à
filosofia, impregnada das formas ditadas por seu muito profundo e
arraigado conhecimento das línguas e letras antigas da \indicex{Grécia}{Grecia} e de
\indice{Roma}, de onde os diálogos e os numerosos fragmentos filosóficos do
sugestivo título \textit{Zibaldone}, póstumo, que poderíamos traduzir
por “Miscelânea”. Vejamos, por exemplo, como escreve, em 1820, sobre a
obra de arte (ou “gênio”, como era o hábito à época):

\begin{hedraquote}
E mesmo o conhecimento da irreparável vaidade e falsidade de todo belo e de todo
grande é de certa beleza e grandeza que preenchem a alma, quando
esse conhecimento se encontra nas obras de gênio.\footnote{ ``\textit{E 
lo stesso conoscere l’irreparabile vanità e falsità di ogni bello e di
ogni grande è una certa bellezza e grandezza che rempie l’anima, quando
questa conoscenza si trova nelle opere di genio}'', em \index{Leopardi, Giacomo}Giacomo Leopardi, 
\textit{Poesie e prose} (a cura di Siro Attilio Nulli), Milano, Hoepli, 1972, p.~590.}           \index{Nulli@Nulli, Sirio Attilio|nn}
\end{hedraquote}

Embora o estilo nessas duas obras, uma inglesa e outra italiana, se
aproxime mais das \textit{Fusées} de \index{Baudelaire, Charles}Baudelaire, de seus diários e
fragmentos esparsos, ambas já provocam um efeito semelhante ao do
ensaio de poeta ou escritor que revela uma faceta crítica, que se
combina à inteligência prática de sua arte.

\section{fluência, ou folhas ao vento}

A fluência permitida pela página de jornal, e mesmo o princípio de
efemeridade desse tipo ordinário de papel impresso, fecundaram o
gênero: veremos que alguns dos ensaios de \index{Baudelaire, Charles}Baudelaire --- como o artigo
sobre \index{Delacroix, Eugène}Delacroix --- foram compostos como cartas à direção de jornais, ou
artigos breves de crítica. Em “Da essência do riso e, de um modo geral,
do cômico nas artes plásticas”,\footnote{ Publicado em
\textit{Curiosités esthétiques} (1868).} o próprio \index{Baudelaire, Charles}Baudelaire iria
assinalar esse aspecto transitório do jornal, referindo a função
imediata das caricaturas: “Assim como as folhas volantes do jornalismo,
elas desaparecem levadas pelo vento incessante que delas traz
notícias”. Esse caráter fugidio, quase de \textit{impromptu}, trouxe ao
ensaio --- que, já de nascença, como vimos, pretendia ser um registro
mais flexível do pensamento --- uma maleabilidade ainda maior, porque era
supostamente coisa bastante perecível.

Não foi tão perecível assim, verdade seja dita, ou não estaríamos neste
exato momento gastando tantas linhas numa apresentação aos ensaios de
\index{Baudelaire, Charles}Baudelaire: sendo o molde tão favorável à veiculação das ideias, os
ensaios são então recolhidos em livro, edições que compilam esse
devanear orientado como uma topografia do terreno acidentado do
pensamento de determinado autor. Hoje, a reputação do ensaio é também
considerável nos meios acadêmicos, e muitos professores reúnem sua
crítica esparsa em livros que colecionam a obra ensaística. \index{Burckhardt, Jacob}Jacob Burckhardt, 
ainda no século \textsc{xix}, chamou seu livro sobre a cultura do
\indice{Renascimento} italiano um “ensaio”. É um dispositivo que, além da
modéstia, confere limites ao estudo.

A brevidade do ensaio, patente desde o livro memorável e essencial de
\index{Montaigne, Michel de}Montaigne, permite também a concentração em um só ponto, excluindo a
necessidade de articular peças avulsas num todo coeso, e assim os
assuntos tratados marginalmente dentro de obras maiores tomaram o
primeiro plano, também porque se supunha que, como fragmento,
iluminariam de modo reflexivo o resto das matérias com que partilhassem
interesse.\footnote{ Uma ideia que só será ampliada por aquilo que, com
Ezra Pound, chamou"-se \textit{método ideogrâmico}, ou seja: imitando a  \index{Pound@Pound, Erza|nn}
estrutura dos ideogramas chineses e opondo"-se à hierarquização
dita \textit{aristotélica} do discurso (o que significa “começar
primeiro das coisas primeiras”, e assim por diante), os fragmentos
colados uns nos outros funcionariam como \textit{punti luminosi}, ou
“pontos luminosos”, em que a relação entre um e outro produziria um
novo e luminoso nexo de sentido.} 

Como resultado, avulta no gênero a opinião, que não é, no entanto, mero
palpite ignorante. A opinião dentro de um ensaio seguiria a receita
que, em outra ocasião, mas com propósito assemelhado, \index{Newton, Isaac}Isaac Newton
prescreveu assim: “Me apoio nos ombros de gigantes” (mesmo essa frase
ele havia tirado de algum de seus gigantes). Montaigne refere"-se a poetas,
pensadores, a todos aqueles indivíduos extraordinários que podem acorrer
em defesa ou ilustração de seu pensamento, ainda que “pessoal”, ainda
que recolhido à sua “ingenuidade física e moral”, como nos diz na
\textit{captatio benevolentia} da abertura do livro. \index{Baudelaire, Charles}Baudelaire
escreverá, logo na abertura do primeiro de seus próprios ensaios: “Este
é, puramente, portanto, um artigo de filósofo e de artista”.

A opinião, no ensaio, é assim a opinião educada e focalizada, que pode
ser debatida nos termos de uma discussão literária, artística ou
filosófica civilizada, na qual um argumento persuasivo se sustenta com
exemplos e modelos, ou por contraste, e poderá ser contraposta por
argumentos de peso equivalente: é o que nos asseguram aquelas páginas
dedicadas a um assunto escolhido, e sob foco intenso de atenção mais ou
menos especializada, que se aproxima por vezes do diletantismo. E,
assim, um ensaio pode ser um artigo bastante persuasivo. Ele se
inscreve talvez numa das primeiras ambiguidades entre público e
privado: ele aproxima e afasta, ele afeta uma conversa, mas se
desenvolve como um monólogo indutor.

São essas as características do gênero ensaístico que permaneceram
fazendo dele um veículo apropriado para o uso das discussões sobre
estética e filosofia; ele se adequa também a uma cultura do fragmento
--- como tem sido a nossa cultura ocidental há mais de um século ---,
mas obedece a uma lei clássica, até meio militar, de “dividir para
conquistar”. É a divisão, a brevidade, o fragmento aparentemente
descompromissado que nos conquista.

\section{solidificando um gênero}

A obra de crítica de \index{Baudelaire, Charles}Baudelaire praticamente rivaliza, em peso, com sua
obra literária. Essa afirmação se justifica da seguinte maneira: como
poeta, \index{Baudelaire, Charles}Baudelaire é lembrado por estabelecer o princípio do que
chamariam depois \textit{modernidade}, ou seja, aquela então nova
atitude de negatividade em relação ao objeto de atenção da arte e em
relação ao público (que ele apelidou, no prefácio às \textit{Flores do
mal}, “leitor hipócrita”, e era, no mesmo verso, seu semelhante, seu
irmão); como crítico, de modo análogo, \index{Baudelaire, Charles}Baudelaire sedimentou essa
figura hoje largamente difundida, a do poeta"-crítico,\footnote{ É
evidente que, como em tudo, não se tratava de uma novidade absoluta:
%r índice nota Dante
basta pensarmos em Meleagro na \textit{Antologia grega}, em Dante no	\index{Dante|nn}	\index{Meleagro|nn}
tratado \textit{De vulgari eloquentia}, nos poetas"-filólogos da
renascença italiana etc. Mas o diferencial está nessa crítica
ensaística, muitas vezes breve, e na atenção que não raro se opõe a
algo também recente como categoria isolada: a crítica de arte, a
crítica literária de corte acadêmico.} do artista que conjuga as duas
funções em sua obra e fornece um contraponto aos usos da crítica
acadêmica ou jornalística de sua época. Como esses poetas"-críticos que
já conhecemos desde o século \textsc{xx}, também \index{Baudelaire, Charles}Baudelaire visitou artistas
negligenciados pela crítica, e seu estilo cuidadosamente pensado e seu 
notório faro qualitativo escolheram sempre com refinado senso estético
e formal, criando núcleos de interesse e sentido que ainda hoje lemos
como um mapa de direções das artes na segunda metade do século \textsc{xix}.

Assim, enfrentou com elegância a ideia monolítica do belo e do
verdadeiro em arte, dedicando"-se ao riso e à caricatura em alguns de
seus mais notáveis textos críticos, reformando ou desfazendo boa parte
do edifício estético romântico; defendeu o artifício não apenas como
coisa cosmética,\footnote{ Como Ovídio, que dedicou um poema muito	\index{Ovidio@Ovídio|nn}
interessante aos cosméticos (``Medicamina faciei feminae'',
“Cosméticos para o rosto da mulher”), \index{Baudelaire, Charles}Baudelaire escreveu um
``Éloge du maquillage'', “Elogio da maquiagem”, em que comenta,
entre outras delicadezas, o direito da mulher parecer sempre “mágica e
sobrenatural”.} mas em termos que veríamos depois muito aprofundados
por \index{Pessoa, Fernando}Fernando Pessoa e Ezra Pound, ou seja, de uma técnica que serve a  \index{Pound@Pound, Erza}
revelar a emoção, transformando"-a ou \textit{fingindo"-a},\footnote{ O
verbo \textit{fingo}, em latim, significa “forjar”.} em arte, o que
na época soava provavelmente como antítese à espontaneidade declarada e
aparente do romantismo, ou simples aderência ao código de técnica
superficial do parnasianismo.\footnote{ Baudelaire entendia de modo
muito diferente o lema “\textit{l’art pour l’art}”, que lhe agradava: não a arte
como um enfeite descolado de qualquer relação com o mundo, mas
instalada e oferecida nele como uma contradição do artifício excelente
frente à espontaneidade, ao impulso natural.} 

Era, no entanto, uma visão que, coincidindo com a de \index{Poe, Edgar Allan}Edgar Allan Poe\footnote{ Como 
vimos, Poe também era um ensaísta. Mais lembrado
pelo sagaz \textit{Philosohy of composition} (\textit{Filosofia da composição}),
de 1846, em que aborda o processo de escrita do seu poema ``The raven”
(``O corvo'') e no qual achamos a célebre observação de tê"-lo começado pelo
fim, com o objetivo de melhor controlar seus efeitos sobre o leitor,
como uma lei da composição.} em muitos pontos, sobretudo no domínio
técnico do autor sobre os efeitos calculados de sua arte, se tornaria
um índice muito explorado depois por autores como \index{Mallarmé, Stéphane}Mallarmé e um bom
número de poetas modernos, principalmente. Devemos lembrar, juntamente,
que isso também ecoava o aprendizado clássico de \index{Baudelaire, Charles}Baudelaire, que sempre
lhe foi muito útil e pode ser constatado no emprego habilidoso e muito
frequente da alegoria em seus poemas, ou no emprego da \textit{clarté
française}, a clareza da prosa em francês,\footnote{ Baudelaire escreve:
“França, país de pensamento e de demonstração claros”.} tornando seus	\index{França|nn}
ensaios muito bem divididos e argumentados.

\index{Baudelaire, Charles}Baudelaire, portanto, preparou e antecipou, sob muitos aspectos, as
questões da chamada “modernidade”; estabeleceu um elo entre o
romantismo e o parnasianismo (no caso, leia"-se apenas \index{Guatier, Théophile}
Théophile Guatier) e forneceu, por suas ideias místicas e com o poema
“\textit{Correspondances}”, o princípio a partir do qual decadentistas e
simbolistas iriam mais tarde compor suas obras. Por fim, nos deu um dos
primeiros exemplos, com \index{Coleridge, Samuel Taylor}Coleridge, \index{Leopardi, Giacomo}Leopardi e \index{Poe, Edgar Allan}Poe, 
ao menos, do que seria o poeta"-crítico. 

\section{ideias: do particular para o geral}
Por que, então, as ideias de \index{Baudelaire, Charles}Baudelaire --- como as que encontramos
neste livro --- se tornaram tão importantes? Basicamente, porque
compõem o desenho apresentado anteriormente: tendo surgido de uma intuição que
é ao mesmo tempo artística e crítica, sobre um detalhe potencialmente
iluminador, se desenvolve num dos primeiros conjuntos expressivos de
crítica exercida por um artista. 

Assim é que, profundamente cristão e particularmente satânico,
\index{Baudelaire, Charles}Baudelaire flagra no riso, objeto de sua atenção no primeiro ensaio, a
dualidade que apresenta uma superfície de alegria para guardar no fundo
um sentimento demoníaco de superioridade. Leríamos mais tarde num texto
do filósofo \index{Bergson, Henri}Henri Bergson o mesmo ponto de partida sobre o assunto,
excluindo naturalmente a bruma fantástica do satanismo, para se deter
no aspecto da superioridade, o cerne da ironia.\footnote{ Embora Bergson	\index{Bergson@Bergson, Henri|nn}
se detenha no riso desde seus motivos mais elementares, desde reafirmar
que é especificamente uma característica humana, passando por vasculhar
as quimeras ridículas até asseverar o porquê de rirmos de um simples
tombo (a pessoa que cai demonstra rigidez quando deveria apresentar
flexibilidade), conclui que por meio do \textit{ridendo castigat mores}
(“com o riso corrigem"-se os costumes”) “o riso é sobretudo uma
correção”, em Henri Bergson, \textit{O riso, ensaio sobre o
significado do cômico} (tradução de Guilherme de Castilho),  Lisboa,		\index{Castilho@Castilho, Guilherme|nn}
Guimarães Editores, 1993.} Na argumentação desse primeiro ensaio de
\index{Baudelaire, Charles}Baudelaire, os leitores lembrarão, imagino, da discussão mais ou menos
teológica entre \index{Baskerville, William de}William de Baskerville e o venerável Jorge no romance
\textit{O nome da rosa}, de \index{Eco, Umberto}Umberto Eco, que tem por base o segundo
livro, perdido, da \textit{Poética} de \indice{Aristóteles}, sobre a comédia. 

Baskerville, franciscano, quer ver no riso uma virtude da sobrevivência,
do conhecimento e da sanidade, enquanto o venerável Jorge afirma que o
riso deforma tanto a moral quanto o rosto, que assume traços de macaco.
Baskerville responde que “macacos não riem, o riso é próprio do homem”.
Aos exemplos saudáveis do riso em \indice{Quintiliano} e outros autores gregos e
latinos, o venerável Jorge apenas retruca: “Eram pagãos”.\footnote{ Umberto Eco,
 \textit{O nome da rosa} (trad. Aurora Fornoni Bernardini
e Homero Freitas de Andrade), Rio de Janeiro, Record, 1986, p.~158.}

O riso seria portanto uma deformidade do rosto que implicaria uma
deformidade moral civilizadíssima, já que, pressupõe"-se, a inocência
não conhece o sentimento de superioridade, é toda candura (e esse é o
riso das crianças). A ideia, puramente uma divisão de categoria em
\indice{Aristóteles} (“a comédia é a imitação dos piores”), transforma"-se numa
reflexão moral que atende a outro sistema de classificação, o cristão.
O venerável Jorge, de \index{Eco, Umberto}Umberto Eco, abomina o riso por esse motivo, o mesmo pelo
qual o nosso poeta francês, a seu turno, o admira. O sábio também
desconfia moralmente do riso: “Ele se detém à beira do riso assim como
à beira da tentação”, escreve \index{Baudelaire, Charles}Baudelaire, e lembramos que é como o
infeliz Mêmnon\footnote{ “Memnon, ou la sagesse humaine” (``Mêmnon, ou a
sabedoria humana''). Curioso é notar que Baudelaire tem certa aversão a
\indice{Voltaire}, mesmo se, aqui e ali, encontramos pensamentos muito
semelhantes. O conto pode ser lido nesta mesma
coleção: Voltaire, \textit{Micromegas e outros contos} (tradução de
Graziela Marcolin), São Paulo, Hedra, 2007, pp. 59---65.} de Voltaire,
que traça um plano de sabedoria arruinado pelas tentações do mundo tão
pouco conhecidas por ele, que, portanto, não tem como se defender
delas. A desconfiança do sábio se dá porque, embora o homem não tenha
na boca os dentes do leão, “ele morde com o riso”. É o riso a um só
tempo cristão e satânico: \index{Baudelaire, Charles}Baudelaire dará o exemplo do riso de \indice{Melmoth},
de Charles Maturin,\footnote{ Charles Maturin, \textit{Melmoth the 		\index{Maturin@Maturin, Charles Robert}
wanderer} (ed. Victor Sage), London, Penguin, 2001. Ao mencionar o
bizarro romance gótico de Maturin, Baudelaire teria em mente passagens
como esta: ``[\ldots] \textit{And after looking on them for some time, burst into
a laugh so loud, wild, and protracted, that the peasants, starting with
as much horror at the sound as at that of the storm, hurried away}'', isto é,  “[\ldots] e
após observá"-los por algum tempo, explodiu numa gargalhada tão alta,
selvagem e prolongada, que os camponeses, com tanto horror por tal som
quanto que pelo da tempestade, fugiram correndo”, p.~35.} daquele que
%r linha viúva entre páginas alma ao diabo --- neste nosso caso, figuradamente.
vendeu a alma ao diabo --- neste nosso caso, figuradamente.

Devemos notar, ainda nesse ensaio, que a divisão que \index{Baudelaire, Charles}Baudelaire
estabelece entre os diferentes tipos de humor das nacionalidades
europeias foi feita, de modo assemelhado, na tradução do filho de
\index{Gautier, Théophile}Théophile Gautier para \textit{As aventuras do Barão de Münchausen}, a
partir da versão de Gottfried August Bürger, publicada em \indice{Londres} em			\index{Burger@Bürger, Gottfried August}
1866. Lemos, no prefácio:

\begin{hedraquote}
A \textit{gaieté} francesa nada tem a ver com o \textit{humour}
britânico; o \textit{Witz} alemão difere da \textit{bufoneria}
italiana, e o caráter de cada nacionalidade aí se revela em sua livre
expansão.\footnote{ Gottfried August Bürger, \textit{As aventuras do
Barão de Münchausen} (trad. Moacir Werneck de Castro; ilustr. 
Gustave Doré), Belo Horizonte, Villa Rica, 1990, p.~13.}		\index{Dore@Doré, Gustave|nn}
\end{hedraquote}

\index{Gautier, Théophile}Théophile Gautier \textit{fils} também não inventou isso, que é apenas
um dos modos mais antigos e comuns de se diferenciar culturalmente
desenhos num mapa. Decerto, \index{Baudelaire, Charles}Baudelaire explora com maior detalhe o que
pensa que seja a veia humorística de cada país; dirá que \index{Rabelais, François}Rabelais, em
seus textos grotescos, conserva sempre algo “útil e racional”; a
“sonhadora \indice{Germânia}” fornece o “cômico absoluto”; os ingleses, ou os
que vivem nos “reinos brumosos do \textit{spleen}”, têm um “cômico feroz”; o
humor dos italianos é “excêntrico”, o dos espanhóis, “cruel”,
“sombrio”.

Uma extensão do tema pode ser lida em dois ensaios de \index{Baudelaire, Charles}Baudelaire: um
deles pertence a esta coleção e se chama “Alguns caricaturistas
estrangeiros”.

Nele, descobrimos como nosso autor percebe o poder do detalhe deformador
da caricatura, que manipula a opinião sem dizer uma só palavra. A
deformação é um aspecto da arte, que para representar produz
deformações proporcionadas;\footnote{ Baudelaire tem perfeita
consciência disso. Escreve: “O grotesco domina o cômico de uma altura
proporcional”.} ou seja, a arte necessariamente deforma o objeto, e
não é outra a função das figuras de linguagem, de conceitos como a
concisão, a proporção. Na caricatura, como em toda arte que aborda o
\textit{genus humile} (gênero humilde) do cômico, a deformação tem o
caráter específico de representar coisa ridícula e desfavorável.
\index{Baudelaire, Charles}Baudelaire comenta artistas muito diversos no gênero, e o faz de caso
pensado, rabiscando o croqui de uma tipologia da caricatura.

Todos, no entanto, partilham o uso da alegoria (como, de resto, ocorre
ainda hoje: basta lembrar das charges políticas de \indice{Angeli} na
\textit{Folha de S.~Paulo}, por exemplo), mas \index{Hogarth, William}William Hogarth, o
primeiro a ser abordado, é visto na particularidade de uma imaginação
humorística mórbida, dos detalhes macabros da morte tornados risíveis.
Hogarth é um erudito, um grande curioso cheio de referências e,
evidentemente, um artista da inteligência, que produz menos
deformidades explícitas (como o rosto ridiculamente sinistro do padre
examinando um machado em \textit{The invasion plate} \textsc{i}, que talvez nos
recorde algo de \index{Goya, Francisco José de}Goya) e mais uma ironia entre as figuras em condições
indecentes ou francamente ordinárias.

Já Goya e \index{Brueghel, Pieter}Brueghel são vistos por \index{Baudelaire, Charles}Baudelaire como os tipos “sempre
grande artista” ou que fazem “coisas belíssimas” --- nesse último caso,
falando dos flamengos. Ao abordar os dois, \index{Baudelaire, Charles}Baudelaire, que não é apenas
um crítico literário, mas muito talentoso e cuidadoso crítico de arte,
separa a arte da gravura (sobretudo das técnicas de água"-forte e
água"-tinta), que é a arte da caricatura por excelência de \index{Goya, Francisco José de}Goya ---
tecnicamente com pontos de contato com \indice{Rembrandt} ---, da arte
miniaturista flamenga de \index{Brueghel, Pieter}Brueghel, para a qual a alegoria é como uma anedota,
ou toma a forma de um ditado popular, que demonstram em suas divertidas
diabruras toda a “força da alucinação”. Os \textit{Caprichos} de Goya
são vistos com interesse por \index{Baudelaire, Charles}Baudelaire porque criam um ruído na
alegoria ao misturarem"-na com enigmas e símbolos de interpretação
duvidosa ou aberta, em geral para efeito assustador: a deformidade
conduz a um mistério sombrio.

É como \index{Hansen, João Adolfo}João Adolgo Hansen escreve em seu livro
\textit{Alegoria}:\footnote{ João Adolfo Hansen, \textit{Alegoria:
construção e interpretação da metáfora}, São Paulo/Campinas,
Hedra/Unicamp, 2006, pp. 66---67 e 86.} trata"-se da
\textit{permixta apertis allegoria}, ou alegoria imperfeita, quando
escreve sobre o casal amarrado a um tronco de árvore, com a figura
perturbadora de uma gigantesca coruja cujas garras se apoiam na cabeça
da mulher, gravura que \index{Goya, Francisco José de}Goya chamou “¿\textit{No hay quien nos desate}?”
(``Não há quem nos desate?''). O casal atado sem que um consiga
escapar do outro é uma alegoria bastante clara do casamento infeliz,
mas a terrível coruja insere um elemento de interpretação enigmática,
indecisa.

E \index{Baudelaire, Charles}Baudelaire, claramente extasiado com as visões de \index{Goya, Francisco José de}Goya, escreve: 

\begin{hedraquote}
Une à alegoria, à jovialidade, à sátira espanhola do bom tempo de
\index{Cervantes, Miguel de}Cervantes um espírito muito mais moderno, ou pelo menos que foi muito
mais procurado nos tempos modernos, o amor do inapreensível, o
sentimento dos contrastes violentos, dos pavores da natureza e das
fisionomias humanas estranhamente animalescas pelas circunstâncias.
\end{hedraquote}

O que é também muito interessante por nos dar uma definição do próprio
\index{Baudelaire, Charles}Baudelaire sobre como via a modernidade; ou, digamos com mais precisão,
a \textit{sua} modernidade.

Evidentemente, as imagens de bruxas e sabás não lhe passam
despercebidas. Assinala, adiante, num comentário aos desenhos de \index{da Vinci, Leonardo}Leonardo da
Vinci, o que lhe parece a ausência de uma \textit{uis comica}, a
ausência de uma veia cômica, e lê as figuras do mestre italiano como
retratos da feiura, resultados da onívora curiosidade de \index{da Vinci, Leonardo}da Vinci pela
natureza em suas variadas formas --- o que é, por outro lado, lê"-lo
através do retrato de \indice{Vasari}, em \textit{Le vite dei più eccellenti
pittori, scultori e architetti}, até certo ponto anedótico naquele modo
de se redigir uma nota biográfica, bastante diverso da “veracidade
factual” suposta no gênero atualmente.

Percebemos que mesmo seu artigo sobre a arte filosófica se transforma
num exame da monstruosidade, uma vez que a filosofia na arte provocaria
a deformação do pensamento sobre a representação, através da fórmula
%r grafado Chavenard na prova
que atribui a \index{Chenavard, Paul}Chenavard e à arte alemã: “É uma arte plástica que tem a
pretensão de substituir o livro, quer dizer, rivalizar com a arte de
imprimir para ensinar a história, a moral e a filosofia.” Ele percebe
que a arte começa a ser trocada pelo \textit{típico histórico}, pelo
efeito de mera ilustração episódica. 

Sua noção disso é a da decadência de uma arte --- não como
veríamos depois na \indice{França} com os nomes de \index{Mallarmé, Stéphane}Mallarmé, \index{Huysmans, J.-K.}Huysmans, \index{Verlaine, Paul}Verlaine
etc., em que “decadência” é revertida para o limite de um sentido
“positivo” do excesso de civilização, do brilho mais intenso da estrela
que está para se apagar ---, mas sim a decadência de velhice e cansaço
de uma cultura que exaure suas energias criativas, e que “corresponde
ao período no qual entraremos em breve e cujo começo está marcado pela
supremacia da \indice{América} e da indústria”, escreve, revelando inesperados
dons proféticos.

Estão entremeados nessa crítica desfavorável de \index{Baudelaire, Charles}Baudelaire o
raciocínio, os limites da forma, a contraposição da inocência pelo
engenho --- que nos devolveria, num círculo, às suas observações sobre
a natureza do riso. Vemos \index{Baudelaire, Charles}Baudelaire fascinado pelas figuras grotescas,
pelas figuras que desempenham funções que se opõem às de sua natureza,
como a Morte tocando música encantadora ao violino, na \textit{Danse
des mortes}; ou fascinado por preciosas bizarrices, como será mais
tarde o caso do baudelairiano personagem Des Esseintes (do romance
\textit{Às avessas}, de \index{Huysmans, J.-K.}J.-K. Huysmans), quando observa e descreve
gravuras holandesas nas quais a crueldade ou o horror se tornam itens
de curiosidade quase que desinteressada, de concentração nos
esplendores das incongruências; ou o que, como nas \textit{Flores do mal}, 
poderíamos chamar de “horror simpático”:

\begin{hedraquote}
\begin{verse}
--- Deste céu bizarro e cinzento,\\
E turvo como o teu destino, \\
À tua alma que pensamento \\ 
Desce? Responde, libertino. \\ 
\ \\ 
--- Insaciavelmente guloso \\
Do que é incerto e que é sibilino, \\ 
\indicex{Ovídio}{Ovidio} não serei choroso \\ 
Quando expulso do Éden latino. \\ 
 \ \\
Céus lacerados e tristonhos, \\  
Em vós meu orgulho se fita; \\
As nuvens de dor infinita\\ 
 \ \\
São carros fúnebres dos sonhos, \\ 
Vosso clarão a sombra traz \\ 
Do Inferno à que minha alma apraz!\footnote{ \textit{``--- De ce ciel bizarre et livide, / 
Tourmenté comme ton destin, /
Quels pensers dans ton âme vide / 
Descendent? réponds, libertin. //
--- Insatiablement avide /
De l'obscur et de l'incertain, /
Je ne geindrai pas comme Ovide /
Chassé du paradis latin.  //
Cieux déchirés comme des grèves /
En vous se mire mon orgueil; /
Vos vastes nuages en deuil //
Sont les corbillards de mes rêves, / 
Et vos lueurs sont le reflet /
De l'enfer où mon coeur se plaît.''}
Baudelaire, Charles. \textit{As flores do mal} (trad.,
pref. e notas de Jamil Almansur Haddad), São Paulo, Difel, 1964, p.~217.} 
\end{verse}
\end{hedraquote}

“O Inferno em que meu coração se compraz”. Esse poema codifica uma das
bases do pensamento poético de \index{Baudelaire, Charles}Baudelaire nas \textit{Flores do mal},
unindo repulsa e atração, concomitantes, como lemos também nestes
ensaios e nos \textit{Pequenos poemas em prosa}: como flagra neles a
cidade de \indice{Paris}, a metrópole. A força, a cor, a agitação, os contrastes
ferozes e o arrebatamento são também os motivos que o levam, neste
contínuo de sentido, dos caricaturistas às deformações de \index{Goya, Francisco José de}Goya, da
desproporção filosófica à arte plástica de \index{Delacroix, Eugène}Eugène Delacroix. Sobre ele
\index{Baudelaire, Charles}Baudelaire escreve: “Eugène Delacroix era uma curiosa mistura de
ceticismo, polidez, dandismo, vontade ardente, astúcia, despotismo e,
enfim, uma espécie de bondade particular e de ternura moderada que
acompanha sempre o gênio”.

Delacroix é um dos pintores mais intensos da pintura francesa.
Contraposto à delicadeza de caixa de música dos quadros adoráveis de
\indice{Watteau}, que compunham uma sociedade do gosto superficial e requintado,
\index{Delacroix, Eugène}Delacroix visitará as terras do heroísmo e da revolução, e irá do
exótico e peregrino ao auto"-retrato de linhas dramáticas. O equivalente
disso para \index{Baudelaire, Charles}Baudelaire em música é \index{Wagner, Richard}Wagner. Não por acaso, \index{Baudelaire, Charles}Baudelaire,
que não era crítico musical tão pronto como crítico de arte, escreveu a
Wagner que passagens de sua música o faziam imaginar um vermelho
profundo. 

Podemos encontrar o equivalente visual da música trágica e poderosa de
\index{Wagner, Richard}Wagner na pintura de \index{Delacroix, Eugène}Delacroix. Basta lembrar de \textit{Fantasia
árabe}, quadro inscrito no Salão de 1834, no qual o céu de batalha, cor
de terra e alaranjado, encontra uma equivalência visual de sua
intensidade no cavalo negro do árabe em primeiro plano, que se empina à
vista do tumulto. Sua pintura é construída um passo além daquela que,
dois séculos antes dele, se propunha como o contraste dramático, fosse
da escuridão com luzes fortes e focalizadas, fosse o aproveitamento
também dramático das diagonais no quadro.

A amizade do jovem poeta pelo pintor mais velho começa em 1845, segundo
o próprio \index{Baudelaire, Charles}Baudelaire na carta"-artigo a \textit{L’Opinion Nationale}, de
novembro de 1863, que integra este nosso livro. E \index{Baudelaire, Charles}Baudelaire afirma que
a grandeza de \index{Delacroix, Eugène}Delacroix está em pintar o invisível: certamente. Outro
pintor que menciona entre os grandes franceses é \index{David, Jacques"-Louis}Jacques"-Louis David,
que vem, aliás, muito a propósito para entendermos o que \index{Baudelaire, Charles}Baudelaire
quer dizer com pintar o invisível. Se \indice{Watteau} (ou \indice{Fragonard}, pouco mais
tarde) é quase a delicadeza da miniatura, que mimetiza sua sociedade do
detalhe,\footnote{ Como escreve o historiador Piero Camporesi sobre o \index{Camporesi, Piero|nn}
gosto no século \textsc{xviii}: “A interiorização do prazer é destilada pela
miniaturização da paisagem e o apequenamento das coisas. O olho tem de
ser acariciado por coisas agradáveis, de medida justa mas tendentes à
bem temperada graciosidade”, em Piero Camporesi,
\textit{Hedonismo e exotismo: A arte de viver na Época das Luzes
}(trad. Gilson César Cardoso de Souza), São Paulo, Editora Unesp,
1996, p.~127.} David tem escopo maior; como Delacroix, ele também
pinta o invisível, mas um invisível de “fria” alegoria, de fixidez
estatuária, como vemos no famoso \textit{Juramento dos horácios}, de
1784, e não a velocidade e a paixão, como Delacroix, de quem \index{Baudelaire, Charles}Baudelaire
sabiamente diz: “Ele fez isso --- observe"-o bem, senhor --- sem outros
meios além do contorno e da cor; ele o fez melhor do que ninguém; ele o
fez com a perfeição de um pintor consumado, com o rigor de um literato
sutil, com a eloquência de um músico apaixonado”. Nesse perpassar das
artes e sensações, seria preciso fixar o adjetivo “apaixonado”.

E essa descrição nos apresenta também uma lei que é tanto plástica
quanto poética, encontrável nas ideias de \index{Baudelaire, Charles}Baudelaire também sob a forma
daquela mística infusão de sentidos do soneto “Correspondances”, das
\textit{Flores do mal}, quando escreve: 

\begin{hedraquote}
\begin{verse}
Como os longos ecos que de longe se confundem \\ 
Numa tenebrosa e profunda unidade, \\ 
Vasta como a noite e como a claridade, \\ 
Os perfumes, as cores e os sons se correspondem.\footnote{ \textit{``Comme de longs échos qui de loin se confondent /
Dans une ténébreuse et profonde unité, /
Vaste comme la nuit et comme la clarté, / 
Les parfums, les couleurs et les sons se répondent.''} 
Charles Baudelaire, \textit{As flores do mal}
(trad., intr. e notas Ivan Junqueira), Rio de Janeiro, Nova
Fronteira, 1985, pp.~114---115.}
\end{verse}
\end{hedraquote}

O que mencionei anteriormente da visão de \index{Baudelaire, Charles}Baudelaire ao ouvir \index{Wagner, Richard}Wagner também se
aplica ao artigo “Richard Wagner e \indice{Tannhäuser} em \indice{Paris}”, no qual
\index{Baudelaire, Charles}Baudelaire dizia suas impressões sobre a apresentação musical e rebatia
críticas francesas ao compositor. Encontramos o soneto
“Correspondances” citado como exemplo do que queria dizer, associando
sons a cores, ou visões. E completa: “O que seria verdadeiramente
surpreendente é que o som \textit{não pudesse} sugerir a cor, que as
cores \textit{não pudessem} dar a ideia de uma melodia, e que o som e a
cor fossem impróprios para traduzir ideias”.\footnote{ Charles Baudelaire,
 \textit{Obras estéticas: filosofia da imaginação criadora}
(trad. Edison Darci Heldt), Petrópolis, Vozes, 1993, p.~162. 
O célebre soneto das vogais, de Rimbaud, deriva dessas ideias.} 	\index{Rimbaud|nn}

Podemos depreender disso o valor que \index{Baudelaire, Charles}Baudelaire confere à
\textit{sinestesia}, essa mágica mistura e correspondência, algo que
ele encontrava na arte que mais lhe suscitava interesse, fosse a poesia
ou a pintura: elas estão imbuídas de qualidades sensórias
intercambiáveis. 

O entusiasmo de \index{Baudelaire, Charles}Baudelaire pela arte de \index{Delacroix, Eugène}Delacroix, assim como os hábitos
clássicos de sua escrita, ou o apreço por \index{Gautier, Théophile}Gautier, devem também
demonstrar aos leitores que a costumeira divisão periódica da
literatura em “movimentos” que se oporiam não é tão factual quanto
parece, pois os escritores elegem seus antecessores com muito mais
liberdade do que os manuais de literatura fazem acreditar: é possível
perceber como \index{Baudelaire, Charles}Baudelaire, através de suas posições críticas, se
estabelece num contínuo em que interagem lemas clássicos, preceitos
“barrocos”, romantismo, parnasianismo, decadência, simbolismo,
“modernidade”. A fluidez do pensamento artístico dos grandes autores
normalmente não se fecha em meia dúzia de estilemas de fácil
identificação.

Essa é a essência do que, suponho, poderemos apreciar na mente invulgar
que teceu as considerações inteligentes e bem"-humoradas destes ensaios
a seguir. Boa leitura.

\part{escritos sobre arte} % em minúscula

\chapter[Da essência do riso]{da essência do riso\subtitulo{e, de um modo geral,\break do cômico nas artes plásticas}}


\index{Napoleão|see{Bonaparte}}

\sectionitem

\noindent\textsc{Não quero escrever} um tratado da caricatura; quero simplesmente
participar ao leitor algumas reflexões que me ocorrem com frequência em
relação a esse gênero singular. Essas reflexões tinham se tornado para
mim uma espécie de obsessão; eu quis me acalmar. Fiz, por sinal, todos
os esforços para colocar neste texto uma certa ordem e tornar, assim,
sua assimilação mais fácil. Este é, puramente, portanto, um artigo de
filósofo e de artista. Sem dúvida uma história geral da caricatura em
suas relações com todos os fatos políticos e religiosos, sérios ou
frívolos, relativos ao espírito nacional ou à moda, que agitaram a
humanidade, é uma obra gloriosa e importante. O trabalho ainda está por
ser feito, pois os ensaios publicados até o presente momento são apenas
materiais; todavia, pensei que era preciso dividir o trabalho. É claro
que um trabalho sobre a caricatura, assim compreendido, é uma história
de fatos, uma imensa galeria anedótica. Na caricatura, bem mais do que
nos outros ramos da arte, existem dois tipos de obras preciosas e
recomendáveis sob diferentes aspectos e quase contrários. Estas só
valem pelo fato que elas representam.

Têm direito, sem dúvida, à atenção do historiador, do arqueólogo e até
mesmo do filósofo; devem tomar seu lugar nos arquivos nacionais, nos
registros biográficos do pensamento humano. Assim como as folhas
volantes do jornalismo, elas desaparecem levadas pelo vento incessante
que delas traz notícias; mas as outras, e são aquelas das quais quero
especialmente me ocupar, contêm um elemento misterioso, durável,
eterno, que as recomenda à atenção dos artistas. Coisa curiosa e
verdadeiramente digna de atenção a introdução desse elemento
inapreensível do belo até nas obras destinadas a representar ao homem
sua própria feiura moral e física! E, coisa não menos misteriosa, esse
espetáculo lamentável excita nele uma hilaridade imortal e
incorrigível. Eis, portanto, o verdadeiro tema deste artigo.

Um escrúpulo me arrebata. Seria preciso responder, por meio de uma
demonstração sistemática, a uma espécie de questão prévia que sem
dúvida desejariam, maliciosamente, trazer à luz certos professores
tidos como sérios, charlatães da seriedade, cadáveres pedantescos
saídos dos frios hipogeus do Instituto, e retornados à terra dos vivos,
como certos fantasmas avaros, para arrancar algum dinheiro de
complacentes ministérios? Em primeiro lugar, diriam eles, a caricatura
é um gênero? Não, responderiam seus cúmplices, a caricatura não é um
gênero. Ouvi ressoar em meus ouvidos semelhantes heresias em jantares
de acadêmicos. Essas boas pessoas deixavam passar a seu lado a comédia
de \indiceAB{Robert}{Macaire} sem perceber nisso grandes sintomas morais e
literários. Contemporâneos de \indiceBA{Rabelais}{François}, eles o teriam tratado de vil e
grotesco bufão. Em verdade, é preciso demonstrar, portanto, que nada do
que sai do homem é frívolo aos olhos do filósofo? Com toda certeza
será, menos que qualquer outro, esse elemento profundo e misterioso que
nenhuma filosofia analisou a fundo até agora.

Iremos, portanto, nos ocupar da essência do riso e dos elementos
constitutivos da caricatura. Mais tarde, examinaremos, talvez, algumas
das obras mais extraordinárias produzidas nesse gênero.

\sectionitem

\textit{O Sábio só ri ao tremer. }De que lábios cheios de autoridade, de
que pena perfeitamente ortodoxa saiu essa estranha e surpreendente
máxima? Ela nos vem do rei filosófico da \indice{Judeia}? Deve"-se atribuí"-la a
\indiceAB{Joseph}{de Maistre}, esse soldado animado pelo Espírito Santo? Tenho uma
vaga lembrança de tê"-la lido num de seus livros, mas dada como citação,
sem dúvida. Essa severidade de pensamento e de estilo combina com a
santidade majestosa de \indice{Bossuet}. Todavia, o estilo elíptico do
pensamento e a fineza quintessenciada me levariam antes a atribuir a
honra dessa citação a \indice{Bourdaloue}, o impiedoso psicólogo cristão. Essa
singular máxima ocorre"-me incessantemente desde que concebi o projeto
deste artigo, e eu quis me livrar dela para começar. Analisemos, com
efeito, essa curiosa proposição.

O Sábio, isto é, aquele que é animado pelo espírito do Senhor, aquele
que possui a prática do formulário divino, não ri, não se entrega ao
riso senão tremendo. O Sábio treme por ter rido; o Sábio teme o riso
assim como teme os espetáculos mundanos, a concupiscência. Ele se detém
à beira do riso assim como à beira da tentação. Há, portanto, segundo o
Sábio, uma certa contradição secreta entre seu caráter de sábio e o
caráter primordial do riso. Com efeito, para apenas roçar de passagem
lembranças mais do que solenes, ressaltarei --- o que corrobora de modo
perfeito o caráter oficialmente cristão dessa máxima --- que o Sábio por
excelência, o Verbo Encarnado, nunca riu. Aos olhos Daquele que tudo
sabe e que tudo pode, o cômico não existe. Entretanto, o Verbo
Encarnado conheceu a cólera, conheceu inclusive as lágrimas.

Assim, observemos bem o seguinte: em primeiro lugar, eis um autor --- um
cristão, sem dúvida --- que considera como certo que o Sábio examine de
bem perto antes de se permitir rir, como se disso tivesse que
permanecer nele não sei qual mal"-estar e qual inquietude, e, em segundo
lugar, o cômico desaparece do ponto de vista da ciência e da potência
absolutas. Ora, invertendo as duas proposições, resultaria disso que o
riso é geralmente o apanágio dos loucos, e que implica sempre mais ou
menos ignorância e fraqueza. Não quero em absoluto navegar
aventurosamente em um mar teológico, para o qual não estarei, com toda
certeza, munido de bússola nem de velas suficientes; contento"-me em
indicar ao leitor e apontar"-lhe esses horizontes singulares. É certo,
se se quiser estar de acordo com o espírito ortodoxo, que o riso humano
está intimamente ligado ao acidente de uma queda antiga, de uma
degradação física e moral. O riso e a dor exprimem"-se pelos órgãos onde
residem o comando e a ciência do bem ou do mal: os olhos e a boca. No
paraíso terrestre (que se o suponha passado ou futuro, lembrança ou
profecia, como os teólogos ou como os socialistas), no paraíso
terrestre, quer dizer, no meio onde parecia ao homem que todas as
coisas criadas eram boas, a alegria não se encontrava no riso. Visto
que nenhum sofrimento o afligia, seu rosto era simples e unido, e o
riso que agora agita as nações não deformava em absoluto as feições de
seu rosto. O riso e as lágrimas não podem se fazer ver no paraíso de
delícias. Eles são igualmente os filhos da aflição, e surgiram porque
faltava, ao corpo do homem enervado, força para contê"-los.\footnote{
Philippe de Chennevières [N.~do~A.].} Do ponto de vista de meu filósofo	\index{Chennevières@Chennevières, Philippe de|nn}
cristão, o riso de seus lábios é sinal de tão grande miséria quanto as
lágrimas de seus olhos. O Ser que quis multiplicar sua imagem não
colocou absolutamente na boca do homem os dentes do leão, todavia, o
homem morde com o riso; tampouco em seus olhos toda a astúcia
fascinante da serpente, contudo, ele seduz com as lágrimas. E observem
que também é com as lágrimas que o homem lava as aflições do homem, que
é com o riso que ele suaviza algumas vezes seu coração e o cativa; pois
os fenômenos engendrados pela queda tornar"-se"-ão os meios de redenção.

Que se me permita fazer uma suposição poética que me servirá para
verificar a exatidão dessas asserções, que muitas pessoas acharão sem
dúvida manchadas do \textit{a priori} do misticismo. Tentemos, visto
que o cômico é um elemento condenável e de origem diabólica, visualizar
uma alma absolutamente primitiva e saindo, por assim dizer, das mãos da
natureza. Tomemos por exemplo a grande e típica figura de
\indice{Virginie},\footnote{ \textit{ Paul et Virginie},\textit{ }publicado em
1787 [N.~do~T.].} que simboliza com perfeição a pureza e a ingenuidade
absolutas. \indice{Virginie} chega a \indice{Paris} ainda toda molhada das brumas do mar
e dourada pelo sol dos trópicos, os olhos cheios das grandes imagens
primitivas das ondas, das montanhas e das florestas. Cai, aqui, em
plena civilização turbulenta, expansiva e mefítica, ela, inteiramente
impregnada das puras e ricas fragrâncias da Índia; liga"-se à humanidade	\index{India@Índia}
pela família e pelo amor, por sua mãe e por seu amante, seu Paul,
angélico como ela, e cujo sexo não se distingue, por assim dizer, do
seu nos ardores insaciados de um amor que se ignora. Deus, ela o
conheceu na igreja dos Pamplemousses, uma pequena igreja bem modesta e
bem insignificante, e na imensidão do indescritível céu tropical, e na
música imortal das florestas e das torrentes. É verdade, Virginie é uma
grande inteligência; todavia, poucas imagens e poucas recordações lhe
bastam, assim como para o Sábio poucos livros. Ora, um dia, Virginie
encontra por acaso, inocentemente, no Palais"-Royal, nas vidraças de um	\index{Palais"-Royal}
vidreiro, sobre uma mesa, num local público, uma caricatura! Uma
caricatura bem atraente para nós, densa de fel e rancor, como sabe
fazê"-las uma civilização perspicaz e enfadada. Suponhamos alguma boa
farsa de boxeadores, alguma barbaridade britânica, cheia de sangue
coagulado e temperado com alguns monstruosos
\textit{goddam};\footnote{ Apelido dado outrora na França aos ingleses	\index{França|nn}
[N.~do~T.].} ou, se isso agrada mais à sua imaginação curiosa,
suponhamos diante dos olhos de nossa virginal Virginie alguma charmosa
e provocante impureza, um \indiceBA{Gavarni}{Paul} daquele tempo, e dos melhores, alguma
sátira insultante contra loucuras reais, alguma diatribe plástica
contra o \indice{Parc"-aux"-Cerfs}, ou os antecedentes abjetos de uma grande
favorita, ou as escapulidas noturnas da proverbial
%r índice nota Maria Antonieta
Austríaca.\footnote{ Alusão à rainha Maria Antonieta [N.~do~T.].} A	\index{Antonieta@Antonieta, Maria|nn}
caricatura é dupla: o desenho e a ideia; o desenho violento, a ideia
mordaz e velada; complicação de elementos penosos para um espírito
ingênuo, acostumado a compreender por intuição coisas simples como ele.
Virginie viu; agora observa. Por quê? Ela observa o desconhecido. Por
sinal, não compreende em absoluto o que isso quer dizer, nem para que
serve. Entretanto, vocês veem essa dobradura de asas súbita, esse
frêmito de uma alma que se vela e quer se retirar? O anjo sentiu que o
escândalo estava presente. E, na verdade, digo"-lhes, que ela tenha ou
não compreendido, ficar"-lhe"-á dessa impressão não sei qual mal"-estar,
algo que se assemelha ao medo. Sem dúvida que, se Virginie permanece em
\indice{Paris} e adquire experiência, o riso lhe chegará; veremos por quê.
Todavia, por enquanto, nós, analista e crítico, que não ousaríamos com
toda certeza afirmar que nossa inteligência é superior à de \indice{Virginie},
constatamos o temor e o sofrimento do anjo imaculado diante da
caricatura.

\sectionitem

A concordância unânime dos fisiologistas do riso sobre a principal razão
desse monstruoso fenômeno bastaria para demonstrar que o cômico é um
dos mais claros signos satânicos do homem e uma das inúmeras
complicações contidas na maçã simbólica. Por sinal, sua descoberta não
é muito profunda e não vai longe. O riso, dizem, vem da superioridade.
Eu não ficaria surpreso se diante dessa descoberta o fisiologista se
pusesse a rir pensando em sua própria superioridade. Da mesma forma,
era preciso dizer: o riso vem da ideia de sua própria superioridade.
Uma perfeita ideia satânica! Orgulho e aberração! Ora, é notório que
todos os loucos dos manicômios possuem a ideia de sua própria
superioridade desenvolvida em excesso. Eu não conheço em absoluto
loucos humildes. Observem que o riso é uma das expressões mais
frequentes e mais numerosas da loucura. E vejam como tudo se associa:
quando \indice{Virginie}, decaída, tiver baixado um grau em pureza, começará a
ter a ideia de sua própria superioridade, será mais sábia do ponto de
vista do mundo, e rirá.

Eu disse que havia sintoma de fraqueza no riso; e, com efeito, que sinal
mais marcante de debilidade do que uma convulsão nervosa, um espasmo
involuntário comparável à esternutação, e causado pela imagem da
desgraça alheia? Essa desgraça é algumas vezes de uma espécie muito
inferior, uma enfermidade na ordem física. Para tomar um dos exemplos
mais vulgares da vida, o que há de tão engraçado no espetáculo de um
homem que cai sobre o gelo ou na rua, que tropeça na beira de uma
calçada, para que o rosto de seu irmão em \indice{Jesus Cristo} se contraia de
um modo desordenado, para que os músculos de seu rosto comecem a
funcionar subitamente como um relógio ao meio"-dia ou um brinquedo de
molas? Esse pobre diabo no mínimo se desfigurou, talvez tenha fraturado
um membro essencial. Entretanto, o riso saiu, irresistível e súbito. É
certo que, se se quiser aprofundar essa situação, encontrar"-se"-á no
fundo do pensamento daquele que ri um certo orgulho inconsciente. Eis
aí o ponto de partida: \textit{eu }não caio; \textit{eu }caminho
direito; \textit{eu}, meu pé é firme e seguro. Não sou eu quem cometeria
a asneira de não enxergar uma calçada interrompida ou um paralelepípedo
que barra o caminho.

A escola romântica, melhor dizendo, uma das subdivisões da escola
romântica, a escola satânica, compreendeu muito bem essa lei primordial
do riso; ou pelo menos, se todos não a compreenderam, todos, mesmo em
suas mais grosseiras extravagâncias e exageros, a sentiram e a
aplicaram corretamente. Todos os ímpios de melodrama, malditos,
danados, fatalmente marcados por um ricto que chega até as orelhas,
estão na ortodoxia pura do riso. De resto, eles são quase todos netos
legítimos ou ilegítimos do célebre viajante Melmoth, a grande criação
satânica do reverendo \indiceBA{Maturin}{Charles Robert}. O que de maior, o que de mais poderoso
em relação à pobre humanidade do que esse pálido e entediado \indice{Melmoth}?
Todavia, há nele um lado fraco, abjeto, antidivino e antiluminoso.
Assim como ele ri, comparando"-se incessantemente às
lagartas humanas, ele tão forte, tão inteligente, para quem uma parte
das leis condicionais da humanidade, físicas e intelectuais, não
existem mais! E esse riso é a explosão perpétua de sua cólera e de seu
sofrimento. Ele é, que me compreendam bem, a resultante necessária de
sua dupla natureza contraditória, infinitamente grande em relação ao
homem, infinitamente vil e baixa em relação ao Verdadeiro e ao Justo
absolutos. \indice{Melmoth} é uma contradição viva. Saiu das condições
fundamentais da vida; seus órgãos não suportam mais seu pensamento. Eis
por que esse riso congela e revira as entranhas. É um riso que nunca
adormece, como uma doença que segue sempre seu caminho e executa uma
ordem providencial. E assim o riso de \indice{Melmoth}, que é a expressão mais
elevada do orgulho, realiza perpetuamente sua função, rasgando e
queimando os lábios do ridente irremissível.

\sectionitem

Agora, resumamos um pouco e estabeleçamos de modo mais visível as
principais proposições, que são como uma espécie de teoria do riso. O
riso é satânico, é, portanto, profundamente humano. Ele é no homem a
consequência da ideia de sua própria superioridade; e, com efeito, como
o riso é essencialmente humano, é essencialmente contraditório, quer
dizer, é ao mesmo tempo sinal de uma grandeza infinita e de uma miséria
infinita, miséria infinita em relação ao Ser Absoluto do qual ele
possui a concepção, grandeza infinita em relação aos animais. É do
choque perpétuo desses dois infinitos que o riso se libera. O cômico, a
potência do riso, se encontra no ridente e de forma alguma no objeto do
riso. Não é absolutamente o homem que cai que ri de sua própria queda,
a menos que seja um filósofo, um homem que tenha adquirido, por hábito,
a força de se desdobrar rapidamente e assistir como espectador
desinteressado aos fenômenos de seu eu. Mas o caso é raro. Os animais
mais cômicos são os mais sérios, como os macacos e os papagaios. Por
sinal, suponham o homem excluído da criação: não haverá mais o cômico,
pois os animais não se creem superiores aos vegetais, nem os vegetais
aos minerais. Sinal de superioridade em relação aos animais, e
entendendo sob essa denominação numerosos párias da inteligência, o
riso é sinal de inferioridade em relação aos sábios, que pela inocência
contemplativa de seu espírito se aproximam da infância. Comparando,
assim como temos o direito de fazê"-lo, a humanidade ao homem, vemos que
as nações primitivas, assim como \indice{Virginie}, não concebem a caricatura e
não possuem comédias (os livros sagrados, a quaisquer nações que
pertençam, nunca riem), e que, aproximando"-se pouco a pouco dos picos
nebulosos da inteligência, ou examinando as fornalhas tenebrosas da
metafísica, as nações põem"-se a rir diabolicamente do riso de \indice{Melmoth};
e, enfim, que se nessas mesmas nações ultracivilizadas, uma
inteligência, levada por uma ambição superior, quiser ultrapassar os
limites do orgulho mundano e se lançar ousadamente rumo à poesia pura,
nessa poesia, límpida e profunda como a natureza, o riso estará ausente
como na alma do Sábio.

Visto que o cômico é sinal de superioridade ou de crença em sua própria
superioridade, é natural acreditar que, antes de terem alcançado a
purificação absoluta por certos profetas místicos, as nações verão
aumentar nelas os motivos do cômico à medida que cresça sua
superioridade. Todavia, o cômico também muda de natureza. Assim, o
elemento angélico e o elemento diabólico atuam paralelamente. A
humanidade se eleva, e ela conquista para o mal e para a inteligência
do mal uma força proporcional à que conquistou para o bem. É por essa
razão que não acho surpreendente que nós, filhos de uma lei melhor que
as leis religiosas antigas, nós, discípulos favorecidos de Jesus\index{Jesus Cristo},
possuamos mais elementos cômicos do que a pagã Antiguidade. Isso mesmo
é uma condição de nossa força intelectual geral. Permitido aos
contraditores declarados citar a clássica historieta do filósofo que
morreu de rir ao ver um asno que comia figos, e mesmo as comédias de
\indice{Aristófanes} e as de \indice{Plauto}. Responderei que, além do fato de essas
épocas serem essencialmente civilizadas, e de a crença já haver se
retirado, esse cômico não é exatamente o nosso. Ele tem inclusive
alguma coisa de selvagem, e não podemos em absoluto nos apropriar dele
senão por um esforço de espírito por recuo, cujo resultado se chama
\textit{pastiche}. Quanto às figuras grotescas que a Antiguidade nos
deixou, as máscaras, as estatuetas de bronze, os Hércules musculosos,
os pequenos príapos de língua enrolada no ar, de orelhas pontudas,
todos em cerebelo e em falo --- quanto a esses falos prodigiosos sobre os
quais as brancas filhas de Rômulo\index{Romulo@Rômulo} cavalgam inocentemente, esses
monstruosos órgãos da reprodução munidos de sinetas e asas --- creio que
todas essas coisas são cheias de seriedade. \indicex{Vênus}{Venus}, \indicex{Pã}{Pa}, \indicex{Hércules}{Hercules}, não
eram personagens risíveis. Riu"-se deles depois da vinda de Jesus\index{Jesus Cristo}, com
\indicex{Platão}{Platao} e \indicex{Sêneca}{Seneca} contribuindo. Creio que a Antiguidade era cheia de
respeito pelos tambores"-mores e pelos feitores de façanhas em todos os
gêneros, e que todos os fetiches extravagantes que eu citava são apenas
signos de adoração, ou, quando muito, símbolos de força, e de forma
alguma emanações do espírito intencionalmente cômicas. Os ídolos
indianos e chineses ignoram que são ridículos; é em nós, cristãos, que
se encontra o cômico.

\sectionitem

Não se deve crer que estejamos livres de toda dificuldade. O espírito
menos acostumado a essas sutilezas estéticas poderia rapidamente me
opor essa objeção insidiosa: o riso é variado. Nem sempre se se
regozija de uma desgraça, de uma fraqueza, de uma inferioridade. Muitos
espetáculos que excitam em nós o riso são bastante inocentes, e não
somente as diversões da infância, mas ainda muitas outras coisas que
servem ao entretenimento dos artistas nada têm a ver com o espírito de
Satã.

Há nisso alguma aparência de verdade. Todavia, deve"-se inicialmente
distinguir a alegria do riso. A alegria existe por si mesma, mas ela
apresenta manifestações variadas. Algumas vezes, é quase invisível;
outras, exprime"-se pelas lágrimas. O riso não é outra coisa senão uma
expressão, um sintoma, um diagnóstico. Sintoma de quê? Eis a questão. A
alegria é \textit{una}. O riso é a expressão de um sentimento duplo, ou
contraditório; e é por isso que há convulsão. Também o riso das
crianças, que se se desejaria em vão me objetar, é completamente
diferente, mesmo como expressão física, como forma, do riso do homem
que assiste a uma comédia, observa uma caricatura, ou do riso terrível
de \indice{Melmoth}; de \indice{Melmoth}, o ser desclassificado, o indivíduo situado
entre os últimos limites da pátria humana e as fronteiras da vida
superior; de Melmoth imaginando"-se sempre prestes a se livrar de seu
pacto infernal, esperando incessantemente trocar esse poder
sobre"-humano, que provoca sua infelicidade, contra a consciência pura
de um ignorante que lhe causa inveja. O riso das crianças é como um
desabrochar da flor. É a alegria de receber, a alegria de respirar, a
alegria de se abrir, a alegria de contemplar, viver, crescer. É uma
alegria de planta. Assim, geralmente, trata"-se mais do sorriso, algo de
análogo ao balanço de rabo dos cães ou ao ronrom dos gatos. Entretanto,
observem bem que, se o riso das crianças difere ainda das expressões do
contentamento animais, é que esse riso não é inteiramente isento de
ambição, assim como convém a pedaços de homem, quer dizer, a Satãs em
formação.

Há um caso em que a questão é mais complicada. É o riso do homem, mas o
riso verdadeiro, riso violento, vendo objetos que não são um sinal de
fraqueza ou de desgraça entre seus semelhantes. É fácil adivinhar que
quero falar do riso causado pelo grotesco. As criações fabulosas, os
seres dos quais a razão, a legitimação, não pode ser extraída do código
do senso comum, excitam com frequência em nós uma hilaridade louca,
excessiva, e que se traduz em lacerações e esvaecimentos intermináveis.
É evidente que é preciso distinguir, e que há aí um grau a mais. O
cômico é, do ponto de vista artístico, uma imitação; o grotesco, uma
criação. O cômico é uma imitação mesclada de uma certa faculdade
criadora, quer dizer, de uma idealidade artística. Ora, o orgulho
humano, que sempre tem a preeminência, e que é a causa natural do riso
no caso do cômico, torna"-se também causa natural do riso no caso do
grotesco, que é uma criação mesclada de uma certa faculdade imitadora
de elementos preexistentes na natureza. Quero dizer que nesse caso o
riso é expressão da ideia de superioridade, não mais do homem sobre o
homem, mas do homem sobre a natureza. Não se deve achar essa ideia
muito sutil; não seria razão suficiente para rejeitá"-la. Trata"-se de
encontrar uma outra explicação plausível. Se esta parece extraída de
longe e um pouco difícil de admitir, é que o riso causado pelo grotesco
possui em si algo de profundo, de axiomático e primitivo, que se
aproxima muito mais da vida inocente e da alegria absoluta do que o
riso causado pela comédia de costumes. Há entre esses dois risos,
abstração feita da questão de utilidade, a mesma diferença que há entre
a escola literária interessada e a escola da arte pela arte. Assim, o
grotesco domina o cômico de uma altura proporcional.

Chamarei doravante o grotesco cômico absoluto, como antítese ao cômico
ordinário, que chamarei de cômico significativo. O cômico significativo
é uma linguagem mais clara, mas fácil de compreender pelo vulgo, e
sobretudo mais fácil de analisar; seu elemento era visivelmente duplo:
a arte e a ideia moral; entretanto, o cômico absoluto, aproximando"-se
muito mais da natureza, apresenta"-se sob uma espécie \textit{una}, e
que quer ser apreendida por intuição. Só há uma verificação do
grotesco, é o riso, e o riso súbito; diante do cômico significativo,
não é proibido rir \textit{a posteriori}; isso não infere contra seu
valor; é uma questão de rapidez de análise.

Eu disse: cômico absoluto; é preciso, todavia, tomar cuidado. Do ponto
de vista do absoluto definitivo, só resta a alegria. O cômico só pode
ser absoluto em relação à humanidade decaída, e é assim que o entendo.

\sectionitem

A essência muito nobre do cômico absoluto faz dele o apanágio dos
artistas superiores que possuem neles a receptibilidade suficiente de
toda ideia absoluta. Dessa forma, o homem que, até o presente momento,
melhor sentiu essas ideias, e que executou uma parte delas em trabalhos
de pura estética e também de criação, foi Théodore \indiceBA{Hoffmann}{Théodore}. Ele sempre
distinguiu muito bem o cômico ordinário do cômico que ele denomina
cômico inocente. Procurou com frequência resolver em obras artísticas
as sábias teorias que havia apresentado didaticamente, ou lançado sob a
forma de conversações inspiradas e de diálogos críticos; e é nessas
mesmas obras que irei buscar logo mais os exemplos mais
extraordinários, quando virei dar uma série de aplicações dos
princípios supramencionados e colar uma amostra sob cada título de
categoria.

Por sinal, encontramos no cômico absoluto e no cômico significativo
gêneros, subgêneros e famílias. A divisão pode ocorrer sobre diferentes
bases. Pode"-se construí"-la inicialmente segundo uma lei filosófica
pura, assim como comecei a fazê"-lo, em seguida, segundo a lei artística
de criação. A primeira é criada pela separação primitiva do cômico
absoluto do cômico significativo; a segunda tem por base o gênero de
faculdades especiais de cada artista. E, enfim, pode"-se também
estabelecer uma classificação de cômicos segundo os climas e as
diversas aptidões nacionais. Deve"-se observar que cada termo de cada
classificação pode se completar e se nuançar pela adjunção de um termo
de uma outra, como a lei gramatical nos ensina a modificar o
substantivo pelo adjetivo. Assim, tal artista alemão ou inglês é mais
ou menos próprio ao cômico absoluto, e ao mesmo tempo é mais ou menos
idealizador. Vou tentar dar exemplos escolhidos de cômico absoluto e
significativo, e caracterizar brevemente o espírito cômico próprio de
algumas nações sobretudo artistas, antes de chegar à parte em que
desejo discutir e analisar mais longamente o talento dos homens que
fizeram dele seu estudo e sua existência.

Exagerando e levando aos últimos limites as consequências do cômico
significativo, obtém"-se o cômico feroz, assim como a expressão
sinonímica do cômico inocente, com um grau a mais, é o cômico absoluto.
Na \indice{França}, país de pensamento e de demonstração claros, onde a arte
visa natural e diretamente à utilidade, o cômico é geralmente
significativo. \indice{Molière} foi nesse gênero a melhor expressão francesa;
todavia, como o fundo de nosso caráter é um distanciamento de toda
coisa extrema, como um dos diagnósticos particulares de toda paixão
francesa, de toda ciência, de toda arte francesa é fugir do excessivo,
do absoluto e do profundo, há aqui, em consequência, pouco cômico
feroz; da mesma forma nosso grotesco raramente se eleva ao absoluto.

\indiceBA{Rabelais}{François}, que é o grande mestre francês do grotesco, conserva no meio de
suas mais gigantescas fantasias algo de útil e racional. Ele é
diretamente simbólico. Seu cômico tem quase sempre a transparência de
um apólogo. Na caricatura francesa, na expressão plástica do cômico,
reencontraremos esse espírito dominante. É preciso confessá"-lo, o
prodigioso bom humor poético necessário ao verdadeiro grotesco
encontra"-se raramente entre nós em uma dose igual e contínua. De vez em
quando, vê"-se reaparecer o filão; mas ele não é essencialmente
nacional. É necessário mencionar nesse gênero alguns intermédios de
Molière, infelizmente muito pouco lidos e muito pouco encenados, entre
outros os de \textit{O doente imaginário }e de \textit{O burguês
gentil"-homem}, e as figuras carnavalescas de \indiceBA{Callot}{Jacques}. Quanto ao cômico
dos \textit{Contos }de \indice{Voltaire}, essencialmente francês, sempre extrai
sua razão de ser da ideia de superioridade; ele é completamente
significativo.

A sonhadora \indice{Germânia} nos dará excelentes amostras de cômico absoluto. Lá
tudo é grave, profundo, excessivo. Para encontrar o cômico feroz e
muito feroz, é preciso atravessar o \indiceAB{canal da}{Mancha} e
visitar os reinos brumosos do \textit{spleen}. A alegre, ruidosa e
descuidada Itália abunda em cômico inocente. É em plena \indice{Itália}, no
coração do carnaval meridional, no meio do turbulento Corso, que
Théodore \indiceBA{Hoffmann}{Théodore} situou de modo judicioso o drama excêntrico de
%r índice Brambilla
\textit{A princesa Brambilla}. Os espanhóis são muito bem"-dotados em
matéria de cômico. Chegam rapidamente ao cruel, e suas fantasias mais
grotescas contêm, amiúde, algo de sombrio.

Conservarei por muito tempo a lembrança da primeira pantomima inglesa
que vi representarem. Foi no \indice{Théâtre des Variétés}, há alguns anos.
Poucas pessoas dela se recordarão, sem dúvida, pois bem poucas
pareceram apreciar esse gênero de diversão, essas pobres mímicas
inglesas tiveram entre nós uma triste acolhida. O público francês não
gosta absolutamente de se sentir desorientado. Não tem o gosto muito
cosmopolita e as mudanças de horizonte lhe perturbam a vista. No que me
concerne, fui excessivamente surpreendido por essa maneira de
compreender o cômico. Dizia"-se, e eram os indulgentes, para explicar o
insucesso, que eram artistas vulgares e medíocres, dublês; todavia,
essa não era a questão. Eram ingleses, eis o importante.

Pareceu"-me que o traço distintivo desse gênero de cômico era a
violência. Darei a prova disso por algumas amostras de minhas
lembranças.

Inicialmente, o \indice{Pierrô} não era esse personagem pálido como a lua,
misterioso como o silêncio, leve e silencioso como a serpente, reto e
alto como uma forca, esse homem artificial, movido por mecanismos
singulares, ao qual o lamentável \indice{Debureau} nos havia acostumado. O
\indice{Pierrô} inglês chegava como a tempestade, caía como um imbecil e, quando
ele ria, seu riso fazia estremecer a sala; esse riso se assemelhava a
um radiante trovão. Era um homem baixo e gordo, tendo aumentado seu
garbo por um traje carregado de fitas, que funcionavam, em torno de sua
jubilosa pessoa, como as plumas e a penugem em torno dos pássaros, ou a
peliça em torno dos angorás. Por cima do pó de seu rosto, ele havia
colado cruamente, sem gradação, sem transição, duas enormes placas de
ruge puro. A boca tinha sido aumentada por um prolongamento simulado
dos lábios por meio de duas tiras de carmim, de modo que, quando ele
ria, a boca parecia correr até as orelhas.

Quanto à moral, o fundo era igual ao do \indice{Pierrô} que todos conhecem:
indiferença e neutralidade e, em consequência, realização de todas as
fantasias glutônicas e rapaces, em detrimento, ora de \indice{Arlequim}, ora de
\indice{Cassandra} ou de \indice{Leandro}. Entretanto, lá onde \indice{Debureau} mergulhara a
ponta do dedo para lambê"-lo, ele mergulhava os dois punhos e os dois
pés.

E todas as coisas se exprimiam assim, nessa singular peça, com
arrebatamento; era a vertigem da hipérbole.

\indice{Pierrô} passa diante de uma mulher que lava o chão de sua porta: depois
de ter"-lhe esvaziado os bolsos, quer fazer passar para os seus a
esponja, a vassoura, a tina e até mesmo a água. Quanto à maneira pela
qual ele tentava exprimir"-lhe seu amor, todos podem imaginá"-la pelas
lembranças conservadas da contemplação dos hábitos fanerogâmicos dos
macacos, na célebre jaula do \indice{Jardin"-des"-Plantes}. É preciso acrescentar
que o papel da mulher era representado por um homem muito alto e muito
magro, cujo pudor violado lançava altos brados. Era realmente uma
embriaguez de riso, algo de terrível e irresistível.

Por não sei qual crime, Pierrô devia ser finalmente guilhotinado. Por
que a guilhotina em vez da forca, em terra inglesa?\ldots Eu o ignoro; sem
dúvida para conduzir ao que se vai ver. O instrumento fúnebre estava,
portanto, lá, erguido sobre palcos franceses, muito espantados com essa
romântica novidade. Após ter lutado e mugido como um boi que fareja o
abatedouro, \indice{Pierrô} sofria, enfim, seu destino. A cabeça se separava do
pescoço, uma grande cabeça branca e vermelha, e rolava com barulho
diante da abertura do ponto, mostrando o disco sangrento do pescoço, a
vértebra cindida, e todos os detalhes de uma carne de açougue
recém"-cortada para a exposição. Mas eis que, de súbito, o torso
encurtado, movido pela monomania do roubo, se erguia, escamoteava
vitoriosamente sua própria cabeça como um presunto ou uma garrafa de
vinho e, bem mais prudente que o grande \indice{São Dionísio}, colocava"-a em
seu bolso!

Com a pena tudo isso é pálido e gélido. Como a pena poderia competir com
a pantomima? A pantomima é a depuração da comédia; é sua quintessência;
é o elemento cômico puro, liberado e concentrado. Por isso, com o
talento especial dos atores ingleses pela hipérbole, todas essas
monstruosas farsas adquiriam uma realidade singularmente surpreendente.

Uma das coisas mais extraordinárias como o cômico absoluto, e, por assim
dizer, como metafísica do cômico absoluto, era com certeza o começo
dessa bela peça, um prólogo repleto de uma elevada estética. Os
principais personagens da peça, \indice{Pierrô}, \indice{Cassandra}, \indice{Arlequim}, \indice{Colombina},
\indice{Leandro}, estão diante do público, bem dóceis e bem tranquilos. Eles são
\textit{grosso modo} racionais e não diferem muito das gentis pessoas
que estão na sala. A brisa maravilhosa que fará com que se movam
extraordinariamente ainda não soprou sobre seus cérebros. Algumas
infantilidades de \indice{Pierrô} podem dar apenas uma tênue ideia do que ele
fará mais tarde. A rivalidade entre \indice{Arlequim} e \indice{Leandro} acaba de se
manifestar. Uma fada se interessa por Arlequim: é a eterna protetora
dos mortais enamorados e pobres. Ela lhe promete proteção e, para lhe dar
uma prova imediata disso, movimenta com gesto misterioso e cheio de
autoridade sua varinha no ar.

Imediatamente surge a vertigem, a vertigem circula no ar; respira"-se a
vertigem; é a vertigem que enche os pulmões e renova o sangue no
ventrículo.

O que é essa vertigem? É o cômico absoluto; ele se apoderou de cada ser,
\indice{Leandro}, \indice{Pierrô}, \indice{Cassandra}, fazem gestos extraordinários, que
demonstram claramente que eles se sentem introduzidos à força em uma
nova existência. Não demonstram contrariedade por isso. Manifestam"-se
em relação aos grandes desastres e ao destino tumultuoso que os aguarda
como alguém que cospe em suas mãos e as esfrega uma na outra antes de
realizar uma ação extraordinária. Fazem com seus braços movimentos de
rotação, assemelham"-se a moinhos de vento agitados pela tempestade. É
sem dúvida para tornar flexíveis suas articulações, precisarão delas.
Tudo isso acontece com sonoras gargalhadas, repletas de um vasto
contentamento; em seguida, saltam uns por cima dos outros e, sua
agilidade e sua aptidão tendo sido devidamente constatadas, segue"-se um
deslumbrante buquê de pontapés, socos e tapas que fazem o barulho e
luminosidade de uma artilharia; mas tudo isso se dá sem rancor. Todos
os seus gestos, todos os seus gritos, todas as suas expressões dizem: a
fada o quis, o destino nos apressa, não me aflijo com isso; vamos!,
corramos!, lancemo"-nos! E eles se lançam através da obra fantástica,
que, para dizer a verdade, só começa aí, isto é, na fronteira do
maravilhoso.

\indice{Arlequim} e \indice{Colombina}, aproveitando esse delírio, fugiram dançando, e com
o passo ágil vão em busca das aventuras.

Mais um exemplo: este é extraído de um autor singular, espírito muito
geral, apesar do que dizem disso, e que une à zombaria significativa
francesa a hilaridade louca, suave e leve dos países do sol, ao mesmo
tempo que o profundo cômico germânico. Ainda quero falar de \indiceBA{Hoffmann}{Théodore}.

No conto intitulado \textit{Daucus Carota, o rei das cenouras}, e por
alguns tradutores \textit{A noiva do rei}, quando a grande tropa das
Cenouras chega ao terreno onde mora a noiva, nada é mais esplêndido
para se ver. Todos esses pequenos personagens de um vermelho escarlate
como um regimento inglês, como um vasto penacho verde sobre a cabeça
como criados de carruagem, executam piruetas e acrobacias maravilhosas
sobre pequenos cavalos. Tudo isso se move com uma agilidade
surpreendente. Eles são hábeis e lhes é fácil recair sobre a cabeça
visto que ela é maior e mais pesada do que o resto do corpo, como os
soldados em sabugo que têm um pouco de chumbo em sua barretina.

A infeliz jovem, alucinada por sonhos de grandeza, está fascinada por
essa exibição de forças militares. Entretanto, um exército em desfile é
diferente de um exército em suas casernas, polindo suas armas,
lustrando seu equipamento ou, pior ainda, roncando ignominiosamente
sobre suas camas de campanha fedorentas e sujas! Eis o reverso da
medalha; pois tudo isso nada mais era senão sortilégio, instrumento de
sedução. Seu pai, homem prudente e bem instruído na feitiçaria, quer
lhe mostrar o contrario de todos esses esplendores. Assim, enquanto os
legumes dormem um sono pesado, sem suspeitar de que possam ser
surpreendidos pelo olhar de um espião, o pai entreabre uma das tendas
desse magnífico exército; e então a pobre sonhadora vê essa massa de
soldados vermelhos e verdes em seu assombroso desnudamento, nadando e
dormindo na lama terrosa de onde saiu. Todo esse esplendor militar em
gorro de dormir não é mais do que um pântano infecto.

Eu poderia extrair de \indiceBA{Hoffmann}{Théodore} muitos outros exemplos de cômico
absoluto. Se se quiser compreender muito bem minhas ideias, é preciso
ler com cuidado \textit{Daucus Carota, Peregrinus Tyss, O pote de
ouro}, e principalmente, \textit{A princesa Brambilla}, que é como um
catecismo de elevada estética.

O que distingue muito particularmente \indiceBA{Hoffmann}{Théodore} é a mistura involuntária,
e algumas vezes muito voluntária, de certa dose de cômico
significativo com o cômico mais absoluto. Suas concepções cômicas mais
supranaturais, as mais fugidias, e que se parecem amiúde com visões da
embriaguez, têm um senso moral muito visível: é de crer que se está
diante de um fisiologista ou de um médico de doidos dos mais profundos,
e que se divertiria a revestir essa profunda ciência de formas
poéticas, como erudito que falaria por apólogos e parábolas.

Tomem, se vocês quiserem, por exemplo, o personagem de \indiceAB{Giglio}{Fava}, o
comediante acometido de dualismo crônico, em \textit{A princesa
Brambilla}. Esse personagem \textit{um }muda de vez em quando de
personalidade e, sob nome de \indiceAB{Giglio}{Fava}, declara"-se o inimigo do
príncipe assírio \indiceAB{Cornélio}{Chiapperi}; e, quando ele é príncipe assírio,
extravasa o mais profundo e o mais real desprezo sobre seu rival junto
à princesa, sobre um miserável histrião que se chama, ao que se diz,
\indiceAB{Giglio}{Fava}.

Deve"-se acrescentar que um dos sinais muito particulares do cômico
absoluto é ignorar"-se a si mesmo. Isso é visível não só em certos
animais do cômico, dos quais a gravidade faz parte essencial, como os
macacos, e em certas caricaturas esculturais antigas das quais já
falei, mais ainda nas monstruosidades chinesas que tanto nos divertem,
e que têm muito menos intenções cômicas do que geralmente se crê. Um
ídolo chinês, ainda que seja um objeto de veneração, não difere
absolutamente de um \textit{poussah}\footnote{ Busto de um homem gordo
ou representação popular de Buda [N.~do~T.].}\textit{ }ou de um		\index{Buda|nn}
\textit{magot}\footnote{ Figura atarracada do Extremo Oriente, em
porcelana, pedra ou jade [N.~do~T.].} de chaminé.

Assim, para terminar com todas essas sutilezas e todas essas definições,
e para concluir, observarei uma última vez que reencontramos a ideia
dominante de superioridade no cômico absoluto como no cômico
significativo, assim como expliquei, exaustivamente, talvez; que, para
que haja cômico, isto é, emanação, explosão, liberação do cômico, é
necessário haver dois seres cara a cara; que é especialmente no
ridente, no espectador, que jaz o cômico; que, entretanto, em relação a
essa lei de ignorância, deve"-se fazer uma exceção para os homens que
fizeram ofício de desenvolver neles o sentimento do cômico e de
extraí"-lo deles próprios para o divertimento de seus semelhantes, cujo
fenômeno entra na classe de todos os fenômenos artísticos que denotam
no ser humano a existência de uma dualidade permanente, o poder de ser
simultaneamente ele mesmo e um outro.

E, para retornar às minhas primeiras definições e me exprimir mais
claramente, digo que, quando \indiceBA{Hoffmann}{Théodore} engendra o cômico absoluto, é bem
verdade que ele o conhece; mas também sabe que a essência desse cômico
é parecer ignorar"-se a si mesmo e desenvolver no espectador, ou melhor,
no leitor, a alegria de sua própria superioridade e a alegria da
superioridade do homem sobre a natureza. Os artistas criam o cômico;
tendo estudado e reunido os elementos do cômico, sabem que tal ser é
cômico, e que só o é sob a condição de ignorar sua natureza; da mesma
forma, por uma lei inversa, o artista só é artista sob a condição de
ser duplo e de não ignorar nenhum fenômeno de sua dupla natureza.


\chapter[Alguns caricaturistas estrangeiros]{alguns
caricaturistas\break estrangeiros\subtitulo{\indiceBA{Hogarth}{William} ---
\indiceBA{Cruikshank}{Georges} --- \break \indiceBA{Goya}{Francisco José de} ---
\indiceBA{Pinelli}{Bartolomeo} --- \indiceBA{Brueghel}{Pieter}}}

\vspace{-1cm}

\sectionitem

\noindent\textsc{Um nome completamente} popular, não só entre os artistas, mas também
entre as pessoas do mundo, um artista dos mais eminentes em matéria do
cômico, e que preenche a memória como um provérbio, é \indiceBA{Hogarth}{William}.
Frequentemente ouvi dizer de Hogarth: “É o enterro do cômico”. Estou de
acordo; a expressão pode ser tomada por maliciosa, mas desejo que ela
seja compreendida como elogio; extraio dessa fórmula malevolente o
sintoma, o diagnóstico de um mérito bem particular. Com efeito, se
atentarmos para isso verificamos que o talento de \indiceBA{Hogarth}{William} comporta em
si algo de frio, adstringente, fúnebre. Isso oprime o coração. Brutal e
violento, mas sempre preocupado com o senso moral de suas composições,
moralista antes de tudo, ele as carrega, como nosso \indiceBA{Grandville}{(Jean"-Ignace"-Isidore Gérard)}, de
detalhes alegóricos e alusivos, cuja função, segundo ele, é completar e
elucidar seu pensamento. Para o espectador, ia dizer, creio, para o
leitor, ocorre algumas vezes, contra o seu desejo, que elas retardem a
inteligência e a perturbem.

Por sinal, \indiceBA{Hogarth}{William} possui, como todos os artistas que pesquisam muito,
estilos e trechos bastante variados. Seu procedimento nem sempre é
assim tão duro, tão literal, tão minucioso. Por exemplo, se compararmos
as pranchas do \textit{Casamento à moda} com aquelas que representam
\textit{Os perigos e as consequências da incontinência, O palácio do
gim, O suplício do músico, O poeta em sua casa,} reconhecer"-se"-á nessas
últimas muito mais desembaraço e abandono. Uma das mais curiosas é
certamente aquela que nos mostra um cadáver prostrado, rígido e
estendido sobre a mesa de dissecação. Sobre uma polia ou qualquer outra
mecânica presa ao teto dobram"-se os intestinos do morto corrompido.
Esse morto é horrível, e nada pode fazer um contraste mais singular com
esse cadáver, cadavérico entre todos, do que as altas, longas, magras
ou rotundas figuras, grotescamente graves, de todos esses doutores
britânicos, carregadas de monstruosas perucas à inglesa. Num canto, um
cão mergulha avidamente seu focinho num balde e de lá pilha alguns
restos humanos. \indiceBA{Hogarth}{William}, o enterro do cômico! Eu preferiria dizer que é
cômico no enterro. Esse cão antropófago fez"-me sempre lembrar do porco
histórico que se saciava impudentemente com o sangue do desafortunado
Fualdès, enquanto um realejo executava, por assim dizer, o serviço	\index{Fualdès}
fúnebre do moribundo.

Afirmei, há pouco, que o bom termo de ateliê devia ser tomado como um
elogio. Com efeito, encontro em \indiceBA{Hogarth}{William} esse não sei quê de sinistro,
de violento e de resoluto, que se manifesta em quase todas as obras do
país do \textit{spleen}. Em \textit{O palácio do gim}, ao lado das
desventuras inumeráveis e dos acidentes grotescos dos quais são
semeadas a vida e a estrada dos bêbados, encontramos casos terríveis
que são pouco cômicos do nosso ponto de vista francês: quase sempre
casos de morte violenta. Não quero fazer aqui uma análise detalhada das
obras de \indiceBA{Hogarth}{William}; inúmeras apreciações já foram feitas do singular e
minucioso moralista, e quero me limitar a constatar o caráter geral que
domina as obras de cada artista importante.

Seria injusto, ao falar da \indice{Inglaterra}, não mencionar \indice{Seymour}, do qual
todo mundo viu as admiráveis caricaturas sobre a pesca e a caça, dupla
epopeia de maníacos. Foi dele que se tomou emprestado primitivamente
essa maravilhosa alegoria da aranha que teceu sua teia entre a linha e
o braço desse pescador que a impaciência nunca faz tremer.

Em \indice{Seymour}, como nos outros ingleses, violência e amor pelo exagero;
maneira simples, arquibrutal e direta, de apresentar o tema. Em matéria
de caricatura, os ingleses são radicais. \textit{Oh! the deep, deep
sea!} exclama numa beata contemplação, tranquilamente sentado sobre o
banco de um bote, um gordo londrino, a um quarto de légua do porto.
Creio inclusive que se percebem ainda alguns telhados ao fundo. O
êxtase desse imbecil é extremo; por isso, ele não vê as duas gordas
pernas de sua querida esposa, que ultrapassam a água e se mantêm retas,
as extremidades no ar. Parece que essa gorda pessoa deixou"-se cair, a
cabeça por primeiro, no líquido elemento cujo aspecto entusiasma esse
pesado cérebro. Dessa infeliz criatura as pernas são tudo o que se vê.
Logo mais esse poderoso amante da natureza procurara fleumaticamente
sua mulher e não a encontrará mais.

O mérito especial de \indiceAB{Georges}{Cruikshank} (faço abstração de todos os seus
outros méritos, fineza de expressão, apreensão do fantástico etc.) é
uma abundância inesgotável no grotesco. Essa verve é inconcebível, e
seria considerada impossível se as provas não estivessem lá sob a forma
de uma obra imensa, coleção inumerável de vinhetas, longa série de
álbuns cômicos, enfim, de tal quantidade de personagens, situações,
fisionomias, quadros grotescos, que a memória do observador se perde
nele; o grotesco flui incessante e inevitavelmente da ponta de
Cruikshank assim como as rimas ricas da pena dos poetas naturais. O
grotesco é seu hábito.

Se se pudesse analisar de modo seguro uma coisa tão fugaz e impalpável
quanto o sentimento na arte, esse não sei quê que distingue sempre um
artista de um outro, por mais íntimo que seja na aparência seu
parentesco, direi que o que constitui principalmente o grotesco de
Cruikshank é a violência extravagante do gesto e do movimento, e a
explosão na expressão. Todos os seus pequenos personagens mimam com
furor e turbulência como atores de pantomima. O único defeito que se
lhe possa censurar é o de ser com frequência mais homem de espírito,
mais rabiscador do que artista, enfim, de nem sempre desenhar de uma
maneira bastante conscienciosa. Dir"-se"-ia que, no prazer que ele
ressente em se entregar à sua prodigiosa verve, o autor esquece de
dotar seus personagens de uma vitalidade suficiente. Desenha um pouco
como os homens de letras que se divertem em rabiscar croqui. Essas
prestigiosas pequenas criaturas nem sempre nasceram viáveis. Todo esse
mundo minúsculo se revira, se agita e se mescla com uma petulância
indizível, sem se inquietar muito se todos os seus membros estão bem em
seu lugar natural. Com muita frequência são apenas hipóteses humanas
que se debatem como podem. Enfim, tal como é, Cruikshank é um artista	\index{Cruikshank@Cruikshank, Georges}
dotado de ricas faculdades cômicas, e que permanecerá em todas as
coleções. Mas o que dizer desses plagiadores franceses modernos,
impertinentes até o ponto de se apropriarem não só dos temas e dos
\textit{canevas}, mas até mesmo da maneira e do estilo? Felizmente a
ingenuidade não pode ser roubada. Eles conseguiram ser insensíveis em
sua infantilidade afetada, e desenham de um modo ainda mais
insuficiente.

\sectionitem

Na \indice{Espanha}, um homem singular abriu novos horizontes no cômico.

A propósito de \indiceBA{Goya}{Francisco José de}, devo inicialmente conduzir meus leitores ao
excelente artigo que \indiceAB{Théophile}{Gautier} escreveu sobre ele em \textit{Le
cabinet de l’amateur}, e que foi depois reproduzido numa antologia.
\indiceAB{Théophile}{Gautier} é perfeitamente dotado para compreender semelhantes
naturezas. Por sinal, em relação às técnicas de \indiceBA{Goya}{Francisco José de} --- água"-tinta e
água"-forte misturadas, com retoques a ponta seca ---, o artigo em questão
contém tudo o que é preciso. Quero apenas acrescentar algumas palavras
sobre o elemento muito raro que \indiceBA{Goya}{Francisco José de} introduziu no cômico: quero falar
do fantástico. \indiceBA{Goya}{Francisco José de} não é precisamente nada de especial, de particular,
nem cômico absoluto, nem cômico puramente significativo, à maneira
francesa. Sem dúvida, mergulha com frequência no cômico feroz e se
eleva até o cômico absoluto; todavia, o aspecto geral sob o qual vê as
coisas é sobretudo fantástico, ou melhor, o olhar que lança sobre as
coisas é um tradutor naturalmente fantástico. \textit{Os caprichos} é
uma obra maravilhosa não só pela originalidade das concepções, mas
também pela execução. Imagino diante de \textit{Os caprichos} um homem,
um curioso, um amante, não tendo nenhuma noção dos fatos históricos aos
quais várias dessas pranchas fazem alusão, um simples espírito de
artista que não saiba quem é \indice{Godói}, nem o rei Carlos, nem a rainha;
experimentará, todavia, no fundo do seu cérebro, uma viva comoção por
causa da maneira original, da plenitude e da certeza dos meios do
artista, e também dessa atmosfera fantástica que envolve todos os seus
temas. De resto, há nas obras surgidas das profundas individualidades
algo que se assemelha a esses sonhos periódicos ou crônicos que
assediam regularmente nosso sono. É isso que marca o verdadeiro
artista, sempre durável e vivaz, inclusive nessas obras fugazes, por
assim dizer, suspensas nos acontecimentos, denominadas caricaturas; é
isso, eu dizia, o que distingue os caricaturistas históricos dos
caricaturistas artísticos, o cômico fugaz do cômico eterno.

\indiceBA{Goya}{Francisco José de} é sempre um grande artista, com frequência, assustador. Ele une à
graça, à jovialidade, à sátira espanhola do bom tempo de \indiceBA{Cervantes}{Miguel de}, um
espírito bem mais moderno ou, pelo menos, que foi bem mais escrutado
nos tempos modernos, o amor pelo inapreensível, o sentimento pelos
contrastes violentos, pelos espantos da natureza e pelas fisionomias
humanas estranhamente animalizadas pelas circunstâncias. É algo curioso
de observar que esse espírito que surge após o grande movimento
satírico e demolidor do século \textsc{xviii} e com quem \indice{Voltaire} se mostraria
satisfeito, pela ideia apenas (pois o pobre grande homem mal se
conhecia quanto ao resto), de todas essas caricaturas monacais --- monges
bocejantes, monges glutões, feições quadradas de assassinos se
preparando para as matinas, feições ardilosas, hipócritas, finas e
perversas como perfis de aves de rapina ---; é curioso, eu dizia, que
esse que odiava monges tenha sonhado tanto com bruxas, sabá, bruxarias,
crianças assadas no espeto, sei lá o quê, todas as orgias do sonho,
todas as hipérboles da alucinação, e depois todas essas brancas e
esbeltas espanholas que velhas sempiternas lavam e preparam seja para o
sabá, seja para a prostituição da noite, sabá da civilização! A luz e
as trevas se divertem através de todos esses grotescos horrores. Que
singular jovialidade! Lembro"-me principalmente de duas pranchas
extraordinárias: uma representa uma paisagem fantástica, uma mistura de
nuvens e rochedos. Trata"-se de um canto de \textit{sierra} desconhecida e não
frequentada? Uma amostra do caos? Lá, no seio desse teatro abominável
ocorre uma batalha encarniçada entre duas bruxas suspensas nos ares.
Uma cavalga sobre a outra; ela a espanca, subjuga"-a. Esses dois
monstros rolam pelo ar tenebroso. Toda a hediondez, toda a sordidez
moral, todos os vícios que o espírito humano pode conceber estão
escritos sobre essas duas faces, que, segundo um hábito frequente e um
procedimento inexplicável do artista, estão entre o homem e a fera.

A outra prancha representa um ser, um infeliz, uma mônada solitária e
desesperada, que quer sair de seu túmulo. Demônios
malfazejos, uma miríade de vis gnomos liliputianos fazem pressão com
todos os seus esforços reunidos sobre a tampa do túmulo entreaberto.
Esses guardiães vigilantes da morte se coligaram contra a alma
recalcitrante que se consome em luta impossível. Esse pesadelo se
agita no horror do vago e do infinito.

No final de sua carreira, os olhos de \indiceBA{Goya}{Francisco José de} se enfraqueceram a ponto de
ser preciso, segundo dizem, que apontassem seus lápis. Todavia, mesmo
nessa época fez grandes litografias muito importantes, entre outras,
touradas cheias de multidões e de efervescência, pranchas admiráveis,
vastos quadros em miniatura --- novas provas para sustentar essa lei
singular que preside o destino dos grandes artistas e que quer que, a
vida se governando ao inverso da inteligência, eles ganhem de um lado o
que perdem do outro e vão assim, segundo uma juventude progressiva,
reforçando"-se, revigorando"-se e crescendo em audácia até a beira do
túmulo.

No primeiro plano de uma dessas imagens, onde reinam um tumulto e uma
confusão admiráveis, um touro furioso, um desses vingativos que se
lançam ferozmente sobre os mortos, rasgou o fundilho da calça de um dos
combatentes. Este, que está apenas ferido, arrasta"-se pesadamente sobre
os joelhos. A formidável fera levantou com seus chifres a camisa
lacerada e pôs à mostra as nádegas do infeliz, e baixa de novo seu
focinho ameaçador; todavia, essa indecência na carnificina não comove
absolutamente a assembleia.

O grande mérito de \indiceBA{Goya}{Francisco José de} consiste em criar a monstruosa verossimilhança.
Seus monstros nasceram viáveis, harmônicos. Ninguém ousou mais do que
ele no sentido do absurdo possível, todas essas contorções, esses
rostos bestiais, essas caretas diabólicas estão penetradas de
humanidade. Mesmo do ponto de vista particular da história natural,
seria difícil condená"-los de tanto que há analogia e harmonia em todas
as partes de seu ser; numa palavra, a linha de sutura, o ponto de
junção entre o real e o fantástico é impossível de determinar; é uma
fronteira vaga que a análise mais sutil não poderia traçar de tanto que
a arte é simultaneamente transcendente e natural.\footnote{ Possuíamos, 
há alguns anos, várias pinturas preciosas de Goya,
relegadas infelizmente a cantos obscuros da galeria; elas desapareceram
com o Museu espanhol [N.~do~A.].}

\sectionitem

O clima da \indice{Itália}, por mais meridional que seja, não é o da \indice{Espanha}, e a
fermentação do cômico não dá, lá, os mesmos resultados. O pedantismo
italiano (sirvo"-me desse termo na falta de um mais apropriado)
encontrou sua expressão nas caricaturas de \indiceAB{Leonardo}{da Vinci} e nas
cenas de costumes de \indiceBA{Pinelli}{Bartolomeo}. Todos os artistas conhecem as caricaturas
de \indiceAB{Leonardo}{da Vinci}, verdadeiros retratos. Hediondas e frias, a essas
caricaturas não faltam crueldade, mas lhes falta o cômico; nada de
expansão, nada de abandono; o grande artista não se divertia ao
desenhá"-las, ele as fez como cientista, como geômetra, como professor
de história natural. Evitou omitir a mínima verruga, o menor fio de
cabelo. Talvez, em suma, não tivesse a pretensão de fazer caricaturas.
Procurou em torno dele tipos de feiura excêntricas e as copiou.

Entretanto, tal não é, em geral, o caráter italiano. O chiste está em
baixa, mas é franco. Os quadros de Bassan\index{Bassan (Jacopo da Ponte)} que representam o carnaval de
\indice{Veneza} nos dão uma ideia exata disso. Esse bom humor transborda de
salames, presuntos e macarrão. Uma vez por ano, o cômico italiano
explode no Corso e lá alcança os limites do furor. Todo mundo é
espirituoso, todos se tornam artistas cômicos; \indice{Marselha} e \indice{Bordeaux}
talvez pudessem nos dar amostras desses temperamentos. Deve"-se ver, em
\textit{A princesa Brambilla}, como \indiceBA{Hoffmann}{Théodore} compreendeu tão bem o
caráter italiano, e como os artistas alemães que bebem no \indice{Café Greco}
falam disso delicadamente. Os artistas italianos são mais bufões do que
cômicos. Falta"-lhes profundidade, todavia, todos eles sofrem da franca
embriaguez do bom humor nacional. Materialista, como é geralmente o
\textit{Midi},\footnote{ O sul de um país, região meridional [N.~do
T.].}\textit{ }seu chiste está sempre relacionado com a cozinha e o
local mal"-afamado. Em resumo, é um artista francês, é \indiceBA{Callot}{Jacques}, quem,
pela concentração de espírito e pela firmeza de vontade próprias de
nosso país, deu a esse gênero cômico sua mais bela expressão. É um
francês que permaneceu o melhor bufão italiano.

Falei há pouco de \indiceBA{Pinelli}{Bartolomeo}, que é agora uma glória bem reduzida. Não
diremos dele que se trata precisamente de um caricaturista; trata"-se
mais de um devorador\footnote{ No original \textit{croqueur }[N.~do~T.].} 
de cenas pitorescas. Só o menciono porque minha juventude foi
fatigada de ouvir louvores a ele como sendo o tipo do caricaturista
nobre. Na verdade, o cômico só entra nisso tudo numa quantidade
infinitesimal. Em todos os estudos desse artista encontramos uma
preocupação constante da linha e das composições antigas, uma aspiração
sistemática ao estilo.

Mas \indiceBA{Pinelli}{Bartolomeo} --- o que sem dúvida muito contribuiu para sua reputação ---
teve uma existência muito mais romântica do que seu talento. Sua
originalidade manifestou"-se bem mais em seu caráter do que em suas
obras; pois foi um dos tipos mais completos do artista, tal como o
imaginam os bons burgueses, isto é, da desordem clássica, da inspiração
exprimindo"-se pelo mau comportamento e pelo costumes violentos. \indiceBA{Pinelli}{Bartolomeo}
possuía todo o charlatanismo de alguns artistas: seus dois cães enormes
que o acompanhavam em todos os lugares como confidentes e camaradas,
sua grossa bengala nodosa, seus cabelos em tranças que escorriam ao
longo de suas faces, o cabaré, a má companhia, a atitude de destruir
faustosamente as obras às quais não lhe ofereciam um preço
satisfatório, tudo isso fazia parte de sua reputação. O lar de \indiceBA{Pinelli}{Bartolomeo}
não era em absoluto mais bem ordenado do que a conduta do chefe da
casa. Algumas vezes, ao voltar para casa, encontrava sua mulher e sua
filha arrancando"-se os cabelos, os olhos esbugalhados, em toda
excitação e fúria italianas. \indiceBA{Pinelli}{Bartolomeo} achava isso extraordinário:
“Parem!”, gritava para elas, “não se mexam, permaneçam assim!”. E o
drama se metamorfoseava num desenho. Vê"-se que \indiceBA{Pinelli}{Bartolomeo} era da raça dos
artistas que passeiam pela natureza material para que ela venha em
auxílio à preguiça de seu espírito, sem prontos \textit{a empunhar seus
pincéis}. Ele se aproxima, assim, por um lado, do infeliz \indiceAB{Léopold}{Robert}, 
que também sustentava encontrar na natureza, e somente na
natureza, esses temas prontos, que, para artistas mais imaginativos, só
têm um valor de observação. Mesmo esses temas, inclusive os mais
nacionalmente cômicos e pitorescos, para Pinelli assim como para
\indiceAB{Léopold}{Robert}, sempre passaram pelo crivo, pelo tamis implacável do
gosto.

\indiceBA{Pinelli}{Bartolomeo} foi caluniado? Eu o ignoro, mas tal é a lenda. Ora, tudo isso me
parece sinal de fraqueza. Eu gostaria que se criasse um neologismo, que
se fabricasse uma palavra destinada a censurar esse gênero de clichê,
clichê na aparência e na conduta, que se introduz na vida dos artistas
assim como em suas obras. Por sinal, observo que o contrário se
apresenta com frequência na história, e que os artistas mais
inventivos, os mais surpreendentes, os mais excêntricos em suas
concepções, são amiúde homens cuja vida é calma e minuciosamente
ordenada. Vários dentre eles tiveram as virtudes de vida familiar muito
desenvolvidas. Você não observou com frequência que nada se assemelha
mais ao perfeito burguês do que o artista de gênio concentrado?

\sectionitem

Os flamengos e holandeses fizeram, desde o início, coisas belíssimas, de
um caráter verdadeiramente especial e autóctone. Todo mundo conhece as
antigas e singulares produções de Brueghel,\footnote{ Também conhecido
como \indiceBA{Brueghel}{Pieter}, o Velho [N.~do~T.].} o Engraçado, que não deve ser
confundido, assim como o fizeram vários escritores, com
Brueghel\footnote{ Também conhecido como \indiceBA{Brueghel}{Pieter}, o Jovem [N.~do~T.].} de Inferno. Que haja nisso uma certa sistematização, um
\textit{parti pris} de excentricidade, um método no bizarro, não é de
duvidar. Todavia, também é bem certo que esse estranho talento tem uma
origem mais elevada do que uma espécie de aposta artística. Nos quadros
fantásticos de \indiceBA{Brueghel}{Pieter}, o Engraçado, mostra"-se toda a força da
alucinação. Que artista poderia compor obras tão monstruosamente
paradoxais, se não fosse arrebatado desde o princípio por alguma força
desconhecida? Na arte, uma coisa que não é bastante observada, a parte
deixada à vontade do homem é bem menor do que se poderia crer. Há no
ideal barroco que \indiceBA{Brueghel}{Pieter} parece ter perseguido muitas relações com o
de \indiceBA{Grandville}{(Jean"-Ignace"-Isidore Gérard)}, principalmente se quisermos examinar bem as
tendências que o artista francês manifestou nos últimos anos de sua
vida: visões de um cérebro doente, alucinações da febre, mudanças
repentinas e totais do sonho, associações bizarras de ideias,
combinações de formas fortuitas e heteróclitas.

As obras de \indiceBA{Brueghel}{Pieter}, o Engraçado, podem se dividir em duas classes: uma
contém alegorias políticas quase indecifráveis hoje; é nessa série que
encontramos casas cujas janelas são olhos, moinhos cujas asas são
braços, e mil composições assustadoras onde a natureza é
incessantemente transformada em logogrifo. Além de tudo, bem amiúde, é
impossível distinguir se esse gênero de composição pertence às classes
dos desenhos políticos e alegóricos, ou à segunda classe, que é
evidentemente a mais curiosa. Esta, que nosso século, para o qual nada
é difícil de explicar, graças a seu duplo caráter de incredulidade e de
ignorância, qualificaria simplesmente de fantasias e caprichos, contém,
segundo me parece, uma espécie de \textit{mistério}. Os últimos
trabalhos de alguns médicos, que finalmente entreviram a necessidade de
explicar um grande número de fatos históricos e milagrosos de outra
maneira que não pelos meios cômodos da escola voltairiana, a qual só
via, em todos os lugares, a habilidade da impostura, ainda não
esclareceram todos os arcanos psíquicos. Ora, desafio que se explique o
cafarnaum diabólico e divertido de \indiceBA{Brueghel}{Pieter}, o Engraçado, de outra
forma senão por uma espécie de graça especial e satânica. Ao termo
graça especial substitua, se quiser, o termo loucura, ou alucinação;
mas o mistério permanecerá quase tão negro. A coleção de todas essas
peças espalha um contágio; os gracejos de \indiceBA{Brueghel}{Pieter}, o Engraçado, dão
vertigem. Como uma inteligência humana pôde conter tantas diabruras e
maravilhas, engendrar e descrever tantos absurdos assustadores? Não
posso compreendê"-lo nem determinar positivamente sua razão; todavia,
encontramos amiúde na história, e inclusive em algumas partes modernas
da história, a prova do imenso poder dos contágios, do envenenamento
pela atmosfera moral, e não posso me impedir de observar (mas sem
afetação, sem pedantismo, sem intenção positiva, como para provar que
\indiceBA{Brueghel}{Pieter} pôde ver o diabo em pessoa) que esta prodigiosa floração de
monstruosidades coincide da maneira mais singular com a famosa e
histórica \textit{epidemia dos bruxos}.

\chapter[A arte filosófica]{a arte filosófica}

\noindent\textsc{O que é} a arte pura segundo a concepção moderna? É criar uma magia
sugestiva contendo ao mesmo tempo o objeto e o sujeito, o mundo
exterior ao artista e o próprio artista.

O que é a arte filosófica segundo a concepção de \indiceBA{Chenavard}{Paul} e da escola
alemã? É uma arte plástica que tem a pretensão de substituir o livro,
quer dizer, rivalizar com a arte de imprimir para ensinar a história, a
moral e a filosofia.

Há, com efeito, épocas da história em que a arte plástica está destinada
a pintar os arquivos históricos de um povo e suas crenças religiosas.

Entretanto, há vários séculos, deu"-se na história da arte como que uma
separação cada vez mais marcada pelos poderes, há temas que pertencem à
pintura, outros à música, outros ainda à literatura.

Será por uma fatalidade das decadências que hoje cada arte manifesta a
vontade de invadir a arte vizinha e que os pintores introduzem gamas
musicais na pintura, os escultores, cor na escultura, os literatos,
meios plásticos na literatura, e outros artistas, aqueles dos quais
vamos nos ocupar hoje, um tipo de filosofia enciclopédica na própria
arte plástica?

Toda boa escultura, toda boa pintura, toda boa música, sugere os
sentimentos e os devaneios que ela quer sugerir.

Mas o raciocínio, a dedução, pertencem ao livro.

Assim, a arte filosófica é um retorno à \textit{imagerie}\footnote{ Conjunto 
de imagens de mesma inspiração [N.~do~T.].}\textit{
}necessária à infância dos povos, e, se fosse rigorosamente fiel a si
mesma, ela se obrigaria a justapor tantas imagens sucessivas quantas
estão contidas numa frase qualquer que ela quisesse exprimir.

Temos ainda o direito de duvidar que a frase hieroglífica foi mais clara
do que a frase tipografada.

Estudaremos, portanto, a arte filosófica como uma monstruosidade em que
se mostraram belos talentos.

Observemos ainda que a arte filosófica supõe um absurdo para legitimar
sua razão de existência, isto é, a inteligência do povo em relação às
belas"-artes.

Quanto mais a arte quiser ser filosoficamente clara, mais ela se
degradará e remontará ao hieróglifo infantil; ao contrário, quanto mais
a arte se destacar do ensinamento, mais ascenderá à beleza pura e
desinteressada.

A \indice{Alemanha}, conforme se sabe e como seria fácil adivinhá"-lo se não se
soubesse, é o país que mais incidiu no erro da arte filosófica.

Deixaremos de lado temas bem conhecidos, por exemplo, \indiceBA{Overbeck}{Johann Friedrich} estudando
a beleza no passado apenas para melhor ensinar a religião; \indice{Cornélius} e
\indiceBA{Kaulbach}{Wilhelm von}, para ensinar a história e a filosofia (ainda observaremos que
\indiceBA{Kaulbach}{Wilhelm von}, tendo de tratar um tema puramente pitoresco, a \textit{Casa
dos loucos}, não pôde se impedir de tratá"-lo por categorias e, por
assim dizer, de uma maneira aristotélica, de tanto que é indestrutível
a antinomia entre o espírito poético puro e o espírito didático).

Nós nos ocuparemos hoje, como primeira amostra da arte filosófica, de um
artista alemão muito menos conhecido, mas que, segundo nossa opinião,
era infinitamente mais bem"-dotado do ponto de vista da arte pura.
Refiro"-me a Alfred Réthel\index{Rethel@Réthel, Alfred}, morto louco, há pouco tempo, depois de ter
decorado uma capela às margens do \indice{Reno}, e que só é conhecido em \indice{Paris}
por oito estampas gravadas sobre madeira, cujas duas últimas foram
exibidas na \indice{Exposição Universal}.

O primeiro de seus poemas (somos obrigados a nos servir dessa expressão
falando de uma escola que assimila a arte plástica ao pensamento
escrito), o primeiro de seus poemas data de 1848 e intitula"-se
\textit{A Dança dos Mortos em 1848.}

É um poema conservador cujo tema é a usurpação de todos os poderes e a
sedução operada sobre o povo pela deusa fatal da morte.

(Descrição minuciosa de cada uma das seis pranchas que compõem o poema e
a tradução exata das legendas em verso que as acompanham. Análise do
mérito artístico de Alfred Réthel\index{Rethel@Réthel, Alfred}, o que há de original nele
%Andre: o que são esses colchetes?
(gênio da alegoria épica à maneira alemã), o que há de postiço nele (imitações
dos diferentes mestres do passado, de Albrecht Dürer\index{Durer@Dürer, Albrecht}, de \indiceBA{Holbein}{Hans}, e
inclusive de mestres mais modernos), do valor moral do poema, caráter
satânico e byroniano, caráter de desolação.) O que acho de
verdadeiramente original no poema é que ele se produziu num momento em
que quase toda a humanidade europeia tinha se entusiasmado, com boa"-fé,
com as tolices da revolução.

Duas pranchas tornavam"-se antítese. A primeira: \textit{Primeira invasão
do cólera em Paris, no Baile do Opéra}. As máscaras rígidas, estendidas	\index{Paris}
no chão, expressão hedionda de uma pierrete cujas pontas dos pés estão
no ar e a máscara desfeita; os músicos que fogem com seus instrumentos;
alegoria do flagelo impassível sobre seu banco; caráter geralmente
macabro da composição. A segunda, uma espécie de boa morte fazendo
contraste; um homem virtuoso e calmo é surpreendido pela morte em seu
sono; ele está situado num lugar elevado, um lugar onde, sem dúvida,
viveu longos anos; é um quarto num campanário de onde se percebem os
campos e um vasto horizonte, um lugar feito para pacificar o espírito;
o velho homem está adormecido numa poltrona grosseira, a Morte toca uma
ária encantadora no violino. Um grande sol, cortado em dois pela linha
do horizonte, lança para cima seus raios geométricos. \textit{É o fim
de um belo dia}.

Um pequeno pássaro pousou sobre o parapeito da janela e olha para dentro
do quarto; vem ele escutar o violino da Morte, ou é uma alegoria da
alma prestes a partir?

É preciso, na tradução das obras de arte filosóficas, mostrar uma grande
minúcia e uma grande atenção; aí, os lugares, o cenário, os móveis, os
utensílios (ver \indiceBA{Hogarth}{William}), tudo é alegoria, alusão, hieróglifos, enigma.

\indice{Michelet} tentou interpretar de forma minuciosa a \textit{Melancolia} de
Albrecht Dürer\index{Durer@Dürer, Albrecht}; sua interpretação é suspeita, particularmente em relação
à seringa.

Por sinal, mesmo ao espírito de um artista filósofo, os acessórios se
oferecem, não com um caráter literal e preciso, mas com um caráter
poético, vago e confuso, e amiúde é o tradutor que inventa as
intenções.

A arte filosófica não é tão estranha à natureza francesa quanto se
poderia crer. A \indice{França} ama o mito, a moral, o enigma; ou, melhor
dizendo, país de raciocínio, ela ama o esforço do espírito.

Foi sobretudo a escola romântica que reagiu contra essas tendências
racionais e que fez prevalecer a glória da arte pura; e certas
tendências, particularmente aquelas de \indiceBA{Chenavard}{Paul}, reabilitação da arte
hieroglífica, são uma reação contra a escola da arte pela arte.

Haverá climas filosóficos como há climas amorosos? \indice{Veneza} praticou o
amor da arte pela arte; \indice{Lyon} é uma cidade filosófica. Existe uma
filosofia lionesa, uma escola de poesia lionesa, uma escola de pintura
lionesa e, enfim, uma escola de pintura filosófica lionesa.

Cidade singular, beata e mercante, católica e protestante, repleta de
brumas e carvões, lá as ideias se desembaraçam com dificuldade. Tudo
o que vem de \indice{Lyon} é minucioso, lentamente elaborado e temeroso; o abade
\indiceBA{Noireau}{abade} (\textit{sic}), \indiceBA{Laprade}{Victor de}, \indiceBA{Soulary}{Josephin (Joseph Marie)}, \indiceBA{Chenavard}{Paul}, \indiceBA{Janmot}{Anne"-François"-Louis}. Dir"-se"-ia
que os cérebros são, lá, constipados. Mesmo em \indiceBA{Soulary}{Josephin (Joseph Marie)} encontro esse
espírito de categoria que brilha sobretudo nos trabalhos de \indiceBA{Chenavard}{Paul} e
que se manifesta também nas canções de \indiceAB{Pierre}{Dupont}.

O cérebro de \indiceBA{Chenavard}{Paul} assemelha"-se à cidade de \indice{Lyon}; ele é brumoso,
fuliginoso, eriçado de pontas, como a cidade de campanários e fornos.
Nesse cérebro as coisas não se refletem claramente, refletem"-se apenas
através de um meio de vapores.

Chenavard não é pintor; despreza o que entendemos por pintura. Seria
injusto aplicar"-lhe a fábula de \indiceBA{La Fontaine}{Jean de} (elas estão muito verdes, e
boas apenas para criados);\footnote{ Aqui Baudelaire alude à fábula \index{Baudelaire@Baudelaire, Charles|nn}
\textit{A raposa e as uvas, }referente ao trecho em que a raposa
desdenha das uvas que não consegue alcançar [N.~do~T.].} pois creio
que, mesmo que \indiceBA{Chenavard}{Paul} pudesse pintar com tanta destreza quanto
qualquer um, não desprezaria menos o charme e a amenidade da arte.

Digamos de imediato que \indiceBA{Chenavard}{Paul} tem uma enorme superioridade sobre
todos os artistas: se ele não é bastante animal, eles são
excessivamente pouco espirituais.

\indiceBA{Chenavard}{Paul} sabe ler e raciocinar e se tornou, assim, o amigo de todas as
pessoas que amam o raciocínio; ele é extraordinariamente culto e possui
a prática da meditação.

O amor pelas bibliotecas manifestou"-se nele desde a juventude;
acostumado bem jovem a associar uma ideia a cada forma plástica, nunca
examinou cartões de gravuras ou contemplou museus de quadros senão como
repertórios do pensamento humano geral. Curioso por religiões e dotado
de um espírito enciclopédico, devia naturalmente desaguar na concepção
imparcial de um sistema sincrético.

Ainda que pesado e difícil de manobrar, seu espírito possui seduções das
quais ele sabe extrair grande proveito e, se esperou muito tempo antes
de desempenhar uma função, saibam que suas ambições, apesar de sua
aparente bonomia, nunca foram pequenas.

(Primeiros quadros de \indiceBA{Chenavard}{Paul}: --- \textit{De Dreux"-Brézé e Mirabeau --- A
convenção votando a morte de Luís} \textsc{xvi}. Chenavard escolheu muito bem
seu momento para exibir seu sistema de filosofia histórica, expresso
pelo creiom.)

Dividamos aqui nosso trabalho em duas partes: numa, analisaremos o
mérito intrínseco do artista dotado de uma habilidade surpreendente de
composição e bem maior do que se poderia suspeitar, se o desdém que
professa pelos recursos de sua arte --- habilidade de desenhar as
mulheres --- fosse levado muito a sério; na outra, examinaremos o mérito
que chamo extrínseco, isto é, o sistema filosófico.

Dissemos que ele tinha escolhido muito bem seu momento, quer dizer, o
dia seguinte de uma revolução.

(\indiceBA{Ledru"-Rollin}{Alexandre Auguste}: perturbação geral dos espíritos e viva preocupação
pública e relativa à filosofia da história.)

A Humanidade é análoga ao homem.

Ela tem suas idades e seus prazeres, suas concepções análogas a suas
idades.

(Análise do Calendário emblemático de \indiceBA{Chenavard}{Paul}. --- Que tal arte pertence
a tal idade da humanidade assim como tal paixão a tal idade do homem.

A idade do homem se divide em \textit{infância}, a qual corresponde na
humanidade ao período histórico desde \indice{Adão} até \indice{Babel}; virilidade, a
qual corresponde ao período desde \indice{Babel} até \indice{Jesus Cristo}, que será
considerado como o zênite da vida humana; \textit{meia"-idade}, que
corresponde desde \indice{Jesus Cristo} até \index{Bonaparte@Bonaparte, Napoleão}; e, enfim,
\textit{velhice}, que corresponde ao período no qual entraremos em
breve e cujo começo está marcado pela supremacia da \indice{América} e da
indústria.

A idade total da humanidade será de oito mil e quatrocentos anos.

De algumas opiniões particulares de \indiceBA{Chenavard}{Paul}. Da superioridade absoluta
de Péricles\index{Pericles@Péricles}.

Vileza da paisagem --- sinal de decadência.

A supremacia simultânea da música e da indústria, --- sinal de decadência.

Análise do ponto de vista da arte pura de alguns de seus cartões
expostos em 1855.)

O que serve para polir o caráter utópico e da própria decadência de
Chenavard é que ele queria arregimentar sob sua direção os artistas
como operários para executar em grandes dimensões seus
cartões e colori"-los de uma maneira bárbara.

\indiceBA{Chenavard}{Paul} é um grande espírito de decadência e permanecerá como marca
monstruosa do tempo.

\indiceBA{Janmot}{Anne"-François"-Louis} também é de \indice{Lyon}. 

É um espírito religioso e elegíaco, deve ter sido marcado jovem pela
beatice lionesa.

Os poemas de Réthel\index{Rethel@Réthel, Alfred} são bem construídos como poemas. O calendário
histórico de \indiceBA{Chenavard}{Paul} é uma fantasia de uma simetria irrefutável, mas
a \textit{História de uma alma} é inquietante e confusa.

A religiosidade que nela está impressa havia dado a essa série de
composições um grande valor para o jornalismo clerical, quando foram
expostas na passagem do Salmão; mais tarde as revimos na \indice{Exposição
Universal}, onde foram objeto de um augusto desdém.

Uma explicação em verso foi feita pelo artista, servindo apenas para
melhor demonstrar a indecisão de sua concepção e para melhor embaraçar
o espírito dos espectadores filósofos aos quais ela se endereçava.

Tudo o que compreendi é que esses quadros representavam os estados
sucessivos da alma em diferentes idades; entretanto, como sempre havia
dois seres em cena, um rapaz e uma moça, meu espírito se fatigou em
procurar se o pensamento íntimo do poema não era a história paralela de
duas jovens almas ou a história do duplo elemento masculino e feminino
da mesma alma.

Todas essas censuras postas de lado, provando simplesmente que \indiceBA{Janmot}{Anne"-François"-Louis}
não é um cérebro filosoficamente sólido, é preciso reconhecer que do
ponto de vista da arte pura havia na composição dessas cenas, e até
mesmo na cor amarga que as revestia, um charme infinito e difícil de
descrever, algo das suavidades da solidão, da sacristia, da igreja e do
claustro; uma misticidade inconsciente e infantil. Senti algo de
análogo diante de alguns quadros de \indiceBA{Lesueur}{Charles Alexandre} e de algumas telas
espanholas.

(Análise de alguns dos temas, em particular a \textit{Má instrução}, o
\textit{Pesadelo}, onde brilhava uma extraordinária compreensão do
fantástico. Uma espécie de passeio místico dos dois jovens sobre a
montanha etc.)

Todo espírito profundamente sensível e bem"-dotado para as artes (não se
deve confundir a sensibilidade da imaginação com a do coração) sentirá
como eu que toda arte deve se bastar a si mesma e ao mesmo tempo
permanecer nos limites providenciais; entretanto, o homem conserva esse
privilégio de sempre poder desenvolver grandes talentos num gênero
falso ou violando a constituição natural da arte.

Ainda que eu considere os artistas filósofos como heréticos,
frequentemente cheguei a admirar seus esforços por um efeito de minha
própria razão.

O que me parece sobretudo constatar seu caráter herético é sua
inconsequência, pois eles desenham muito bem, muito espiritualmente e,
se eles fossem lógicos em sua aplicação da arte assimilada a todo meio
de ensinamento, deveriam corajosamente remontar rumo a todas as
inumeráveis e bárbaras convenções da arte hierática.

\chapter[A obra e a vida de Eugène Delacroix]{a obra e a vida\break de eugène delacroix}

Ao Redator de \textit{L'Opinion Nationale}

\vspace{1em}
Senhor,

Eu gostaria, uma vez mais, uma última vez, de prestar homenagem ao gênio
de Eugène \indiceBA{Delacroix}{Eugène}, e lhe peço ter a amabilidade de acolher em seu
jornal essas poucas páginas onde tentarei encerrar, tão breve quanto
seja possível, a história de seu talento, a razão de sua superioridade,
que ainda não é, na minha opinião, suficientemente reconhecida e,
enfim, algumas anedotas e observações sobre sua vida e seu caráter.

Tive a felicidade de me ter ligado muito jovem (desde 1845, se não me
falha a memória) ao ilustre defunto, e nessa ligação, na qual o
respeito de minha parte e a indulgência da sua não excluíam a confiança
e a familiaridade recíprocas, pude calmamente extrair as noções mais
exatas, não apenas sobre seu método, mas também sobre as qualidades
mais íntimas de sua grande alma.

Não espere, Senhor, que eu faça aqui uma análise detalhada das obras de
\indiceBA{Delacroix}{Eugène}. Não só cada um de nós a fez, segundo suas forças e à medida
que o grande pintor mostrava ao público os trabalhos sucessivos de seu
pensamento, como o seu total é tão longo que, só ao conceder algumas
linhas a cada uma de suas principais obras, semelhante análise encheria
quase um volume. Que nos baste expor aqui um vivaz resumo.

Suas pinturas monumentais estão exibidas no \textit{Salão do Rei} na
Câmara dos Deputados, na biblioteca da Câmara dos Deputados, na
biblioteca do Palácio do Luxemburgo, na galeria de \indice{Apolo} no \indice{Louvre}, e	\index{Luxemburgo}
no Salão da Paz na Prefeitura. Essas decorações compreendem uma massa
enorme de temas alegóricos, religiosos e históricos, todos pertencendo
ao domínio mais nobre da inteligência. Quanto aos seus quadros ditos de
cavalete, seus esboços, suas grisalhas,\footnote{ Do francês,
\textit{grisailles}, pinturas monocromáticas em diferentes tonalidades
de cinza [N.~do~T.].} suas aquarelas etc., o total se eleva a um
número aproximado de duzentos e trinta e seis.

Os grandes temas expostos em diversos \textit{Salões} são em número de
setenta e sete. Extraio essas observações do catálogo que \indiceAB{Théophile}{Silvestre} dispôs ao final de sua excelente nota biográfica sobre Eugène
\indiceBA{Delacroix}{Eugène}, em seu livro intitulado: \textit{História dos pintores
vivos}.

Tentei, eu mesmo, várias vezes, elaborar esse enorme catálogo; todavia,
minha paciência foi quebrada por essa incrível fecundidade, e, cansado da
luta, renunciei a isso. Se \indiceAB{Théophile}{Silvestre} se enganou, só pôde ter
se enganado para menos.

Creio, senhor, que o importante aqui é simplesmente buscar a qualidade
característica do gênio de \indiceBA{Delacroix}{Eugène} e tentar defini"-la; buscar em que
ele difere de seus mais ilustres predecessores, igualando"-os ao mesmo
tempo; mostrar enfim, tanto quanto a palavra escrita o permita, a arte
mágica graças à qual ele pôde traduzir a \textit{palavra} por imagens
plásticas mais vivas e mais apropriadas do que as de algum criador de
mesma profissão --- em resumo, de que especialidade a Providência havia
encarregado Eugène \indiceBA{Delacroix}{Eugène} no desenvolvimento histórico da Pintura.

\sectionitem

Quem é Delacroix? Quais foram seu papel e seu dever nesse mundo? Tal é a
primeira questão a examinar. Serei breve e aspiro a conclusões
imediatas. \indice{Flandres} tem \indiceBA{Rubens}{Peter Paul}; a \indice{Itália} tem \indice{Rafael} e \indiceBA{Veronese}{Paolo}; a
\indice{França} tem \indiceBA{Le Brun}{Charles}, \indiceBA{David}{Jacques"-Louis} e \indiceBA{Delacroix}{Eugène}.

Um espírito artificial poderá se chocar, à primeira vista, pela reunião
desses nomes, que representam qualidade e métodos tão diferentes.
Todavia, um olhar espiritual mais atento logo verá que existe entre
todos um parentesco comum, uma espécie de fraternidade ou de parentela
que deriva de seu amor pelo grande, pelo nacional, pelo imenso e pelo
universal, amor que sempre se exprimiu na pintura dita decorativa ou
nas grandes \textit{machines}.\footnote{ Quadros pomposos [N.~do~T.].}

Muitos outros, sem dúvida, fizeram grandes \textit{machines}; mas esses
que citei fizeram"-nas da maneira mais adequada a deixar um vestígio
eterno na memória humana. Qual é o maior desses grandes homens tão
diferentes? Cada um pode decidir sobre isso a seu gosto, conforme seu
temperamento o leve a preferir a abundância prolífica, radiante, quase
jovial, de \indiceBA{Rubens}{Peter Paul}, a doce majestade e a ordem eurrítmica de \indice{Rafael}, a
cor paradisíaca e quase vespertina de \indiceBA{Veronese}{Paolo}, a severidade austera e
tensa de \indiceBA{David}{Jacques"-Louis}, ou a facúndia dramática e quase literária de \indiceBA{Le Brun}{Charles}.

Nenhum desses homens pode ser substituído; visando todos a semelhante
objetivo, empregaram meios diferentes extraídos de sua natureza
pessoal. \indiceBA{Delacroix}{Eugène}, o último a chegar, exprimiu, com uma veemência e um
fervor admiráveis, o que os outros haviam traduzido apenas de uma
maneira incompleta. Em detrimento de alguma outra coisa talvez, como
eles próprios haviam feito, por sinal? É possível; mas essa não é a
questão a ser examinada.

Muitos outros além de mim tiveram cuidado de insistir sobre as
consequências fatais de um gênio essencialmente pessoal; e também seria
bem possível, no fim de contas, que as mais belas expressões do gênio,
em outros lugares que não o céu puro, quer dizer, sobre essa pobre
terra onde a própria perfeição é imperfeita, só possam ser obtidas ao
preço de um inevitável sacrifício.

Mas enfim, senhor, dirá sem dúvida, qual é, portanto, esse não sei quê
de misterioso que \indiceBA{Delacroix}{Eugène}, para a glória de nosso século, traduziu
melhor do que qualquer outro? É o invisível, é o impalpável, é o sonho,
são os nervos, é a alma; e ele fez isso ---, observe"-o bem, senhor --- sem
outros meios além do contorno e da cor; ele o fez melhor do que
ninguém; ele o fez com a perfeição de um pintor consumado, com o rigor
de um literato sutil, com a eloquência de um músico apaixonado. É, de
resto, um dos diagnósticos do estado espiritual de nosso século que as
artes aspiram, senão a se suprir uma à outra, pelo menos a se dar
reciprocamente novas forças.

\indiceBA{Delacroix}{Eugène} é o mais \textit{sugestivo} de todos os pintores, aquele cujas
obras, escolhidas mesmo entre os secundários e os inferiores, mais
fazem pensar, e mais lembram à memória sentimentos e pensamentos
poéticos já conhecidos, mas que se acreditava enterrados para sempre na
noite do passado.

A obra de \indiceBA{Delacroix}{Eugène} parece"-me às vezes como uma espécie de mnemotecnia
da grandeza e da paixão nativa do homem universal. Esse mérito muito
particular e bem novo de Delacroix que lhe permitiu exprimir,
simplesmente com o contorno, o gesto do homem, por mais violento que
seja, e com a cor o que se poderia denominar a atmosfera do drama
humano, ou o estado da alma do criador --- esse mérito bem original
sempre reuniu em torno dele as simpatias de todos os poetas; e se, de
uma pura manifestação material, fosse permitido extrair uma verificação
filosófica, eu lhe pediria para observar, senhor, que, entre a multidão
acorrida para lhe prestar as derradeiras homenagens, se podiam contar
muito mais literatos do que pintores. Para dizer a verdade crua, esses
últimos jamais o compreenderam perfeitamente.

\sectionitem

E, apesar de tudo, o que há de surpreendente nisso? Não sabemos que a
época dos \indice{Michelangelo}, dos \indice{Rafael}, dos \indiceAB{Leonardo}{da Vinci}, digamos até
mesmo dos \indice{Reynold}, passou há muito tempo, e que o nível intelectual
geral dos artistas baixou singularmente? Seria injusto, sem dúvida,
procurar entre os artistas do momento filósofos, poetas e cientistas;
todavia, seria legítimo exigir deles que se interessem, um pouco mais
do que fazem, à religião, à poesia e à ciência.

Fora de seus ateliês o que sabem? O que amam? O que exprimem? Ora,
Eugène \indiceBA{Delacroix}{Eugène} era, ao mesmo tempo que um pintor apaixonado por seu
ofício, um homem de educação geral, ao contrário dos outros artistas
modernos que, em sua maioria, não passam de ilustres ou obscuros
pinta"-monos, tristes especialistas, velhos ou jovens; puros operários,
uns sabendo fabricar figuras acadêmicas, outros, frutas, e, outros
ainda, animais. Eugène \indiceBA{Delacroix}{Eugène} amava tudo, sabia pintar tudo, e sabia
apreciar todos os gêneros de talentos. Era o espírito mais aberto a
todas as noções e a todas as impressões, o fruidor mais eclético e mais
imparcial.

Grande leitor, isso é evidente. A leitura dos poetas deixava nele
imagens grandiosas e rapidamente definidas, quadros acabados, por assim
dizer. Por mais diferente que seja de seu mestre \indiceBA{Guérin}{Jean"-Urbain} pelo método e
pela cor, herdou da grande escola republicana e imperial o amor pelos
poetas e não sei qual espírito endiabrado de rivalidade com a palavra
escrita. \indiceBA{David}{Jacques"-Louis}, \indiceBA{Guérin}{Jean"-Urbain} e \indiceBA{Girodet}{Anne"-Louis} inflamavam seu espírito em contato com
\indice{Homero}, \indice{Virgílio}, \indice{Racine} e \indice{Ossian}. \indiceBA{Delacroix}{Eugène} foi o tradutor comovente
de \indiceBA{Shakespeare}{William}, \indice{Dante}, \indiceBA{Byron}{Lord} e \indiceBA{Ariosto}{Ludovico}. Semelhança importante e
diferença insignificante.

Mas avancemos um pouco mais, peço"-lhe, ao que se poderia chamar o
ensinamento do mestre, ensinamento que, para mim, resulta não só da
contemplação sucessiva de todas as suas obras e da contemplação
simultânea de algumas, conforme o senhor deve ter apreciado na
\indice{Exposição Universal} de 1855, mas também de muitas conversações que tive
com ele.

\sectionitem

\indiceBA{Delacroix}{Eugène} era ardorosamente amoroso da paixão, e friamente determinado a
procurar os meios de exprimi"-la da maneira mais visível.

Nesse duplo caráter, encontramos, digamo"-lo de passagem, os dois traços
que marcam os mais sólidos gênios, gênios extremos que não são
absolutamente feitos para agradar as almas temerosas, fáceis de
satisfazer, e que encontram um alimento suficiente nas obras apáticas,
débeis, imperfeitas. Uma imensa paixão, acrescida de uma vontade
formidável, tal era o homem.

Ora, ele dizia sem parar:

\begin{hedraquote}
Visto que considero a impressão transmitida ao artista pela natureza
como a coisa mais importante para traduzir, não será necessário que
este esteja armado de antemão de todos os mais rápidos meios de
tradução?
\end{hedraquote}

É evidente que aos seus olhos a imaginação era o dom mais precioso, a
faculdade mais importante; todavia, essa faculdade permaneceria
impotente e estéril, se não tivesse a seu serviço uma habilidade
rápida, que pudesse acompanhar a grande faculdade despótica em seus
caprichos impacientes. Ele não teria necessidade, com certeza, de
ativar o fogo de sua imaginação, sempre incandescente; entretanto,
sempre achava o dia muito curto para estudar os meios de expressão.

É a essa preocupação incessante que devem ser atribuídas suas pesquisas
perpétuas relativas à cor, à qualidade das cores, sua curiosidade pelas
coisas da química e suas conversações com os fabricantes de tintas.
Nisso ele se aproxima de \indiceAB{Leonardo}{da Vinci}, que também foi invadido
pelas mesmas obsessões.

Jamais Eugène \indiceBA{Delacroix}{Eugène}, apesar de sua admiração pelos fenômenos
ardentes da vida, será confundido com essa turba de artistas e
literatos vulgares cuja inteligência míope abriga"-se atrás da palavra
vaga e obscura de realismo. A primeira vez que vi \indiceBA{Delacroix}{Eugène}, em 1845,
creio (como os anos transcorrem, rápido e vorazes!), conversamos muito
sobre lugares"-comuns, isto é, questões das mais vastas e, contudo, das
mais simples: assim, da natureza, por exemplo. Aqui, senhor,
pedir"-lhe"-ei permissão para citar a mim mesmo, pois uma paráfrase não
valeria as palavras que outrora escrevi, quase sob ditado do mestre:

“A natureza outra coisa não é senão um dicionário, ele repetia com
frequência. Para bem compreender a amplitude do sentido implicado nessa
frase, deve"-se imaginar os usos ordinários e numerosos do dicionário.
Nele, procura"-se o sentido das palavras, a geração das palavras, a
etimologia das palavras, enfim, extraem"-se dele todos os elementos que
compõem uma frase ou uma narrativa; mas ninguém jamais considerou o
dicionário como uma composição, no sentido poético da palavra. Os
pintores que obedecem à imaginação procuram em seu dicionário os
elementos que se acomodam à sua concepção; e ainda, ajustando"-os com
uma certa arte, dão"-lhes uma fisionomia bem nova. Aqueles que não têm
imaginação copiam o dicionário. Resulta disso um enorme vício, o vício
da banalidade, que é mais particularmente próprio daqueles dentre os
pintores cuja especialidade mais se aproxima da natureza dita
inanimada, por exemplo os paisagistas, que consideram geralmente como
um triunfo não mostrar sua personalidade. Por muito contemplar e
copiar, eles esquecem de sentir e pensar.

“Para esse grande pintor, todas as partes da arte, da qual uma toma
esta, e a outra toma aquela como a principal, não eram, não são, quero
dizer, senão as muito humildes servas de uma faculdade única superior.
Se uma execução muito clara é necessária, é para que o sonho seja
claramente traduzido; que ela seja muito rápida, é para que nada se
perca da impressão extraordinária que acompanhava a concepção; que a
atenção do artista se dirija inclusive sobre o apuro material dos
instrumentos, concebe"-se isso sem dificuldade, pois todas as precauções
devem ser tomadas para tornar a execução ágil e decisiva.”

Para dizê"-lo de passagem, nunca vi paleta tão minuciosa e delicadamente
preparada quanto a de \indiceBA{Delacroix}{Eugène}. Assemelhava"-se a um buquê de flores
sabiamente combinadas.

“Num semelhante método, que é essencialmente lógico, todos os
personagens, sua disposição relativa, a paisagem ou o interior que lhes
serve de fundo ou de horizonte, suas vestes, tudo enfim deve servir
para iluminar a ideia geral e trazer sua cor original, sua marca, por
assim dizer. Do mesmo modo que um sonho é situado numa atmosfera
colorida que lhe é própria, assim também uma concepção, tornada
composição, necessita mover"-se num meio colorido que lhe seja
particular. Há evidentemente um tom particular atribuído a uma parte
qualquer do quadro que se torna chave e que governa as outras. Todo
mundo sabe que o amarelo, o alaranjado, o vermelho, inspiram e
representam ideias de alegria, de riqueza, de glória e de amor;
todavia, há milhares de atmosferas amarelas ou vermelhas, e todas as
outras cores serão afetadas logicamente numa quantidade proporcional
pela atmosfera dominante. A arte do colorista depende, sob certos
aspectos, da matemática e da música.

“Entretanto, suas operações mais delicadas se fazem por um sentimento ao
qual um longo exercício deu uma segurança inqualificável. Vê"-se que
essa grande lei de harmonia geral condena muitos ofuscamentos e muitas
cruezas, mesmo entre os pintores mais ilustres. Há quadros de Rubens
que não só fazem pensar num fogo de artifício colorido, mas até mesmo
em vários fogos de artifícios lançados do mesmo lugar. Quanto maior é
um quadro, mais larga deve ser a pincelada, isso é óbvio; mas é bom que
as pinceladas não sejam materialmente fundidas; elas se fundem
naturalmente a uma distância desejada pela lei simpática que as
associou. A cor obtém, assim, mais energia e frescor.

“Um bom quadro, fiel e igual ao sonho que o criou, deve ser produzido
como um mundo. Da mesma forma, a criação tal como a vemos é o resultado
de várias criações das quais as precedentes são sempre completadas pela
seguinte. Assim também um quadro, conduzido harmonicamente, consiste
numa série de quadros superpostos, cada nova camada dando ao sonho mais
realidade e fazendo"-o subir um grau rumo à perfeição. Bem ao contrário,
lembro"-me de ter visto nos ateliês de \indiceAB{Paul}{Delaroche} e de \indiceAB{Horace}{Vernet}
vastos quadros, não esboçados, mas começados, isto é, absolutamente
terminados em certas partes, enquanto algumas outras estavam apenas
indicadas por um contorno negro ou branco. Poder"-se"-ia comparar esse
gênero de obra com um trabalho puramente manual que deve cobrir uma
certa quantidade de espaço num tempo determinado, ou com uma longa
estrada dividida num grande número de etapas. Quando uma etapa é
concluída, ela não é retomada; e quando toda a estrada é percorrida, o
artista é liberado de seu quadro.

“Todos esses preceitos são evidentemente modificados mais ou menos pelo
temperamento variado dos artistas. Entretanto, estou convencido de que
esse é o método mais seguro para as imaginações férteis. Em
consequência, enormes desvios feitos fora do método em questão
testemunham uma importância anormal e injusta dada a alguma parte
secundária da arte.

“Não temo que se diga que é absurdo supor um mesmo método aplicado por
uma multidão de indivíduos diferentes. Pois é evidente que as retóricas
e as prosódias não são tiranias inventadas arbitrariamente, mas uma
coleção de regras exigidas pela própria organização do ser espiritual;
e nunca as prosódias e as retóricas impediram a originalidade de se
produzir distintamente. O contrário, ou seja, que elas ajudaram a
eclosão da originalidade, seria infinitamente mais verdadeiro.

“Para ser breve, sou obrigado a omitir uma grande quantidade de
corolários resultantes da fórmula principal, onde está contido, por
assim dizer, todo o formulário da verdadeira estética, e que pode ser
assim expresso: todo o universo não é senão um depósito de imagens e
sinais aos quais a imaginação dará um lugar e um valor relativo; é uma
espécie de alimento que a imaginação deve digerir e transformar. Todas
as faculdades da alma humana devem estar subordinadas à imaginação que
as requisita todas ao mesmo tempo. Assim como conhecer bem o dicionário
não implica necessariamente o conhecimento da arte da composição, e que
a arte da composição não implica a imaginação universal, do mesmo modo,
um bom pintor pode não ser um grande pintor; mas um grande pintor é
forçosamente um bom pintor, porque a imaginação universal contém a
compreensão de todos os meios e o desejo de adquiri"-los.

“É evidente que, de acordo com as noções que acabo de elucidar, bem ou
mal (haveria tantas coisas a dizer ainda, em particular acerca das
partes concordantes de todas as artes e as semelhanças em seus
métodos!), a imensa classe dos artistas, isto é, dos homens que se
dedicam à expressão do belo, pode dividir"-se em dois campos bem
distintos. Aquele que chama a si mesmo realista, palavra ambígua e cujo
sentido não é bem determinado, e que denominaremos, para melhor
caracterizar seu erro, um positivista diz: ‘Quero representar as coisas
tais como elas são, ou tais como seriam, supondo que eu não exista’. O
universo sem o homem. E aquele, o imaginativo, diz: ‘Quero iluminar as
coisas com meu espírito e projetar seu reflexo sobre os outros
espíritos’. Ainda que esses dois métodos absolutamente contrários
possam ampliar ou reduzir todos os temas, desde a cena religiosa até a
mais modesta paisagem, todavia, o homem de imaginação teve que, de um
modo geral, se produzir na pintura religiosa e na fantasia, enquanto a
pintura dita de gênero e a paisagem deveriam oferecer em aparência
várias fontes aos espíritos preguiçosos e dificilmente excitáveis...

“A imaginação de Delacroix! Essa nunca temeu escalar as alturas difíceis
da religião; o céu lhe pertence, como o inferno, como a guerra, como o
\indice{Olimpo}, como a volúpia. Eis o tipo do pintor"-poeta! Ele é um dos raros
eleitos, e a amplitude de seu espírito compreende a religião em seu
domínio. Sua imaginação, ardente como as câmaras"-ardentes, brilha com
todas as chamas e com todas as púrpuras. Tudo o que há de dor na
paixão, o apaixona; tudo o que há de esplendor na Igreja o ilumina.
Verte sucessivamente sobre suas telas inspiradas o sangue, a luz e as
trevas. Creio que ele ajuntaria de bom grado, como acréscimo, seu
fausto natural às majestades do \indice{Evangelho}.

“Vi uma pequena \textit{Anunciação}, de \indiceBA{Delacroix}{Eugène}, em que o anjo
visitando Maria não estava só, mas conduzido em cerimônia por dois
outros anjos, e o efeito dessa corte celeste era poderoso e encantador.
Um de seus quadros de juventude, o \textit{Cristo no Monte das
Oliveiras} (‘Senhor, afasta de mim esse cálice’), extravasa de ternura
feminina e de unção poética. A dor e a pompa, que eclodem tão alto na
religião, ecoam sempre em seu espírito.”

E ainda mais recentemente, a propósito dessa Capela dos Santos Anjos, em
\indice{São Sulpício} (\textit{Heliodoro expulso do templo} e \textit{A luta de
Jacob com o anjo}), seu último grande trabalho, tão estupidamente
criticado, eu dizia:

“Nunca, mesmo em \textit{A clemência de Trajano}, mesmo em \textit{A
entrada dos cruzados em Costantinopla}, Delacroix exibiu um colorido
mais esplêndido e sabiamente sobrenatural; nunca expôs um desenho mais
voluntariamente épico. Eu sei que algumas pessoas, pedreiros sem
dúvida, talvez arquitetos, pronunciaram, acerca dessa última obra, a
palavra decadência. Cabe aqui lembrar que os grandes mestres, poetas ou
pintores, \indiceBA{Hugo}{Victor} ou \indiceBA{Delacroix}{Eugène}, estão sempre vários anos à frente de seus
tímidos admiradores.

“O público é, em relação ao gênio, um relógio que atrasa. Quem, entre as
pessoas clarividentes, não compreende que o primeiro quadro do mestre
continha todos os outros em germe? Mas que ele aperfeiçoe
incessantemente seus dons naturais, que os torne mais vivos com zelo,
que extraia deles novos efeitos, que ele próprio leve sua natureza à
desmedida, isso é inevitável, fatal e louvável. O que é justamente a
marca principal do gênio de \indiceBA{Delacroix}{Eugène} é que ele não conhece a
decadência; só mostra progresso. Apenas, suas qualidades primitivas
eram tão veementes e tão ricas, e impressionaram com tanto vigor os
espíritos, mesmo os mais vulgares, que o progresso cotidiano é para
eles insensível; só os raciocinadores o percebem claramente.

“Eu falava ainda há pouco sobre os pedreiros. Quero caracterizar por
essa palavra essa classe de espíritos grosseiros e materiais (o número
deles é infinitamente grande), que não apreciam os objetos senão pelo
contorno, ou, ainda pior, por suas três dimensões: largura, comprimento
e profundidade, exatamente como os selvagens e os camponeses. Com
frequência, ouvi pessoas dessa espécie estabelecerem uma hierarquia das
qualidades, absolutamente ininteligível para mim; afirmar, por exemplo,
que a faculdade que permite a este criar um contorno exato, ou àquele,
um contorno de uma beleza sobrenatural, é superior à faculdade que sabe
associar cores de uma maneira encantadora. Segundo essas pessoas, a cor
não sonha, não pensa, não fala. Pareceria que, quando contemplo as
obras de um desses homens denominados especialmente coloristas, me
entrego a um prazer que não é de uma natureza nobre; de bom grado me
chamariam de materialista, reservando para eles mesmo o aristocrático
epíteto de espiritualistas.

“Esses espíritos superficiais não imaginam que as duas faculdades nunca
podem estar totalmente separadas, e que todas as duas são o resultado
de um germe primitivo cuidadosamente cultivado. A natureza exterior
fornece ao artista só uma oportunidade incessantemente renascente de
cultivar esse germe; ela é apenas um amontoado incoerente de materiais
que o artista é convidado a associar e ordenar, um
\textit{incitamentum}, um despertar para as faculdades sonolentas. Para
falar de modo preciso, não existem na natureza nem linha nem cor. É o
homem que cria a linha e a cor. São duas abstrações que extraem sua
igual nobreza de uma mesma origem.

“Um desenhista nato (supondo"-o criança) observa na natureza imóvel ou
movente certas sinuosidades, de onde extrai uma certa volúpia, e que
ele se diverte em fixar por linhas sobre o papel, exagerando ou
diminuindo ao bel"-prazer suas inflexões. Aprende assim a criar o garbo,
a elegância, o caráter no desenho. Suponhamos uma criança destinada a
aperfeiçoar a parte da arte que se chama cor: é do choque ou da feliz
associação de dois tons e do prazer que disso lhe resulta que extrairá
o conhecimento infinito das combinações de tons. A natureza foi, nos
dois casos, uma pura excitação.

“A linha e a cor fazem pensar e sonhar, todas as duas; os prazeres que
delas derivam são de natureza diversa, mas perfeitamente igual e
absolutamente independente do tema do quadro.

“Um quadro de \indiceBA{Delacroix}{Eugène}, colocado a uma grande distância para que você
possa julgar acerca da graça dos contornos ou da qualidade mais ou
menos dramática do tema, já o penetra de uma volúpia sobrenatural.
Parece"-lhe que uma atmosfera mágica caminhou em sua direção e os
encobre. Sombria, contudo deliciosa, luminosa, mas tranquila, essa
impressão, que ocupa para sempre seu lugar em sua memória, prova o
verdadeiro, o perfeito colorista. E a análise do tema, quando você se
aproxima, não retirará nada e não acrescentará nada a esse prazer
primitivo, cuja fonte se encontra alhures e longe de todo pensamento
concreto.

“Posso inverter o exemplo. Uma figura bem desenhada penetra"-o de um
prazer completamente estranho ao tema. Voluptuosa ou terrível, essa
figura só deve seu charme ao arabesco que ela recorta no espaço. Os
membros de um mártir que se escorcha, o corpo de uma ninfa desfalecida,
se eles são sabiamente desenhados, comportam um tipo de prazer em cujos
elementos o tema não entra absolutamente; se para você é de outro modo,
serei forçado a acreditar que você é um carrasco ou um libertino.

“Mas, lamentavelmente, para que, para que repetir sempre essas inúteis
verdades?”

Mas talvez, senhor, seus leitores apreciarão muito menos esta retórica
do que os detalhes que estou impaciente por lhes dar sobre a pessoa e
sobre os costumes de nosso saudoso grande poeta.

\sectionitem

É sobretudo nos escritos de \indiceBA{Delacroix}{Eugène} que aparece essa dualidade
de natureza da qual falei. Muitas pessoas, o senhor sabe,
surpreendiam"-se com a sabedoria de suas opiniões escritas e com a
moderação de seu estilo, umas lamentando, outras aprovando. As
\textit{Variações do belo}, os estudos sobre \indice{Poussin}, \indice{Prud'hon},
\indiceBA{Charlet}{Nicolas Toussaint}, e os outros fragmentos publicados seja em \textit{L’artiste}, 
cujo proprietário era, então, \indice{Ricourt}, seja em \textit{La revue dês
deux mondes}, apenas confirmam esse duplo caráter dos grandes artistas,
que os impele, como críticos, a louvar e a analisar mais
voluptuosamente as qualidades de que mais necessitam, enquanto
criadores, e que fazem antítese àquelas que possuem com
superabundância. Se \indiceBA{Delacroix}{Eugène} tivesse louvado, preconizado o que
admiramos principalmente nele, a violência, a brusquidão no gesto, a
turbulência da composição, a magia da cor, na verdade, teria sido o
caso de se surpreender. Por que procurar o que se possui em quantidade
quase supérflua, e como não exaltar o que nos parece mais raro e mais
difícil de adquirir? Sempre veremos, senhor, o mesmo fenômeno se
produzir entre os criadores de gênio, pintores ou literatos, todas as
vezes que aplicarem suas faculdades à crítica. Na época da grande luta
das duas escolas, a clássica e a romântica, os espíritos simples
ficavam pasmos ao ouvir \indiceBA{Delacroix}{Eugène} exaltar incessantemente
\indice{Racine}, \indiceBA{La Fontaine}{Jean de} e \indiceBA{Boileau}{Charles}. Conheço um poeta, de uma natureza sempre
tempestuosa e vibrante, que um verso de \indice{Malherbe}, simétrico e quadrado
de melodia, lança em longos êxtases.

Por sinal, por mais sábios, por mais sensatos e por mais isentos de
torneio e de intenção que nos pareçam os fragmentos literários do
grande pintor, seria absurdo crer que eles foram escritos facilmente e
com a certeza de expressão de seu pincel. Tanto estava seguro de
escrever o que ele pensava sobre uma tela quanto estava preocupado por
não poder \textit{pintar} seu pensamento sobre o papel. “A pena” ---
dizia com frequência --- “não é meu \textit{instrumento}; sinto que penso
correto, mas a necessidade da ordem, à qual sou obrigado a obedecer, me
apavora. Você acreditaria que a necessidade de escrever uma página me
dá enxaqueca?” É por essa dificuldade, resultado da falta de hábito,
que podem ser explicadas certas locuções um pouco gastas, um pouco
\textit{banais, império} mesmo, que escapam muito amiúde dessa pena
naturalmente distinta.

O que marca mais visivelmente o estilo de \indiceBA{Delacroix}{Eugène} é a concisão e uma
espécie de intensidade sem ostentação, resultado habitual da
concentração de todas as forças espirituais para um determinado ponto.
\textit{“The hero is he who is immovably centred”}, diz o moralista de
ultramar \indice{Emerson}, que, ainda que ele passe por chefe da enfadonha
escola bostoniana, não deixa de ter uma certa sutiliza à \indicex{Sêneca}{Seneca},
própria para alfinetar a meditação. “\textit{O herói é aquele que está
imutavelmente concentrado}”. A máxima que o chefe do
\textit{transcendentalismo} americano aplica à conduta da vida e ao
domínio dos negócios pode igualmente se aplicar ao domínio da poesia e
da arte. Poder"-se"-ia dizer da mesma forma: “O herói literário, isto é,
o verdadeiro escritor, é aquele que está imutavelmente concentrado”.
Não lhe parecerá, portanto, surpreendente, senhor, que Delacroix
tivesse uma simpatia muito pronunciada pelos escritores concisos e
concentrados, aqueles cuja prosa pouco carregada de floreios parece
imitar os movimentos rápidos do pensamento, e cuja frase assemelha"-se a
um gesto, \indice{Montesquieu}, por exemplo. Posso fornecer"-lhe um curioso
exemplo dessa brevidade fecunda e poética. O senhor leu como eu, sem
dúvida, nesses últimos dias, em \textit{La Presse}, um muito curioso e
belo estudo de Paul de \indiceBA{Saint"-Victor}{Paul de} 
acerca do teto da galeria de \indice{Apolo}.
As diversas concepções do dilúvio, a maneira como as lendas relativas
ao dilúvio devem ser interpretadas, o senso moral dos episódios e das
ações que compõem o conjunto desse maravilhoso quadro, nada é
esquecido; e o próprio quadro é minuciosamente descrito com esse estilo
encantador, tão espiritual quanto expressivo, do qual o autor nos
mostrou tantos exemplos. Entretanto, o conjunto deixará na lembrança
apenas um espectro difuso, algo como a luz muito vaga de uma
amplificação. Compare esse vasto trecho com as poucas linhas seguintes,
bem mais enérgicas, na minha opinião, e bem mais aptas a fazer quadro,
supondo inclusive que o quadro que elas resumem não exista. Transcrevo
simplesmente o programa distribuído por \indiceBA{Delacroix}{Eugène} a seus amigos, quando
os convidou a visitar a obra em questão: 

\vspace{1em}
\textsc{apolo vencedor da serpente píton.}
\vspace{1em}

“O deus, montado em seu carro, já lançou uma parte de suas setas; \indice{Diana},
sua irmã, voando logo atrás dele, apresenta"-lhe sua aljava. Já
perfurado pelas flechas do deus do calor e da vida, o monstro sangrando
se retorce exalando num vapor inflamado os restos de sua vida e de sua
raiva impotente. As águas do dilúvio começam a exaurir, e depositam
sobre os cimos das montanhas ou arrastam com elas os cadáveres dos
homens e dos animais. Os deuses se indignaram de ver a terra abandonada
a monstros disformes, produtos impuros do limo. Eles se armaram como
\indice{Apolo}, \indice{Minerva} e \indice{Mercúrio}, lançam"-se para exterminá"-los aguardando que a
Sabedoria eterna repovoe a solidão do universo. \indicex{Hércules}{Hercules} os esmaga com
sua maçã; \indice{Vulcano}, o deus do fogo, expulsa diante dele a noite e os
vapores impuros, enquanto \indicex{Bóreas}{Boreas} e os \indice{Zéfiros} secam as águas com seu
sopro e acabam de dissipar as nuvens. As \indice{Ninfas} dos rios e das ribeiras
reencontraram seu leito de bambus e sua ânfora ainda suja pelo lodo e
pelos resíduos. Divindades mais tímidas contemplam à distância esse
combate dos deuses e dos elementos. Entretanto, do alto dos céus a
\indice{Vitória} desce para coroar \indice{Apolo} vencedor, e Íris\index{Iris@Íris}, a mensageira dos
deuses, desfralda no ar sua echarpe, símbolo do triunfo da luz sobre as
trevas e sobre a revolta das águas.”

Eu sei que o leitor será obrigado a adivinhar muito, a colaborar, por
assim dizer, com o redator do artigo; mas o senhor realmente acredita
que a admiração pelo pintor me torne visionário nesse caso, e que eu me
engane absolutamente ao pretender descobrir aqui o vestígio dos hábitos
aristocráticos adquiridos nas boas leituras e dessa retidão de
pensamento que permitiu a homens do mundo, a militares, a aventureiros,
ou mesmo a simples cortesãos, escrever, algumas vezes de modo
desordenado, belos livros que nós, pessoas do ofício, somos obrigados a
admirar?

\sectionitem

Eugène \indiceBA{Delacroix}{Eugène} era uma curiosa mistura de ceticismo, polidez,
dandismo, vontade ardente, astúcia, despotismo e, enfim, uma espécie de
bondade particular e de ternura moderada que acompanha sempre o gênio.
Seu pai pertencia a essa raça de homens fortes dos quais conhecemos os
últimos em nossa infância; uns, fervorosos apóstolos de \index{Rousseau@Rousseau, Jean"-Jacques} Jean"-Jacques,
outros, discípulos determinados de \indice{Voltaire}, que, colaboraram, todos,
com igual obstinação, para a Revolução Francesa, e cujos sobreviventes,
jacobinos ou \textit{cordeliers}, aderiram, com uma perfeita boa"-fé (é
importante ressaltar), às intenções de \indiceBA{Bonaparte}{Napoleão}.

Eugène \indiceBA{Delacroix}{Eugène} sempre conservou os traços dessa origem revolucionária.
Pode"-se dizer dele, assim como de \indice{Stendhal}, que tinha grande pavor
de ser enganado. Cético e aristocrata, só conhecia a paixão e o
sobrenatural por sua convivência forçada com o sonho. Odiava as
multidões, considerava"-as apenas como destruidoras de imagens, e as
violências cometidas em 1848 contra algumas de suas obras não eram
feitas para convertê"-lo ao sentimentalismo político de nosso tempo.
Havia nele, inclusive, como estilo, maneiras e opiniões, alguma coisa
de Victor \indiceBA{Jacquemont}{Victor}. Sei que a comparação é um pouco injuriosa; por
isso, desejo que ela seja entendida apenas com uma extrema moderação.
Há em Jacquemont o belo espírito burguês revoltado e uma zombaria tão
inclinada a mistificar os ministros de Brahma quanto aqueles de \indice{Jesus
Cristo}. Delacroix, advertido pelo gosto sempre inerente ao gênio, não 
podia jamais cair nessas vilanias. Minha comparação, portanto, só diz
respeito ao espírito de prudência e à sobriedade com que todos os dois
são marcados. Da mesma forma, os sinais hereditários que o século \textsc{xviii}
havia deixado sobre sua natureza pareciam emprestados sobretudo dessa
classe tão afastada dos utopistas quanto dos furibundos, da classe dos
céticos polidos, os vencedores e os sobreviventes, que, geralmente,
dependiam mais de \indice{Voltaire} do que de \index{Rousseau@Rousseau, Jean"-Jacques} Jean"-Jacques. Por isso, ao
primeiro olhar, Eugène \indiceBA{Delacroix}{Eugène} aparecia simplesmente como um homem
\textit{esclarecido}, no sentido honorável do termo, como um perfeito
\textit{gentleman} sem preconceitos e sem paixões. Era apenas por um
convívio mais assíduo que se podia penetrar sob o verniz e adivinhar as
partes ocultas de sua alma. Um homem a quem se poderia mais
legitimamente compará"-lo, pela aparência exterior e pelas maneiras,
seria \indiceBAx{Mérimée}{Prosper}{Merimee}. Era a mesma frieza aparente, ligeiramente afetada, a
mesma capa de gelo recobrindo uma pudica sensibilidade e uma ardente
paixão pelo bem e pelo belo; era, sob a mesma hipocrisia de egoísmo, o
mesmo devotamento aos amigos secretos e às ideias de predileção.

Havia em Eugène \indiceBA{Delacroix}{Eugène} muito do \textit{selvagem}; esta era a mais
preciosa parte de sua alma, a parte consagrada integralmente à pintura
de seus sonhos e ao culto de sua arte. Havia nele muito do homem do
mundo; esta parte era destinada a ocultar a primeira e a fazer
desculpá"-la. Foi, creio, uma das grandes preocupações de sua vida,
dissimular as cóleras de seu coração e não parecer um homem de gênio.
Seu espírito de dominação, espírito bem legítimo, fatal por sinal,
havia quase inteiramente desaparecido sob mil gentilezas. Dir"-se"-ia uma
cratera de vulcão artisticamente ocultada por buquês de flores.

Um outro traço de semelhança com \indice{Stendhal} era sua propensão às fórmulas
simples, às máximas breves, pela boa conduta da vida. Como todas as
pessoas tanto mais apaixonadas por método quanto seu temperamento
ardente e sensível parece desviá"-las mais disso, \indiceBA{Delacroix}{Eugène} gostava de
criar esses pequenos catecismos de moral prática que os estouvados e os
ociosos que nada praticam atribuiriam desdenhosamente a \indiceBA{De la Palisse}{(Jacques II de Chabannes)}
mas que o gênio não despreza, porque ele está aparentado com a
simplicidade; máximas sãs, fortes, simples e duras, que servem de
couraça e de escudo àquele que a fatalidade de seu gênio lança numa
batalha perpétua.

Preciso dizer"-lhe que o mesmo espírito de sabedoria firme e desprezível
inspirava as opiniões de \indiceBA{Delacroix}{Eugène} em matéria de política? Ele
acreditava que nada muda, ainda que tudo pareça mudar, e que certas
épocas climatéricas, na história dos povos, trazem de volta
invariavelmente fenômenos análogos. Em suma, seu pensamento, nesses
tipos de coisas, aproximava"-se muito, sobretudo por seus aspectos de
fria e desoladora resignação, do pensamento de um historiador do qual
faço, de minha parte, um caso bem particular, e que o senhor mesmo, tão
perfeitamente habituado a essas teses, e que sabe estimar o talento,
mesmo quando ele o contradiz, o senhor foi, estou certo disso, obrigado
a admirar várias vezes. Quero falar de \indiceBA{Ferrari}{Giuseppe}, o sutil e sábio autor
da \textit{História da razão de Estado}. Por isso, o loquaz que, diante
de \indiceBA{Delacroix}{Eugène}, se entregava aos entusiasmos infantis das utopias,
deveria, em breve, sofrer o efeito de seu riso amargo, impregnado de
uma piedade sarcástica; e se, imprudentemente, lançassem diante dele a
grande quimera dos tempos modernos, o balão"-monstro da perfectibilidade
e do progresso indefinidos, de bom grado ele lhe perguntaria: “onde
estão, portanto, seus \indicex{Fídias}{Fidias}? Onde estão seus \indice{Rafael}?”

Acredite, entretanto, que esse duro bom"-senso não retirava nenhuma graça
de \indiceBA{Delacroix}{Eugène}. Essa verve de incredulidade e essa recusa de ser enganado
temperavam, como um sal byroniano, sua conversação tão poética e tão
expressiva. Ele também extraía de si mesmo, muito mais do que as
emprestava de sua longa convivência com o mundo --- de si mesmo, isto é,
de seu gênio e da consciência de seu gênio ---, uma certeza, uma
facilidade de maneiras maravilhosa, com uma polidez que admitia, como
um prisma, todas as nuanças, desde a bonomia mais cordial até a
impertinência mais irrepreensível. Ele possuía vinte maneiras
diferentes de pronunciar “\textit{meu caro senhor}”, que representavam,
para um ouvido experimentado, uma curiosa gama de sentimentos. Pois
enfim, devo dizê"-lo, posto que vejo nisso um novo motivo de elogio, E.
Delacroix, ainda que fosse um homem de gênio, ou porque era um homem de
gênio completo, participava muito da natureza do dândi. Ele próprio
confessava que em sua juventude se entregara com prazer às vaidades
mais materiais do dandismo, e contava rindo, mas não sem uma certa
gloríola, que tinha, com concurso de seu amigo \indice{Bonington}, trabalhado
muito para introduzir entre a juventude elegante o gosto pelo corte
inglês no calçar e no vestir. Esse detalhe, presumo, não lhe parecerá
inútil, pois não há recordação supérflua quando se deve retratar a
natureza de certos homens.

Eu lhe disse que era sobretudo a parte natural da alma de \indiceBA{Delacroix}{Eugène} que,
apesar do véu atenuante de uma civilização refinada, impressionava o
observador atento. Tudo nele era energia, mas energia derivando dos
nervos e da vontade, pois, fisicamente, ele era fraco e delicado. O
tigre, atento à sua presa, tem menos brilho nos olhos e estremecimentos
impacientes nos músculos do que deixava ver nosso grande pintor, quando
toda sua alma estava lançada sobre uma ideia ou queria se apoderar de
um sonho. O próprio caráter físico de sua fisionomia, sua tez de
peruano ou malaio, seus olhos grandes e negros, mas diminuídos pelos
pestanejos da atenção, e que pareciam degustar a luz, seus cabelos
abundantes e brilhosos, sua fronte obstinada, seus lábios cerrados, aos
quais uma tensão perpétua de vontade comunicava uma expressão cruel,
toda sua pessoa, enfim, sugeria a ideia de uma origem exótica.
Aconteceu"-me várias vezes, ao observá"-lo, imaginar os antigos soberanos
do \indicex{México}{Mexico}, esse \indice{Montezuma} cuja mão hábil nos sacrifícios podia imolar
em um único dia três mil criaturas humanas sobre o altar piramidal do
Sol, ou então algum desses príncipes hindus que, nos esplendores das
mais gloriosas festas, trazem no fundo de seus olhos um tipo de avidez
insatisfeita e uma nostalgia inexplicável, alguma coisa como a
lembrança e o pesar das coisas não conhecidas. Observe, peço"-lhe, que a
cor geral dos quadros de \indiceBA{Delacroix}{Eugène} também participa da cor própria das
paisagens e dos interiores orientais, e que ela produz uma impressão
análoga àquela ressentida nesses países intertropicais onde uma imensa
difusão de luz cria para um olhar sensível, apesar da intensidade dos
tons locais, um resultado geral quase crepuscular. A moralidade de suas
obras, se, contudo, é permitido falar da moral em pintura, também
apresenta um caráter molochista\footnote{ Relativo a Moloch, deus 	\index{Moloch|nn}
cruel dos amonitas, ao qual eram imolados seres humanos [N.~do~T.].}
visível. Tudo, em sua obra, é desolação, massacres, incêndios; tudo
testemunha contra a eterna e incorrigível barbárie do homem. As cidades
incendiadas e fumegantes, as vítimas degoladas, as mulheres estupradas,
as próprias crianças lançadas sob as patas dos cavalos ou sob o punhal
das mães delirantes; toda essa obra, eu dizia, assemelha"-se a um hino
terrível composto em honra da fatalidade e da irremediável dor. Ele
pôde, algumas vezes, pois não lhe faltava certamente ternura, consagrar
seu pincel à expressão de sentimentos ternos e voluptuosos; mas ainda
aí a incurável amargura estava disseminada em forte dose, e a
despreocupação e a alegria (que são as companheiras habituais da
volúpia ingênua) encontravam"-se ausentes. Uma única vez, creio, ele fez
uma tentativa no cômico e no bufão, e, como se tivesse adivinhado que
isso estava além e abaixo de sua natureza, não mais retornou a esse
gênero.

\sectionitem

Conheço várias pessoas que têm o direito de dizer: “\textit{Odi profanum
vulgus}”; mas qual delas pode acrescentar vitoriosamente: “\textit{et
arceo}”? O aperto de mão muito frequente avilta o caráter. Se algum dia
um homem teve uma torre de marfim bem defendida pelas grades e
fechaduras, esse homem foi Eugène \indiceBA{Delacroix}{Eugène}. Quem mais amou sua
\textit{torre de marfim}, quer dizer, o segredo? Ele a teria, creio,
armado de bom grado de canhões e a teria transportado para uma floresta
ou para um rochedo inacessível. Quem mais amou seu \textit{home},
santuário e toca? Assim como outros buscam o segredo para a orgia, ele
busca o segredo para a inspiração, e para isso entregava"-se a
verdadeiros festins de trabalho. “\textit{The one prudence in life is
concentration; the one evil is dissipation}”, diz o filósofo americano
que já citei.

Delacroix teria podido escrever essa máxima; todavia, é verdade, ele a
praticou austeramente. Era demasiado \textit{homem do mundo} para não
desprezar o mundo; e os esforços que despendia nisso, para não ser
muito visivelmente \textit{ele próprio}, levavam"-no de forma natural a
preferir nossa sociedade. Nossa não quer somente implicar o humilde
autor que escreve essas linhas, mas também alguns outros, jovens ou
velhos, jornalistas, poetas, músicos, junto aos quais ele podia
livremente relaxar e entregar"-se.

Em seu delicioso estudo sobre \indiceBA{Chopin}{Frédéric}, \indiceBA{Liszt}{Franz} coloca \indiceBA{Delacroix}{Eugène} entre os
mais assíduos visitantes do músico"-poeta, e diz que ele adorava cair em
profundo devaneio, aos sons dessa música suave e apaixonada que se
assemelha a um brilhante pássaro esvoaçando sobre os horrores de um
abismo.

Foi assim que, graças à sinceridade de nossa admiração, pudemos, ainda
que muito jovem então, penetrar nesse ateliê tão bem protegido, onde
reinava, a despeito de nosso rigoroso clima, uma temperatura
equatorial, e onde o olhar era de antemão surpreendido por uma solenidade sóbria e pela
austeridade particular da velha escola, tais como, em nossa infância,
tínhamos visto os ateliês dos antigos rivais de \indiceBA{David}{Jacques"-Louis}, heróis
comoventes há muito desaparecidos. Bem se percebia que esse refúgio não
podia ser habitado por um espírito frívolo, titilado por mil caprichos
incoerentes.

Em seu ateliê, nada de panóplias enferrujadas, nada de crises malaios,
nada de velhas ferragens góticas, nada de bijuteria, nada de trastes,
nada de bricabraque, nada do que acuse no proprietário o gosto pelo
passatempo e pela vagabundagem rapsódica de um devaneio infantil. Um
maravilhoso retrato por \indiceBA{Jordaens}{Jacob}, que ele havia descoberto não sei
onde, alguns estudos e cópias feitos pelo próprio mestre, bastavam à
decoração desse vasto ateliê, do qual uma luz tênue e suave clareava o
recolhimento.

Ver"-se"-ão provavelmente essas cópias durante a venda dos desenhos e
quadros de Delacroix que está marcada, segundo me disseram, para o mês
de janeiro próximo. Ele tinha duas maneiras muito distintas de copiar.
Uma, livre e ampla, feita metade de fidelidade, metade de traição, e
onde colocava muito de si mesmo. Desse método resultava um composto
híbrido e encantador, lançando o espírito numa incerteza agradável. É
sob esse aspecto paradoxal que se apresentou a mim uma grande cópia dos
\textit{Milagres de São Bento}, de \indiceBA{Rubens}{Peter Paul}. Na outra maneira, \indiceBA{Delacroix}{Eugène}
se torna o escravo mais obediente e mais humilde de seu modelo, e
chegava a uma exatidão de imitação da qual podem duvidar aqueles que
não viram esses milagres. Tais são, por exemplo, aquelas feitas a
partir de duas cabeças de \indice{Rafael} que estão no \indice{Louvre}, e onde a
expressão, o estilo e a maneira são imitados com tão perfeita
singeleza, que se poderia alternativa e reciprocamente tomar os
originais pelas imitações.

Após um almoço mais leve do que o de um árabe, e sua paleta
minuciosamente composta com o esmero de uma ramalheteira ou de um
comerciante de tecidos, \indiceBA{Delacroix}{Eugène} procurava abordar a ideia
interrompida; mas, antes de se lançar em seu trabalho tempestuoso,
experimentava amiúde esses langores, esses medos, esses enervamentos
que fazem pensar na pitonisa fugindo do deus, ou que lembram
Jean"-Jacques \indiceBA{Rousseau}{Jean"-Jacques} entretendo"-se, remexendo em papelada e mudando
seus livros durante uma hora antes de atacar o papel com a pena.
Todavia, uma vez operada a fascinação do artista, só interrompia seu
trabalho vencido pela fadiga física.

Um dia, como conversávamos sobre essa questão sempre tão interessante
para os artistas e os escritores, ou seja, sobre a higiene do trabalho
e sobre a conduta da vida, ele me disse:

``Outrora, em minha juventude, eu não podia me pôr ao
trabalho senão quando tinha a promessa de um prazer para a noite:
música, baile, ou qualquer outra diversão. Hoje, contudo, já não me
pareço mais com os estudantes, posso trabalhar sem cessar e sem nenhuma
esperança de recompensa. E ainda, continuava,
se você soubesse como um trabalho assíduo torna
indulgente e pouco difícil em matéria de prazeres! O homem que ocupou
bem o seu dia estará disposto a encontrar espírito suficiente no
mensageiro da esquina e a jogar cartas com ele.''

Essas palavras me faziam pensar em Maquiavel jogando dados com os
camponeses. Ora, um dia, um domingo, avistei Delacroix no \indice{Louvre}, em
companhia de sua velha criada, aquela que tão devotamente cuidou dele e
o serviu durante trinta anos, e ele, o elegante, o refinado, o erudito,
não desdenhava mostrar e explicar os mistérios da escultura assíria a
essa excelente mulher, que o escutava, por sinal, com uma sincera
atenção. A lembrança de \indiceBA{Maquiavel}{Nicolau} e de nossa antiga conversação
penetrou imediatamente em meu espírito.

A verdade é que, nos últimos anos de sua vida, tudo o que se chama
prazer havia desaparecido dele, tendo um único, severo, exigente,
terrível, substituído todos, o trabalho, que então não era apenas uma
paixão, mas teria podido se chamar um furor.

\indiceBA{Delacroix}{Eugène}, após ter consagrado as horas do dia a pintar, seja em seu
ateliê, seja sobre os andaimes onde o chamavam seus grandes trabalhos
decorativos, ainda encontrava forças em seu amor pela arte, e teria
julgado esse dia mal preenchido se as horas da noite não tivessem sido
empregadas perto da lareira, na claridade da lâmpada, a desenhar, a
cobrir o papel de sonhos, projetos, figuras entrevistas nos acasos da
vida, algumas vezes a copiar desenhos de outros artistas cujo
temperamento era o mais afastado do seu; pois tinha paixão pelas
anotações, pelos croquis, e entregava"-se a isso em qualquer lugar que
fosse. Durante bastante tempo teve por hábito desenhar nas casas dos
amigos, junto aos quais ia passar suas noites. É assim que \indice{Villot}
possui uma quantidade considerável de excelentes desenhos dessa pena
fecunda.

Uma vez, disse a um jovem conhecido meu: ``Se você não é
bastante hábil para fazer o croquis de um homem que se atira pela
janela, durante o tempo que ele leva para cair do quarto andar ao solo,
você nunca poderá produzir grandes quadros''. Vejo nessa
enorme hipérbole a preocupação de toda a sua vida, que era, conforme se
sabe, executar bastante rápido e com bastante certeza para nada deixar
se evaporar da intensidade da ação ou da ideia.

\indiceBA{Delacroix}{Eugène} era, como muitos outros puderam observá"-lo, um homem de
conversação. Mas o engraçado é que tinha medo da conversação como de
uma orgia, de uma dissipação, onde corria o risco de perder suas
forças. Quando se entrava em sua casa, ele logo dizia:

``Não conversaremos esta manhã, não é mesmo? Ou muito
pouco, muito pouco.''

Em seguida, falava durante três horas. Sua conversa era brilhante,
sutil, mas repleta de fatos, lembranças e anedotas; em suma, uma
palavra nutriente.

Quando estava excitado pela contradição, recolhia"-se momentaneamente e,
em vez de se lançar sobre seu adversário de frente, o que tem o perigo
de introduzir as brutalidades da tribuna nas escaramuças de salão,
jogava durante algum tempo com seu adversário, em seguida voltava ao
ataque com argumentos ou fatos imprevistos. Tratava"-se da conversação
de um homem apaixonado por lutas, mas escravo da cortesia, astuta,
intencionalmente flexível, repleta de fugas e de ataques repentinos.

Na intimidade do ateliê, entregava"-se de bom grado até a emitir sua
opinião acerca dos pintores, seus contemporâneos, e era nessas ocasiões
que podíamos admirar, com frequência, essa indulgência do gênio que
deriva talvez de um tipo particular de ingenuidade ou de facilidade ao
gozo.

Ele tinha fraquezas surpreendentes por \indiceBA{Decamps}{Alexandre"-Gabriel}, hoje bem reduzidas, mas
que sem dúvida ainda reinavam em seu espírito pela força da lembrança. O
mesmo em relação a \indiceBA{Charlet}{Nicolas Toussaint}. Fez"-me ir uma vez à sua casa de propósito
para me admoestar, de modo veemente, em consequência de um artigo
irrespeitoso que eu escrevera contra essa criança mimada do
chauvinismo. Tentei em vão explicar"-lhe que não era o Charlet dos
primeiros momentos que eu censurava, mas o \indiceBA{Charlet}{Nicolas Toussaint} da decadência; não o
nobre historiador dos \textit{grognards},\footnote{ Soldados da velha
guarda sob Napoleão \index{Bonaparte@Bonaparte, Napoleão|nn} 
\textsc{i} [N.~do~T.].} mas o pedante do bar. Nunca pude me
fazer perdoar.

Ele admirava \indiceBA{Ingres}{Jean Auguste Dominique} parcialmente e, é verdade, faltava"-lhe uma grande
força de crítica para admirar por razão o que devia rejeitar por
temperamento. Inclusive copiou cuidadosamente fotografias feitas a
partir de alguns desses minuciosos retratos a creiom, em que se faz
melhor apreciar o duro e penetrante talento de \indiceBA{Ingres}{Jean Auguste Dominique}, tanto mais ágil
quanto mais embaraçado está.

A detestável pintura de Horace \indiceBA{Vernet}{Horace} não o impedia de sentir a
virtualidade pessoal que anima a maioria de seus quadros, e encontrava
expressões surpreendentes para elogiar esse cintilamento e esse
infatigável ardor. Sua admiração por Meissonier ia um pouco mais longe.
Apropriara"-se, quase por violência, dos desenhos que tinham servido
para preparar a composição de \textit{A barricada}, o melhor quadro de
\indiceBA{Meissonier}{Jean"-Louis"-Ernest}, cujo talento, por sinal, exprime"-se bem mais energicamente
pelo simples creiom do que pelo pincel. Ele dizia amiúde desse pintor,
como que refletindo com inquietação acerca do futuro: ``Em
fim de contas, de nós todos, é ele quem está mais seguro de
sobreviver!'' Não é curioso ver o autor de tão grandes
obras quase invejar aquele que só excele nas pequenas?

O único homem cujo nome teve força para arrancar alguns palavrões dessa
boca aristocrática foi Paul \indiceBA{Delaroche}{Paul}. Nas obras desse pintor \indiceBA{Delacroix}{Eugène}
não encontrava sem dúvida nenhuma desculpa, e conservava indelével a
lembrança dos sofrimentos que lhe causara essa pintura suja e amarga,
\textit{``feita com tinta e graxa''}, como disse outrora Théophile \indiceBA{Gautier}{Théophile}.

Mas aquele que ele escolhia com maior prazer para se refugiar em imensas
conversas era o homem que menos se assemelhava a ele pelo talento assim
como pelas ideias, seu verdadeiro antípoda, um homem a quem ainda não
se fez toda a justiça que lhe é devida, e cujo cérebro, ainda que
obscurecido como o céu encarvoado de sua cidade natal, contém uma
grande quantidade de coisas admiráveis. Trata"-se de Paul \indiceBA{Chenavard}{Paul}.

As teorias abstrusas do pintor filósofo de \indice{Lyon} faziam \indiceBA{Delacroix}{Eugène} sorrir,
e o pedagogo abstrator considerava as volúpias de pura pintura como
coisas frívolas, senão culpadas. Todavia, por mais distantes que
estivessem um do outro, e até mesmo por causa desse distanciamento,
gostavam de se aproximar, e como dois navios atados pelo arpéus não
podiam mais se separar. Todos os dois, por sinal, sendo muito letrados
e dotados de um extraordinário espírito de sociabilidade,
reencontrar"-se"-iam no terreno comum da erudição. Sabe"-se que em geral
esta não é a qualidade pela qual brilham os artistas.

\indiceBA{Chenavard}{Paul} era para \indiceBA{Delacroix}{Eugène}, portanto, um raro recurso. Era realmente
agradável ver os dois se agitarem numa luta inocente, a palavra de um
avançando pesadamente como um elefante em grande aparelho de guerra, a
palavra do outro vibrando como um florete, igualmente aguda e flexível.
Nas últimas horas de sua vida, nosso grande pintor demonstrou o desejo
de apertar a mão de seu amigável contraditor. Mas este se encontrava,
então, longe de \indice{Paris}.

\sectionitem

As mulheres sentimentais e preciosas ficarão talvez chocadas ao saber
que, semelhante a \indice{Michelangelo} (lembremos do final de um de seus
sonetos: ``Escultura! Divina Escultura, és
minha única amante!''), \indiceBA{Delacroix}{Eugène} fizera da Pintura sua
única musa, sua única amante, sua única e suficiente volúpia.

Sem dúvida ele havia amado muito a mulher nas horas agitadas de sua
juventude. Quem não se entregou em excesso a esse ídolo temível? E quem
não sabe que são justamente aqueles que melhor a serviram que mais se
queixam dela? Entretanto, já muito tempo antes de seu fim, excluíra a
mulher de sua vida. Muçulmano, talvez não a tivesse expulso de sua
mesquita, mas teria se surpreendido de aí vê"-la entrar, não
compreendendo bem que tipo de conversação ela podia ter com \indice{Alá}.

Nessa questão, como em muitas outras, a ideia oriental ganhava nele
preeminência viva e despótica. Considerava a mulher como um objeto de
arte, delicioso e próprio para excitar o espírito, mas um objeto de
arte desobediente e perturbador, se se lhe entrega o limiar do coração,
e que devora glutonicamente o tempo e as forças.

Recordo"-me que uma vez, num local público, como eu lhe mostrava o rosto
de uma mulher de original beleza e de um caráter melancólico, ele bem
quis apreciar sua beleza, mas me disse, com seu leve sorriso, para
responder ao resto: ``Como você quer que uma mulher possa
ser melancólica?'', insinuando com este comentário, sem
dúvida, que, para conhecer o sentimento da melancolia, falta à mulher
\textit{certa coisa }essencial.

Essa é, infelizmente, uma teoria bem injuriosa, e eu não gostaria de
preconizar opiniões difamatórias sobre um sexo que tão frequentemente
mostrou ardentes virtudes. Todavia, admitir"-se"-á que se trata de uma
teoria de prudência; que o talento não poderia se armar de prudência o
suficiente num mundo repleto de ciladas, e que o homem de gênio possui
o privilégio de certas doutrinas (desde que elas não perturbem a ordem)
que nos escandalizariam justamente no puro cidadão ou no simples pai de
família.

Devo acrescentar, ao risco de lançar uma sombra sobre sua memória, ao
julgamento das almas elegíacas, que ele também não demonstrava ternas
fraquezas pela infância. A infância só aparecia a seu espírito de mãos
lambuzadas de geleias (o que suja a tela e o papel), ou batendo tambor
(que perturba a meditação), ou incendiária e animalescamente perigosa
como o macaco.

``Recordo"-me muito bem'', dizia, às vezes,
``que, quando criança, \textit{eu era um monstro}. O
conhecimento do dever só se adquire muito lentamente, e é apenas pela
dor, pelo castigo e pelo exercício progressivo da razão que o homem
diminui devagar sua maldade natural.''

Assim, pelo simples bom"-senso, ele fazia um retorno à ideia católica,
pois se pode dizer que a criança, em geral, está, com relação ao homem,
na maioria das vezes, muito mais próxima do pecado original.

\sectionitem

Ter"-se"-ia dito que \indiceBA{Delacroix}{Eugène} havia reservado toda a sua sensibilidade,
que era viril e profunda, para o austero sentimento da amizade. Há
pessoas que se apaixonam facilmente por qualquer um; outras reservam o
uso da faculdade divina para as grandes ocasiões. O célebre homem de
quem lhe falo com tanto prazer, se não gostava que lhe perturbassem por
pequenas coisas, sabia se tornar obsequioso, corajoso, ardente, quando
se tratava de coisas importantes. Aqueles que o conheceram bem puderam
apreciar, em muitas oportunidades, sua fidelidade, sua exatidão e sua
solidez bem inglesas nas relações sociais. Se era exigente com os
outros, não era menos severo consigo mesmo.

É com tristeza e mau humor que quero dizer algumas palavras sobre certas
acusações lançadas contra Eugène \indiceBA{Delacroix}{Eugène}. Ouvi pessoas tacharem"-no de
egoísta e até mesmo de avaro. Observe, senhor, que essa censura é sempre
dirigida pela inumerável classe das almas banais àquelas que se aplicam
a fundamentar sua generosidade assim como sua amizade.

\indiceBA{Delacroix}{Eugène} era muito econômico; era para ele o único meio de ser,
eventualmente, muito generoso: eu poderia prová"-lo por meio de alguns
exemplos, mas recearia fazê"-lo sem ter sido autorizado por ele, não
mais do que por aqueles que tiveram de rejubilar"-se com ele.

Observe também que durante muitos anos suas pinturas venderam muito mal,
e que seus trabalhos de decoração absorviam quase a totalidade de seu
salário, quando não colocava nisso de seu próprio bolso. Ele provou
inúmeras vezes seu desprezo pelo dinheiro, quando artistas pobres
deixavam perceber o desejo de possuir alguma de suas obras. Então,
semelhantemente aos médicos de espírito liberal e generoso, que ora fazem
pagar seus cuidados e ora os oferecem, ele dava seus quadros ou os
cedia a qualquer preço.

Enfim, senhor, observemos bem que o homem superior é obrigado, mais do
que qualquer outro, a zelar por sua defesa pessoal. Pode"-se dizer que
toda a sociedade está em guerra contra ele. Pudemos verificar o caso
várias vezes. Sua polidez, denominam"-na frieza; sua ironia, por mais
mitigada que seja, maldade; sua economia, avareza. Mas se, ao
contrário, o infeliz se mostra imprevidente, bem longe de se apiedar
dele, a sociedade dirá: ``Bem feito; sua penúria é a
punição por sua prodigalidade''.

Posso afirmar que \indiceBA{Delacroix}{Eugène}, em matéria de dinheiro e de economia,
partilhava completamente a opinião de Stendhal, opinião que concilia a
grandeza e a prudência.

``O homem de espírito'', dizia esse último,
``deve se aplicar a adquirir o que lhe é estritamente
necessário para não depender de ninguém (no tempo de \indice{Stendhal} eram
6.000 francos de renda); mas se, tendo obtido essa segurança, perde seu
tempo a aumentar sua fortuna, é um miserável.''

Busca do necessário e desprezo pelo supérfluo, é uma conduta de homem
sábio e estoico.

Uma das grandes preocupações de nosso pintor, em seus últimos anos, era
o julgamento da posteridade e a solidez incerta de suas obras. Ora sua
imaginação tão sensível se inflamava com a ideia de uma glória imortal,
ora ele falava amargamente da fragilidade das telas e das cores. Outras
vezes citava com inveja os antigos mestres, que tiveram quase todos a
felicidade de ser traduzidos por hábeis gravadores, cuja ponta ou buril
soube se adaptar à natureza de seu talento, e lamentava ardentemente
não ter encontrado seu tradutor. Essa friabilidade da obra pintada,
comparada com a solidez da obra impressa, era um de seus temas
habituais de conversação.

Quando esse homem tão frágil e obstinado, tão nervoso e intrépido, esse
homem único na história da arte europeia, o artista doentio e
friorento, que sonhava sem parar em cobrir paredes com suas grandes
concepções, foi levado por uma dessas fluxões de tórax do qual ele
tinha, segundo parece, o convulsivo pressentimento, todos sentimos algo
de análogo a essa depressão de alma, a essa sensação de solidão
crescente que a morte de \indiceBA{Chateaubriand}{François"-René de} e a de \indiceBA{Balzac}{Honoré de} já nos fizeram
conhecer, sensação renovada recentemente pelo desaparecimento de Alfred
de \indiceBA{Vigny}{Alfred de}. Há num grande luto nacional um arrefecimento de vitalidade
geral, um obscurecimento do intelecto que se assemelha a um eclipse
solar, imitação momentânea do fim do mundo.

\EP[-2]
Creio, entretanto, que essa impressão afeta sobretudo esses altivos
solitários que só podem vislumbrar uma família pelas relações
intelectuais. Quanto aos outros cidadãos, em sua maioria, só aprendem
pouco a pouco a conhecer tudo o que a pátria desperdiçou ao perder o
grande homem, e que vazio ele deixa ao abandoná"-la. Ainda é preciso
adverti"-los.

Agradeço"-lhe de todo meu coração, senhor, por me ter deixado dizer
livremente tudo o que me sugeria a lembrança de um dos raros gênios de
nosso infeliz século --- tão pobre e tão rico ao mesmo tempo, ora muito
exigente, ora muito indulgente, e com muita frequência injusto.


\printindex

\SVN $Id: FINAIS.tex 9570 2011-08-17 18:18:01Z oliveira $

\ifodd\thepage\paginabranca\else\clearpage\fi
\pagestyle{empty}

\begingroup
%\fontsize{7}{8}\selectfont
\scriptsize
{\Formular{\large{COLEÇÃO HEDRA}}}\\
\begin{enumerate} [font=\Formular\scriptsize]
\setlength\parskip{8pt}
\setlength\itemsep{-1.4mm}
\item Iracema,\\ \textbf{José de Alencar}
\item Don Juan,\\ \textbf{Molière}
\item Contos indianos,\\ \textbf{Stéphane Mallarmé}
\item Auto da barca do Inferno,\\ \textbf{Gil Vicente}
\item Poemas completos de Alberto Caeiro,\\ \textbf{Fernando Pessoa}
\item Triunfos,\\ \textbf{Francisco Petrarca}
\item A cidade e as serras,\\ \textbf{Eça de Queiroz}
\item O retrato de Dorian Gray,\\ \textbf{Oscar Wilde}
\item A história trágica do Doutor Fausto,\\ \textbf{Christopher Marlowe}
\item Os sofrimentos do jovem Werther,\\ \textbf{Johann Wolfgang von Goethe}
\item Dos novos sistemas na arte,\\ \textbf{Kazimir Maliévitch}
\item Mensagem,\\ \textbf{Fernando Pessoa}
\item Metamorfoses,\\ \textbf{Ovídio}
\item Micromegas e outros contos,\\ \textbf{Voltaire}
\item O sobrinho de Rameau,\\ \textbf{Denis Diderot}
\item Carta sobre a tolerância,\\ \textbf{John Locke}
\item Discursos ímpios,\\ \textbf{Marquês de Sade}
\item O príncipe,\\ \textbf{Nicolau Maquiavel}
\item Dao De Jing,\\ \textbf{Lao Zi}
\item O fim do ciúme e outros contos,\\ \textbf{Marcel Proust}
\item Pequenos poemas em prosa,\\ \textbf{Charles Baudelaire}
\item Fé e saber,\\ \textbf{Friedrich Hegel}
\item Joana d'Arc,\\ \textbf{Jules Michelet}
\item Livro dos mandamentos: 248 preceitos positivos,\\ \textbf{Maimônides}
\item O indivíduo, a sociedade e o Estado, e outros ensaios,\\ \textbf{Emma Goldman}
\item Eu acuso! | O processo do capitão Dreyfus,\\ \textbf{Zola} | \textbf{Rui Barbosa}
\item Apologia de Galileu,\\ \textbf{Tommaso Campanella}
\item Sobre verdade e mentira,\\ \textbf{Friedrich Nietzsche}
\item O princípio anarquista e outros ensaios,\\ \textbf{Piotr Kropotkin}
\item Os sovietes traídos pelos bolcheviques,\\ \textbf{Rudolf Rocker}
\item Poemas,\\ \textbf{Lord Byron}
\item Sonetos,\\ \textbf{William Shakespeare}
\item A vida é sonho,\\ \textbf{Calderón de la Barca}
\item Escritos revolucionários,\\ \textbf{Errico Malatesta}
\item Sagas,\\ \textbf{August Strindberg}
\item O mundo ou tratado da luz,\\ \textbf{René Descartes}
\item O Ateneu,\\ \textbf{Raul Pompeia}
\item Fábula de Polifemo e Galateia e outros poemas,\\ \textbf{Luis de Góngora y Argote}
\item A vênus das peles,\\ \textbf{Sacher-Masoch}
\item Escritos sobre arte,\\ \textbf{Charles Baudelaire}
\item Cântico dos cânticos,\\ \textbf{{[}Salomão{]}}
\item Americanismo e fordismo,\\ \textbf{Antonio Gramsci}
\item O princípio do Estado e outros ensaios,\\ \textbf{Mikhail Bakunin}
\item História da província Santa Cruz,\\ \textbf{Pero de Magalhães Gandavo}
\item Balada dos enforcados e outros poemas,\\ \textbf{François Villon}
\item Sátiras, fábulas, aforismos e profecias,\\ \textbf{Leonardo Da Vinci}
\item O cego e outros contos,\\ \textbf{D.H.~Lawrence}
\item Rashômon e outros contos,\\ \textbf{Ryūnosuke Akutagawa}
\item História da anarquia (vol.~1),\\ \textbf{Max Nettlau}
\item Imitação de Cristo,\\ \textbf{Tomás de Kempis}
\item O casamento do Céu e do Inferno,\\ \textbf{William Blake}
\item Cartas a favor da escravidão,\\ \textbf{José de Alencar}
\item Utopia Brasil,\\ \textbf{Darcy Ribeiro}
\item Flossie, a Vênus de quinze anos,\\ \textbf{Algernon Charles Swinburne}
\item Teleny, ou o reverso da medalha,\\ \textbf{Oscar Wilde}
\item A filosofia na era trágica dos gregos,\\ \textbf{Friedrich Nietzsche}
\item No coração das trevas,\\ \textbf{Joseph Conrad}
\item Viagem sentimental,\\ \textbf{Laurence Sterne}
\item Arcana C\oe lestia e Apocalipsis revelata,\\ \textbf{Emanuel Swedenborg}
\item Saga dos Volsungos,\\ \textbf{Anônimo do séc.~\textsc{xiii}}
\item Um anarquista e outros contos,\\ \textbf{Joseph Conrad}
\item A monadologia e outros textos,\\ \textbf{Gottfried Wilhelm Leibniz}
\item Cultura estética e liberdade,\\ \textbf{Friedrich Schiller}
\item A pele do lobo e outras peças,\\ \textbf{Artur Azevedo}
\item Poesia basca: das origens à Guerra Civil 
\item Poesia catalã: das origens à Guerra Civil 
\item Poesia espanhola: das origens à Guerra Civil 
\item Poesia galega: das origens à Guerra Civil 
\item O pequeno Zacarias, chamado Cinábrio,\\ \textbf{E.T.A.~Hoffmann}
\item Tratados da terra e gente do Brasil,\\ \textbf{Fernão Cardim}
\item Entre camponeses,\\ \textbf{Errico Malatesta}
\item O Rabi de Bacherach,\\ \textbf{Heinrich Heine}
\item Bom Crioulo,\\ \textbf{Adolfo Caminha}
\item Um gato indiscreto e outros contos,\\ \textbf{Saki}
\item Viagem em volta do meu quarto,\\ \textbf{Xavier de Maistre }
\item Hawthorne e seus musgos,\\ \textbf{Herman Melville}
\item A metamorfose,\\ \textbf{Franz Kafka}
\item Ode ao Vento Oeste e outros poemas,\\ \textbf{Percy Bysshe Shelley}
\item Oração aos moços,\\ \textbf{Rui Barbosa}
\item Feitiço de amor e outros contos,\\ \textbf{Ludwig Tieck}
\item O corno de si próprio e outros contos,\\ \textbf{Marquês de Sade}
\item Investigação sobre o entendimento humano,\\ \textbf{David Hume}
\item Sobre os sonhos e outros diálogos,\\ \textbf{Jorge Luis Borges | Osvaldo Ferrari}
\item Sobre a filosofia e outros diálogos,\\ \textbf{Jorge Luis Borges | Osvaldo Ferrari}
\item Sobre a amizade e outros diálogos,\\ \textbf{Jorge Luis Borges | Osvaldo Ferrari}
\item A voz dos botequins e outros poemas,\\ \textbf{Paul Verlaine}
\item Gente de Hemsö,\\ \textbf{August Strindberg}
\item Senhorita Júlia e outras peças,\\ \textbf{August Strindberg}
\item Correspondência,\\ \textbf{Goethe | Schiller}
\item Índice das coisas mais notáveis,\\ \textbf{Antônio Vieira}
\item Tratado descritivo do Brasil em 1587,\\ \textbf{Gabriel Soares de Sousa}
\item Poemas da cabana montanhesa,\\ \textbf{Saigy\=o}
\item Autobiografia de uma pulga,\\ \textbf{{[}Stanislas de Rhodes{]}}
\item A volta do parafuso,\\ \textbf{Henry James}
\item Ode sobre a melancolia e outros poemas,\\ \textbf{John Keats} 
\item Teatro de êxtase,\\ \textbf{Fernando Pessoa}
\item Carmilla --- A vampira de Karnstein,\\ \textbf{Sheridan Le Fanu}
\item Pensamento político de Maquiavel,\\ \textbf{Johann Gottlieb Fichte}
\item Inferno,\\ \textbf{August Strindberg}
\item Contos clássicos de vampiro,\\ \textbf{Lord Byron, Bram Stoker e outros}
\item O primeiro Hamlet,\\ \textbf{William Shakespeare}
\item Noites egípcias e outros contos,\\ \textbf{Aleksandr Púchkin}
\item A carteira de meu tio,\\ \textbf{Joaquim Manuel de Macedo}
\item O desertor,\\ \textbf{Silva Alvarenga}
\item Jerusalém,\\ \textbf{William Blake}
\item As bacantes,\\ \textbf{Eurípides}
\item Emília Galotti,\\ \textbf{Gotthold Ephraim Lessing}
\item Contos húngaros,\\ \textbf{Dezso Kosztolányi, Frigyes Karinthy, Géza Csáth e Gyula Krúdy}
\item Viagem aos Estados Unidos,\\ \textbf{Alexis de Tocqueville}
\item Émile e Sophie ou os solitários,\\ \textbf{Jean-Jacques Rousseau}
\item Manifesto comunista,\\ \textbf{Karl Marx e Friedrich Engels}
\item A fábrica de robôs,\\ \textbf{Karel Tchápek}
\item Sobre a filosofia e seu método --- Parerga e paralipomena (v.~\textsc{ii}, t.~\textsc{i}),\\ \textbf{Arthur Schopenhauer} 
\item O novo Epicuro: as delícias do sexo,\\ \textbf{Edward Sellon}
\item Revolução e liberdade: cartas de 1845 a 1875,\\ \textbf{Mikhail Bakunin}
\item Sobre a liberdade,\\ \textbf{John Stuart Mill}
\item A velha Izerguil e outros contos,\\ \textbf{Maksim Górki}
\item Pequeno-burgueses,\\ \textbf{Maksim Górki}
\item Primeiro livro dos Amores,\\ \textbf{Ovídio}
\item Educação e sociologia,\\ \textbf{Émile Durkheim}
\item Elixir do pajé --- poemas de humor, sátira e escatologia,\\ \textbf{Bernardo Guimarães}
\item A nostálgica e outros contos,\\ \textbf{Alexandros Papadiamántis}
\item Lisístrata,\\ \textbf{Aristófanes}
\item A cruzada das crianças/ Vidas imaginárias,\\ \textbf{Marcel Schwob}
\item O livro de Monelle,\\ \textbf{Marcel Schwob}
\item A última folha e outros contos,\\ \textbf{O. Henry}
\item Romanceiro cigano,\\ \textbf{Federico García Lorca}
\item Sobre o riso e a loucura,\\ \textbf{{[}Hipócrates{]}}
\item Hino a Afrodite e outros poemas,\\ \textbf{Safo de Lesbos}
\item Anarquia pela educação,\\ \textbf{Élisée Reclus}
\item Ernestine ou o nascimento do amor,\\ \textbf{Stendhal}
\item Odisseia,\\ \textbf{Homero}
\item O estranho caso do Dr. Jekyll e Mr. Hyde,\\ \textbf{Robert Louis Stevenson}
\item História da anarquia (vol.~2),\\ \textbf{Max Nettlau}
\item Eu,\\ \textbf{Augusto dos Anjos}
\item Farsa de Inês Pereira,\\ \textbf{Gil Vicente}
\item Sobre a ética --- Parerga e paralipomena (v.~\textsc{ii}, t.~\textsc{ii}),\\ \textbf{Arthur Schopenhauer} 
\item Contos de amor, de loucura e de morte,\\ \textbf{Horacio Quiroga}
\item Memórias do subsolo,\\ \textbf{Fiódor Dostoiévski}
\item A arte da guerra,\\ \textbf{Nicolau Maquiavel}
\item O cortiço,\\ \textbf{Aluísio Azevedo}
\item Elogio da loucura,\\ \textbf{Erasmo de Rotterdam}
\item Oliver Twist,\\ \textbf{Charles Dickens}
\item O ladrão honesto e outros contos,\\ \textbf{Fiódor Dostoiévski}
\item O que eu vi, o que nós veremos,\\ \textbf{Santos-Dumont}
\item Sobre a utilidade e a desvantagem da história para a vida,\\ \textbf{Friedrich Nietzsche}
\item Édipo Rei,\\ \textbf{Sófocles}
\item Fedro,\\ \textbf{Platão}
\item A conjuração de Catilina,\\ \textbf{Salústio}
\end{enumerate}

\medskip

{\Formular{\large{SÉRIE LARGEPOST}}}

\begin{enumerate} [font=\Formular\scriptsize]
\setlength\parskip{8pt}
\setlength\itemsep{-1.4mm}
\item Dao De Jing,\\ \textbf{Lao Zi}
\item Cadernos: Esperança do mundo,\\ \textbf{Albert Camus}
\item Cadernos: A desmedida na medida,\\ \textbf{Albert Camus}
\item Cadernos: A guerra começou\ldots,\\ \textbf{Albert Camus}
\item Escritos sobre literatura,\\ \textbf{Sigmund Freud}
\item O destino do erudito,\\ \textbf{Johann Gottlieb Fichte}
\item Diários de Adão e Eva,\\ \textbf{Mark Twain}
\item Diário de um escritor (1873),\\ \textbf{Fiódor Dostoiévski}
\end{enumerate}


\medskip
{\Formular{\large{SÉRIE SEXO}}}

\begin{enumerate} [font=\Formular\scriptsize]
\setlength\parskip{8pt}
\setlength\itemsep{-1.4mm}

\item A vênus das peles,\\ \textbf{Sacher{}-Masoch}
\item O outro lado da moeda,\\ \textbf{Oscar Wilde}
\item Poesia Vaginal,\\ \textbf{Glauco Mattoso }
\item Perversão: a forma erótica do ódio,\\ \textbf{Robert Stoller}
\item A vênus de quinze anos,\\ \textbf{Algernon Charles Swinburne}
\item Explosão: romance da etnologia,\\ \textbf{Hubert Fichte}
\end{enumerate}

\medskip
{\Formular{\large{COLEÇÃO QUE HORAS SÃO?}}}

\begin{enumerate} [font=\Formular\scriptsize]
\setlength\parskip{8pt}
\setlength\itemsep{-1.4mm}
\item Lulismo, carisma pop e cultura anticrítica,\\ \textbf{Tales Ab'Sáber}
\item Crédito à morte,\\ \textbf{Anselm Jappe}
\item Universidade, cidade e cidadania,\\ \textbf{Franklin Leopoldo e Silva}
\item O quarto poder: uma outra história,\\ \textbf{Paulo Henrique Amorim}
\item Dilma Rousseff e o ódio político,\\ \textbf{Tales Ab'Sáber}
\item Descobrindo o Islã no Brasil,\\ \textbf{Karla Lima}
\item Michel Temer e o fascismo comum,\\ \textbf{Tales Ab'Sáber}
\item Lugar de negro, lugar de branco?,\\ \textbf{Douglas Rodrigues Barros}
\end{enumerate}

\medskip
{\Formular{\large{COLEÇÃO ARTECRÍTICA}}}

\begin{enumerate} [font=\Formular\scriptsize]
\setlength\parskip{8pt}
\setlength\itemsep{-1.4mm}
\item Dostoiévski e a dialética,\\ \textbf{Flávio Ricardo Vassoler}
\item O renascimento do autor,\\ \textbf{Caio Gagliardi}
\end{enumerate}

\endgroup

\pagebreak
\ifodd\thepage\paginabranca\else\clearpage\fi

\pagebreak

%\begin{blackpages}
\thispagestyle{empty}
\begin{techpage}{42mm}
		\putline{Edição}{André Fernandes}
		\putline{Co{}-edição}{Bruno Costa e Jorge Sallum}
		\putline{Capa e projeto gráfico}{Júlio Dui e Renan Costa Lima}
		\putline{Imagem de capa}{Eugène Delacroix, \textit{O mar de Dieppe}, 1852}
		\putline{Programação em LaTeX}{Marcelo Freitas}
		\putline{Consultoria em LaTeX}{Roberto Maluhy Jr.}
		\putline{Revisão}{Hedra, Daniela Marini}  
		\putline{Assistente editorial}{Bruno Oliveira e Janaína Navarro}
		\putline{Colofão}{Adverte-se aos curiosos que se
			imprimiu esta obra em nossas oficinas em \today, em papel 
			\mbox{off-set} 90~g/m²,
			composta em tipologia Minion Pro, 
			em \textsc{gnu}/Linux (Gentoo, Sabayon e Ubuntu), 
			com os softwares livres 
			\LaTeX, De\TeX, \textsc{vim}, Evince, Pdftk, 
			Aspell, \textsc{svn} e \textsc{trac}.}

\end{techpage}
%\end{blackpages}


\ifdefined\printcheck\printcheck\fi

\end{document}
