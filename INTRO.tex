%Jorge: Vasari, Watteau, Fragonard, Shelley

\chapter[Introdução, por Dirceu Villa]{introdução}
\hedramarkboth{introdução}{dirceu villa}

\section{o ensaísmo, grosso modo}

Não é do objetivo nem do escopo desta apresentação discorrer sobre o
ensaio, mas ela seria manca se não apresentasse uma ou duas ideias
nucleares sobre o assunto. Assim, o ponto inicial da nossa conversa
recua até a França do século \textsc{xvi}, e percebemos que depois de o grande	\index{França}
\indiceAB{Michel de}{Montaigne} (1533---1592) nomear o gênero e fornecer o
\textit{modus faciendi} já em nível de completa excelência em seus
\textit{Ensaios}, muitos autores adotaram a maneira de escrever
pensando com a liberdade erudita, como que de uma conversa educada,
propiciada pela forma. Nas palavras célebres de Montaigne, “leitor, sou
eu mesmo a matéria deste livro”, e nele serão encontrados “alguns
traços do meu caráter e das minhas ideias”.\footnote{ “Do autor ao
leitor”, em Michel Montaigne, \textit{Ensaios} (trad.
Sérgio Milliet), São Paulo, Abril Cultural, \textit{Os
pensadores}, vol.~\textsc{xi}, 1972, p.~11. Essa longa tradição 
reafirma"-se hoje,
nesses mesmos termos, nos ensaios brilhantes, peculiares e muito
pessoais de um grande poeta como Derek Walcott em \textit{What the \index{Walcott@Walcott, Derek|nn}
twilight says: essays}, por exemplo.}

Não se pode afirmar que Montaigne, em 1580, soubesse que a mão da
reflexão estética caberia tão bem na luva do ensaio. Os tratados, tendo
um compromisso mais científico e um andamento mais moroso, não
permitiam a ductibilidade característica de algo que, mesmo ainda tendo
suas regras escritas de composição, era aberta (e cada vez mais) a uma
\textit{ars combinandi}, às variáveis que se intensificariam desde a
erosão do terreno da retórica no século \textsc{xviii}. \indiceBA{Baudelaire}{Charles} nos diz que
sua intenção não era escrever um tratado, mas sim “participar ao leitor
algumas reflexões que me ocorrem com frequência”, e um ou dois
parágrafos adiante iria se separar ainda mais do ponto de vista
acadêmico, atacando diretamente o que seriam pressupostos acadêmicos
sobre o tema, e se queixando de “certos professores tidos como sérios,
charlatães da seriedade, cadáveres pedantescos saídos dos frios
hipogeus do Instituto”.

Quem nada tinha com isso --- e foi traduzido e admirado por \indiceBA{Baudelaire}{Charles}
--- era o estadunidense \indiceAB{Edgar Allan}{Poe}. 
O pensamento artístico de Poe, de orgulhosa autodeliberação,
fugindo dos estereótipos inspiradores do romantismo e propondo, como
diríamos com \indiceBA{Mallarmé}{Stéphane}, “quase uma arte”, serviu a sustentar e
desenvolver as próprias convicções de \indiceBA{Baudelaire}{Charles} não apenas como
artista, mas artista que pensa sobre sua arte, como lemos em sua obra
de ensaios. Os ensaios literários de \indiceBA{Poe}{Edgar Allan} já apresentam precisamente o
formato que conhecemos, ou tome"-se o exemplo de “The poetic principle”
(``O princípio poético''), no qual comenta seu conceito de poesia, além de
poemas de \indiceBA{Byron}{Lord}, \indice{Shelley} e outros, imprimindo"-os no corpo do texto para
mais fácil inteligibilidade. E nos dá, é claro, sua definição de
poesia:

\begin{hedraquote}
Sucintamente, eu definiria a Poesia das palavras como
\textit{A Criação Rítmica da Beleza}. Seu único árbitro é o Gosto. Com
o Intelecto ou com a Consciência tem suas únicas relações colaterais. A
não ser que, incidentalmente, não se ocupa nem do Dever nem da Verdade.\footnote{ ``\textit{I 
would define, in brief, the Poetry of words as \textit{The Rhythmical
Creation of Beauty.} Its sole arbiter is Taste. With the Intellect or
with the Conscience it has only collateral relations. Unless
incidentally, it has no concern whatever either with Duty or with
Truth}'',
em Edgar Allan Poe, \textit{The works of the late Edgar
Allan Poe} (ed. Rufus Wilmot Griswold), New York, 1850,  vol. \textsc{iii}, p.~8. 
Tradução do introdutor, bem como as demais citações sem indicação.}
\end{hedraquote}

A base de \indiceBA{Poe}{Edgar Allan} nos artigos e textos de maior fôlego de \indiceBA{Baudelaire}{Charles} deve
ser levada em consideração, e principalmente uma abordagem menos
derivada de um suposto rigor tratadístico: há nos ensaios de \indiceBA{Baudelaire}{Charles}
certa leveza e humor que terá tomado também da leitura de \indiceBA{Diderot}{Denis},
outra referência importante. Ao menos, percebe"-se que \indiceBA{Baudelaire}{Charles} terá
apreciado o estilo de Diderot que lemos já na primeira frase de “Dos
autores e dos críticos”: “Os viajantes falam de uma espécie de homens
selvagens, que sopram no passante agulhas envenenadas. É a imagem dos
nossos críticos”.\footnote{ Denis Diderot, “Dos autores e dos críticos”
(trad. e notas J.~Guinsburg), em \textit{Os
pensadores}, vol.~\textsc{xxii}, São Paulo, Abril Cultural, 1973, pp.~491---496.}

Um autor como \indiceAB{Samuel Taylor}{Coleridge} (1772---1834), por exemplo, já
dava exemplos de como o texto crítico de poeta funcionaria. A
\textit{Biographia literaria} (1817) é ainda resultado de um tipo de
concepção “minha vida, minhas ideias”, mas é possível perceber nela a
conjunção do ponto de vista de artista da palavra com a estrutura
fragmentária que se dedica a escrutinar um assunto, partilhando
impressões e formulando um modo específico de abordar a literatura e o
pensamento. \indiceBA{Coleridge}{Samuel Taylor} é um autor muito intelectual e seu matiz é
filosófico.

Assim também com a obra intelectual muito notável de outro poeta
do período, o italiano \indiceAB{Giacomo}{Leopardi} (1798---1837), que tende à
filosofia, impregnada das formas ditadas por seu muito profundo e
arraigado conhecimento das línguas e letras antigas da \indicex{Grécia}{Grecia} e de
\indice{Roma}, de onde os diálogos e os numerosos fragmentos filosóficos do
sugestivo título \textit{Zibaldone}, póstumo, que poderíamos traduzir
por “Miscelânea”. Vejamos, por exemplo, como escreve, em 1820, sobre a
obra de arte (ou “gênio”, como era o hábito à época):

\begin{hedraquote}
E mesmo o conhecimento da irreparável vaidade e falsidade de todo belo e de todo
grande é de certa beleza e grandeza que preenchem a alma, quando
esse conhecimento se encontra nas obras de gênio.\footnote{ ``\textit{E 
lo stesso conoscere l’irreparabile vanità e falsità di ogni bello e di
ogni grande è una certa bellezza e grandezza che rempie l’anima, quando
questa conoscenza si trova nelle opere di genio}'', em \indiceAB{Giacomo}{Leopardi}, 
\textit{Poesie e prose} (a cura di Siro Attilio Nulli), Milano, Hoepli, 1972, p.~590.}           \index{Nulli@Nulli, Sirio Attilio|nn}
\end{hedraquote}

Embora o estilo nessas duas obras, uma inglesa e outra italiana, se
aproxime mais das \textit{Fusées} de \indiceBA{Baudelaire}{Charles}, de seus diários e
fragmentos esparsos, ambas já provocam um efeito semelhante ao do
ensaio de poeta ou escritor que revela uma faceta crítica, que se
combina à inteligência prática de sua arte.

\section{fluência, ou folhas ao vento}

A fluência permitida pela página de jornal, e mesmo o princípio de
efemeridade desse tipo ordinário de papel impresso, fecundaram o
gênero: veremos que alguns dos ensaios de \indiceBA{Baudelaire}{Charles} --- como o artigo
sobre \indiceBA{Delacroix}{Eugène} --- foram compostos como cartas à direção de jornais, ou
artigos breves de crítica. Em “Da essência do riso e, de um modo geral,
do cômico nas artes plásticas”,\footnote{ Publicado em
\textit{Curiosités esthétiques} (1868).} o próprio \indiceBA{Baudelaire}{Charles} iria
assinalar esse aspecto transitório do jornal, referindo a função
imediata das caricaturas: “Assim como as folhas volantes do jornalismo,
elas desaparecem levadas pelo vento incessante que delas traz
notícias”. Esse caráter fugidio, quase de \textit{impromptu}, trouxe ao
ensaio --- que, já de nascença, como vimos, pretendia ser um registro
mais flexível do pensamento --- uma maleabilidade ainda maior, porque era
supostamente coisa bastante perecível.

Não foi tão perecível assim, verdade seja dita, ou não estaríamos neste
exato momento gastando tantas linhas numa apresentação aos ensaios de
\indiceBA{Baudelaire}{Charles}: sendo o molde tão favorável à veiculação das ideias, os
ensaios são então recolhidos em livro, edições que compilam esse
devanear orientado como uma topografia do terreno acidentado do
pensamento de determinado autor. Hoje, a reputação do ensaio é também
considerável nos meios acadêmicos, e muitos professores reúnem sua
crítica esparsa em livros que colecionam a obra ensaística. \indiceAB{Jacob}{Burckhardt}, 
ainda no século \textsc{xix}, chamou seu livro sobre a cultura do
\indice{Renascimento} italiano um “ensaio”. É um dispositivo que, além da
modéstia, confere limites ao estudo.

A brevidade do ensaio, patente desde o livro memorável e essencial de
\indiceBA{Montaigne}{Michel de}, permite também a concentração em um só ponto, excluindo a
necessidade de articular peças avulsas num todo coeso, e assim os
assuntos tratados marginalmente dentro de obras maiores tomaram o
primeiro plano, também porque se supunha que, como fragmento,
iluminariam de modo reflexivo o resto das matérias com que partilhassem
interesse.\footnote{ Uma ideia que só será ampliada por aquilo que, com
Ezra Pound, chamou"-se \textit{método ideogrâmico}, ou seja: imitando a  \index{Pound@Pound, Erza|nn}
estrutura dos ideogramas chineses e opondo"-se à hierarquização
dita \textit{aristotélica} do discurso (o que significa “começar
primeiro das coisas primeiras”, e assim por diante), os fragmentos
colados uns nos outros funcionariam como \textit{punti luminosi}, ou
“pontos luminosos”, em que a relação entre um e outro produziria um
novo e luminoso nexo de sentido.} 

Como resultado, avulta no gênero a opinião, que não é, no entanto, mero
palpite ignorante. A opinião dentro de um ensaio seguiria a receita
que, em outra ocasião, mas com propósito assemelhado, \indiceAB{Isaac}{Newton}
prescreveu assim: “Me apoio nos ombros de gigantes” (mesmo essa frase
ele havia tirado de algum de seus gigantes). Montaigne refere"-se a poetas,
pensadores, a todos aqueles indivíduos extraordinários que podem acorrer
em defesa ou ilustração de seu pensamento, ainda que “pessoal”, ainda
que recolhido à sua “ingenuidade física e moral”, como nos diz na
\textit{captatio benevolentia} da abertura do livro. \indiceBA{Baudelaire}{Charles}
escreverá, logo na abertura do primeiro de seus próprios ensaios: “Este
é, puramente, portanto, um artigo de filósofo e de artista”.

A opinião, no ensaio, é assim a opinião educada e focalizada, que pode
ser debatida nos termos de uma discussão literária, artística ou
filosófica civilizada, na qual um argumento persuasivo se sustenta com
exemplos e modelos, ou por contraste, e poderá ser contraposta por
argumentos de peso equivalente: é o que nos asseguram aquelas páginas
dedicadas a um assunto escolhido, e sob foco intenso de atenção mais ou
menos especializada, que se aproxima por vezes do diletantismo. E,
assim, um ensaio pode ser um artigo bastante persuasivo. Ele se
inscreve talvez numa das primeiras ambiguidades entre público e
privado: ele aproxima e afasta, ele afeta uma conversa, mas se
desenvolve como um monólogo indutor.

São essas as características do gênero ensaístico que permaneceram
fazendo dele um veículo apropriado para o uso das discussões sobre
estética e filosofia; ele se adequa também a uma cultura do fragmento
--- como tem sido a nossa cultura ocidental há mais de um século ---,
mas obedece a uma lei clássica, até meio militar, de “dividir para
conquistar”. É a divisão, a brevidade, o fragmento aparentemente
descompromissado que nos conquista.

\section{solidificando um gênero}

A obra de crítica de \indiceBA{Baudelaire}{Charles} praticamente rivaliza, em peso, com sua
obra literária. Essa afirmação se justifica da seguinte maneira: como
poeta, \indiceBA{Baudelaire}{Charles} é lembrado por estabelecer o princípio do que
chamariam depois \textit{modernidade}, ou seja, aquela então nova
atitude de negatividade em relação ao objeto de atenção da arte e em
relação ao público (que ele apelidou, no prefácio às \textit{Flores do
mal}, “leitor hipócrita”, e era, no mesmo verso, seu semelhante, seu
irmão); como crítico, de modo análogo, \indiceBA{Baudelaire}{Charles} sedimentou essa
figura hoje largamente difundida, a do poeta"-crítico,\footnote{ É
evidente que, como em tudo, não se tratava de uma novidade absoluta:
%r índice nota Dante
basta pensarmos em Meleagro na \textit{Antologia grega}, em Dante no	\index{Dante|nn}	\index{Meleagro|nn}
tratado \textit{De vulgari eloquentia}, nos poetas"-filólogos da
renascença italiana etc. Mas o diferencial está nessa crítica
ensaística, muitas vezes breve, e na atenção que não raro se opõe a
algo também recente como categoria isolada: a crítica de arte, a
crítica literária de corte acadêmico.} do artista que conjuga as duas
funções em sua obra e fornece um contraponto aos usos da crítica
acadêmica ou jornalística de sua época. Como esses poetas"-críticos que
já conhecemos desde o século \textsc{xx}, também \indiceBA{Baudelaire}{Charles} visitou artistas
negligenciados pela crítica, e seu estilo cuidadosamente pensado e seu 
notório faro qualitativo escolheram sempre com refinado senso estético
e formal, criando núcleos de interesse e sentido que ainda hoje lemos
como um mapa de direções das artes na segunda metade do século \textsc{xix}.

Assim, enfrentou com elegância a ideia monolítica do belo e do
verdadeiro em arte, dedicando"-se ao riso e à caricatura em alguns de
seus mais notáveis textos críticos, reformando ou desfazendo boa parte
do edifício estético romântico; defendeu o artifício não apenas como
coisa cosmética,\footnote{ Como Ovídio, que dedicou um poema muito	\index{Ovidio@Ovídio|nn}
interessante aos cosméticos (``Medicamina faciei feminae'',
“Cosméticos para o rosto da mulher”), \indiceBA{Baudelaire}{Charles} escreveu um
``Éloge du maquillage'', “Elogio da maquiagem”, em que comenta,
entre outras delicadezas, o direito da mulher parecer sempre “mágica e
sobrenatural”.} mas em termos que veríamos depois muito aprofundados
por \indiceAB{Fernando}{Pessoa} e Ezra Pound, ou seja, de uma técnica que serve a  \index{Pound@Pound, Erza}
revelar a emoção, transformando"-a ou \textit{fingindo"-a},\footnote{ O
verbo \textit{fingo}, em latim, significa “forjar”.} em arte, o que
na época soava provavelmente como antítese à espontaneidade declarada e
aparente do romantismo, ou simples aderência ao código de técnica
superficial do parnasianismo.\footnote{ Baudelaire entendia de modo
muito diferente o lema “\textit{l’art pour l’art}”, que lhe agradava: não a arte
como um enfeite descolado de qualquer relação com o mundo, mas
instalada e oferecida nele como uma contradição do artifício excelente
frente à espontaneidade, ao impulso natural.} 

Era, no entanto, uma visão que, coincidindo com a de \indiceAB{Edgar Allan}{Poe}\footnote{ Como 
vimos, Poe também era um ensaísta. Mais lembrado
pelo sagaz \textit{Philosohy of composition} (\textit{Filosofia da composição}),
de 1846, em que aborda o processo de escrita do seu poema ``The raven”
(``O corvo'') e no qual achamos a célebre observação de tê"-lo começado pelo
fim, com o objetivo de melhor controlar seus efeitos sobre o leitor,
como uma lei da composição.} em muitos pontos, sobretudo no domínio
técnico do autor sobre os efeitos calculados de sua arte, se tornaria
um índice muito explorado depois por autores como \indiceBA{Mallarmé}{Stéphane} e um bom
número de poetas modernos, principalmente. Devemos lembrar, juntamente,
que isso também ecoava o aprendizado clássico de \indiceBA{Baudelaire}{Charles}, que sempre
lhe foi muito útil e pode ser constatado no emprego habilidoso e muito
frequente da alegoria em seus poemas, ou no emprego da \textit{clarté
française}, a clareza da prosa em francês,\footnote{ Baudelaire escreve:
“França, país de pensamento e de demonstração claros”.} tornando seus	\index{França|nn}
ensaios muito bem divididos e argumentados.

\indiceBA{Baudelaire}{Charles}, portanto, preparou e antecipou, sob muitos aspectos, as
questões da chamada “modernidade”; estabeleceu um elo entre o
romantismo e o parnasianismo (no caso, leia"-se apenas \indiceAB{Théophile}%
{Gautier}) e forneceu, por suas ideias místicas e com o poema
“\textit{Correspondances}”, o princípio a partir do qual decadentistas e
simbolistas iriam mais tarde compor suas obras. Por fim, nos deu um dos
primeiros exemplos, com \indiceBA{Coleridge}{Samuel Taylor}, \indiceBA{Leopardi}{Giacomo} e \indiceBA{Poe}{Edgar Allan}, 
ao menos, do que seria o poeta"-crítico. 

\section{ideias: do particular para o geral}
Por que, então, as ideias de \indiceBA{Baudelaire}{Charles} --- como as que encontramos
neste livro --- se tornaram tão importantes? Basicamente, porque
compõem o desenho apresentado anteriormente: tendo surgido de uma intuição que
é ao mesmo tempo artística e crítica, sobre um detalhe potencialmente
iluminador, se desenvolve num dos primeiros conjuntos expressivos de
crítica exercida por um artista. 

Assim é que, profundamente cristão e particularmente satânico,
\indiceBA{Baudelaire}{Charles} flagra no riso, objeto de sua atenção no primeiro ensaio, a
dualidade que apresenta uma superfície de alegria para guardar no fundo
um sentimento demoníaco de superioridade. Leríamos mais tarde num texto
do filósofo \indiceAB{Henri}{Bergson} o mesmo ponto de partida sobre o assunto,
excluindo naturalmente a bruma fantástica do satanismo, para se deter
no aspecto da superioridade, o cerne da ironia.\footnote{ Embora Bergson	\index{Bergson@Bergson, Henri|nn}
se detenha no riso desde seus motivos mais elementares, desde reafirmar
que é especificamente uma característica humana, passando por vasculhar
as quimeras ridículas até asseverar o porquê de rirmos de um simples
tombo (a pessoa que cai demonstra rigidez quando deveria apresentar
flexibilidade), conclui que por meio do \textit{ridendo castigat mores}
(“com o riso corrigem"-se os costumes”) “o riso é sobretudo uma
correção”, em Henri Bergson, \textit{O riso, ensaio sobre o
significado do cômico} (tradução de Guilherme de Castilho),  Lisboa,		\index{Castilho@Castilho, Guilherme|nn}
Guimarães Editores, 1993.} Na argumentação desse primeiro ensaio de
\indiceBA{Baudelaire}{Charles}, os leitores lembrarão, imagino, da discussão mais ou menos
teológica entre \indiceAB{William de}{Baskerville} e o venerável Jorge no romance
\textit{O nome da rosa}, de \indiceAB{Umberto}{Eco}, que tem por base o segundo
livro, perdido, da \textit{Poética} de \indice{Aristóteles}, sobre a comédia. 

Baskerville, franciscano, quer ver no riso uma virtude da sobrevivência,
do conhecimento e da sanidade, enquanto o venerável Jorge afirma que o
riso deforma tanto a moral quanto o rosto, que assume traços de macaco.
Baskerville responde que “macacos não riem, o riso é próprio do homem”.
Aos exemplos saudáveis do riso em \indice{Quintiliano} e outros autores gregos e
latinos, o venerável Jorge apenas retruca: “Eram pagãos”.\footnote{ Umberto Eco,
 \textit{O nome da rosa} (trad. Aurora Fornoni Bernardini
e Homero Freitas de Andrade), Rio de Janeiro, Record, 1986, p.~158.}

O riso seria portanto uma deformidade do rosto que implicaria uma
deformidade moral civilizadíssima, já que, pressupõe"-se, a inocência
não conhece o sentimento de superioridade, é toda candura (e esse é o
riso das crianças). A ideia, puramente uma divisão de categoria em
\indice{Aristóteles} (“a comédia é a imitação dos piores”), transforma"-se numa
reflexão moral que atende a outro sistema de classificação, o cristão.
O venerável Jorge, de \indiceAB{Umberto}{Eco}, abomina o riso por esse motivo, o mesmo pelo
qual o nosso poeta francês, a seu turno, o admira. O sábio também
desconfia moralmente do riso: “Ele se detém à beira do riso assim como
à beira da tentação”, escreve \indiceBA{Baudelaire}{Charles}, e lembramos que é como o
infeliz Mêmnon\footnote{ “Memnon, ou la sagesse humaine” (``Mêmnon, ou a
sabedoria humana''). Curioso é notar que Baudelaire tem certa aversão a
\indice{Voltaire}, mesmo se, aqui e ali, encontramos pensamentos muito
semelhantes. O conto pode ser lido nesta mesma
coleção: Voltaire, \textit{Micromegas e outros contos} (tradução de
Graziela Marcolin), São Paulo, Hedra, 2007, pp. 59---65.} de Voltaire,
que traça um plano de sabedoria arruinado pelas tentações do mundo tão
pouco conhecidas por ele, que, portanto, não tem como se defender
delas. A desconfiança do sábio se dá porque, embora o homem não tenha
na boca os dentes do leão, “ele morde com o riso”. É o riso a um só
tempo cristão e satânico: \indiceBA{Baudelaire}{Charles} dará o exemplo do riso de \indice{Melmoth},
de Charles Maturin,\footnote{ Charles Maturin, \textit{Melmoth the 		\index{Maturin@Maturin, Charles Robert}
wanderer} (ed. Victor Sage), London, Penguin, 2001. Ao mencionar o
bizarro romance gótico de Maturin, Baudelaire teria em mente passagens
como esta: ``[\ldots] \textit{And after looking on them for some time, burst into
a laugh so loud, wild, and protracted, that the peasants, starting with
as much horror at the sound as at that of the storm, hurried away}'', isto é,  “[\ldots] e
após observá"-los por algum tempo, explodiu numa gargalhada tão alta,
selvagem e prolongada, que os camponeses, com tanto horror por tal som
quanto que pelo da tempestade, fugiram correndo”, p.~35.} daquele que
%r linha viúva entre páginas alma ao diabo --- neste nosso caso, figuradamente.
vendeu a alma ao diabo --- neste nosso caso, figuradamente.

Devemos notar, ainda nesse ensaio, que a divisão que \indiceBA{Baudelaire}{Charles}
estabelece entre os diferentes tipos de humor das nacionalidades
europeias foi feita, de modo assemelhado, na tradução do filho de
\indiceAB{Théophile}{Gautier} para \textit{As aventuras do Barão de Münchausen}, a
partir da versão de Gottfried August Bürger, publicada em \indice{Londres} em			\index{Burger@Bürger, Gottfried August}
1866. Lemos, no prefácio:

\begin{hedraquote}
A \textit{gaieté} francesa nada tem a ver com o \textit{humour}
britânico; o \textit{Witz} alemão difere da \textit{bufoneria}
italiana, e o caráter de cada nacionalidade aí se revela em sua livre
expansão.\footnote{ Gottfried August Bürger, \textit{As aventuras do
Barão de Münchausen} (trad. Moacir Werneck de Castro; ilustr. 
Gustave Doré), Belo Horizonte, Villa Rica, 1990, p.~13.}		\index{Dore@Doré, Gustave|nn}
\end{hedraquote}

\indiceAB{Théophile}{Gautier} \textit{fils} também não inventou isso, que é apenas
um dos modos mais antigos e comuns de se diferenciar culturalmente
desenhos num mapa. Decerto, \indiceBA{Baudelaire}{Charles} explora com maior detalhe o que
pensa que seja a veia humorística de cada país; dirá que \indiceBA{Rabelais}{François}, em
seus textos grotescos, conserva sempre algo “útil e racional”; a
“sonhadora \indice{Germânia}” fornece o “cômico absoluto”; os ingleses, ou os
que vivem nos “reinos brumosos do \textit{spleen}”, têm um “cômico feroz”; o
humor dos italianos é “excêntrico”, o dos espanhóis, “cruel”,
“sombrio”.

Uma extensão do tema pode ser lida em dois ensaios de \indiceBA{Baudelaire}{Charles}: um
deles pertence a esta coleção e se chama “Alguns caricaturistas
estrangeiros”.

Nele, descobrimos como nosso autor percebe o poder do detalhe deformador
da caricatura, que manipula a opinião sem dizer uma só palavra. A
deformação é um aspecto da arte, que para representar produz
deformações proporcionadas;\footnote{ Baudelaire tem perfeita
consciência disso. Escreve: “O grotesco domina o cômico de uma altura
proporcional”.} ou seja, a arte necessariamente deforma o objeto, e
não é outra a função das figuras de linguagem, de conceitos como a
concisão, a proporção. Na caricatura, como em toda arte que aborda o
\textit{genus humile} (gênero humilde) do cômico, a deformação tem o
caráter específico de representar coisa ridícula e desfavorável.
\indiceBA{Baudelaire}{Charles} comenta artistas muito diversos no gênero, e o faz de caso
pensado, rabiscando o croqui de uma tipologia da caricatura.

Todos, no entanto, partilham o uso da alegoria (como, de resto, ocorre
ainda hoje: basta lembrar das charges políticas de \indice{Angeli} na
\textit{Folha de S.~Paulo}, por exemplo), mas \indiceAB{William}{Hogarth}, o
primeiro a ser abordado, é visto na particularidade de uma imaginação
humorística mórbida, dos detalhes macabros da morte tornados risíveis.
Hogarth é um erudito, um grande curioso cheio de referências e,
evidentemente, um artista da inteligência, que produz menos
deformidades explícitas (como o rosto ridiculamente sinistro do padre
examinando um machado em \textit{The invasion plate} \textsc{i}, que talvez nos
recorde algo de \indiceBA{Goya}{Francisco José de}) e mais uma ironia entre as figuras em condições
indecentes ou francamente ordinárias.

Já Goya e \indiceBA{Brueghel}{Pieter} são vistos por \indiceBA{Baudelaire}{Charles} como os tipos “sempre
grande artista” ou que fazem “coisas belíssimas” --- nesse último caso,
falando dos flamengos. Ao abordar os dois, \indiceBA{Baudelaire}{Charles}, que não é apenas
um crítico literário, mas muito talentoso e cuidadoso crítico de arte,
separa a arte da gravura (sobretudo das técnicas de água"-forte e
água"-tinta), que é a arte da caricatura por excelência de \indiceBA{Goya}{Francisco José de} ---
tecnicamente com pontos de contato com \indice{Rembrandt} ---, da arte
miniaturista flamenga de \indiceBA{Brueghel}{Pieter}, para a qual a alegoria é como uma anedota,
ou toma a forma de um ditado popular, que demonstram em suas divertidas
diabruras toda a “força da alucinação”. Os \textit{Caprichos} de Goya
são vistos com interesse por \indiceBA{Baudelaire}{Charles} porque criam um ruído na
alegoria ao misturarem"-na com enigmas e símbolos de interpretação
duvidosa ou aberta, em geral para efeito assustador: a deformidade
conduz a um mistério sombrio.

É como \indiceAB{João Adolfo}{Hansen} escreve em seu livro
\textit{Alegoria}:\footnote{ João Adolfo Hansen, \textit{Alegoria:
construção e interpretação da metáfora}, São Paulo/Campinas,
Hedra/Unicamp, 2006, pp. 66---67 e 86.} trata"-se da
\textit{permixta apertis allegoria}, ou alegoria imperfeita, quando
escreve sobre o casal amarrado a um tronco de árvore, com a figura
perturbadora de uma gigantesca coruja cujas garras se apoiam na cabeça
da mulher, gravura que \indiceBA{Goya}{Francisco José de} chamou “¿\textit{No hay quien nos desate}?”
(``Não há quem nos desate?''). O casal atado sem que um consiga
escapar do outro é uma alegoria bastante clara do casamento infeliz,
mas a terrível coruja insere um elemento de interpretação enigmática,
indecisa.

E \indiceBA{Baudelaire}{Charles}, claramente extasiado com as visões de \indiceBA{Goya}{Francisco José de}, escreve: 

\begin{hedraquote}
Une à alegoria, à jovialidade, à sátira espanhola do bom tempo de
\indiceBA{Cervantes}{Miguel de} um espírito muito mais moderno, ou pelo menos que foi muito
mais procurado nos tempos modernos, o amor do inapreensível, o
sentimento dos contrastes violentos, dos pavores da natureza e das
fisionomias humanas estranhamente animalescas pelas circunstâncias.
\end{hedraquote}

O que é também muito interessante por nos dar uma definição do próprio
\indiceBA{Baudelaire}{Charles} sobre como via a modernidade; ou, digamos com mais precisão,
a \textit{sua} modernidade.

Evidentemente, as imagens de bruxas e sabás não lhe passam
despercebidas. Assinala, adiante, num comentário aos desenhos de \indiceAB{Leonardo}{da
Vinci}, o que lhe parece a ausência de uma \textit{uis comica}, a
ausência de uma veia cômica, e lê as figuras do mestre italiano como
retratos da feiura, resultados da onívora curiosidade de \indiceBA{da Vinci}{Leonardo} pela
natureza em suas variadas formas --- o que é, por outro lado, lê"-lo
através do retrato de \indice{Vasari}, em \textit{Le vite dei più eccellenti
pittori, scultori e architetti}, até certo ponto anedótico naquele modo
de se redigir uma nota biográfica, bastante diverso da “veracidade
factual” suposta no gênero atualmente.

Percebemos que mesmo seu artigo sobre a arte filosófica se transforma
num exame da monstruosidade, uma vez que a filosofia na arte provocaria
a deformação do pensamento sobre a representação, através da fórmula
%r grafado Chavenard na prova
que atribui a \indiceBA{Chenavard}{Paul} e à arte alemã: “É uma arte plástica que tem a
pretensão de substituir o livro, quer dizer, rivalizar com a arte de
imprimir para ensinar a história, a moral e a filosofia.” Ele percebe
que a arte começa a ser trocada pelo \textit{típico histórico}, pelo
efeito de mera ilustração episódica. 

Sua noção disso é a da decadência de uma arte --- não como
veríamos depois na \indice{França} com os nomes de \indiceBA{Mallarmé}{Stéphane}, \indiceBA{Huysmans}{J.-K.~}, \indiceBA{Verlaine}{Paul}
etc., em que “decadência” é revertida para o limite de um sentido
“positivo” do excesso de civilização, do brilho mais intenso da estrela
que está para se apagar ---, mas sim a decadência de velhice e cansaço
de uma cultura que exaure suas energias criativas, e que “corresponde
ao período no qual entraremos em breve e cujo começo está marcado pela
supremacia da \indice{América} e da indústria”, escreve, revelando inesperados
dons proféticos.

Estão entremeados nessa crítica desfavorável de \indiceBA{Baudelaire}{Charles} o
raciocínio, os limites da forma, a contraposição da inocência pelo
engenho --- que nos devolveria, num círculo, às suas observações sobre
a natureza do riso. Vemos \indiceBA{Baudelaire}{Charles} fascinado pelas figuras grotescas,
pelas figuras que desempenham funções que se opõem às de sua natureza,
como a Morte tocando música encantadora ao violino, na \textit{Danse
des mortes}; ou fascinado por preciosas bizarrices, como será mais
tarde o caso do baudelairiano personagem Des Esseintes (do romance
\textit{Às avessas}, de \indiceAB{J.-K.~}{Huysmans}), quando observa e descreve
gravuras holandesas nas quais a crueldade ou o horror se tornam itens
de curiosidade quase que desinteressada, de concentração nos
esplendores das incongruências; ou o que, como nas \textit{Flores do mal}, 
poderíamos chamar de “horror simpático”:

\begin{hedraquote}
\begin{verse}
--- Deste céu bizarro e cinzento,\\
E turvo como o teu destino, \\
À tua alma que pensamento \\ 
Desce? Responde, libertino. \\ 
\ \\ 
--- Insaciavelmente guloso \\
Do que é incerto e que é sibilino, \\ 
\indicex{Ovídio}{Ovidio} não serei choroso \\ 
Quando expulso do Éden latino. \\ 
 \ \\
Céus lacerados e tristonhos, \\  
Em vós meu orgulho se fita; \\
As nuvens de dor infinita\\ 
 \ \\
São carros fúnebres dos sonhos, \\ 
Vosso clarão a sombra traz \\ 
Do Inferno à que minha alma apraz!\footnote{ \textit{``--- De ce ciel bizarre et livide, / 
Tourmenté comme ton destin, /
Quels pensers dans ton âme vide / 
Descendent? réponds, libertin. //
--- Insatiablement avide /
De l'obscur et de l'incertain, /
Je ne geindrai pas comme Ovide /
Chassé du paradis latin.  //
Cieux déchirés comme des grèves /
En vous se mire mon orgueil; /
Vos vastes nuages en deuil //
Sont les corbillards de mes rêves, / 
Et vos lueurs sont le reflet /
De l'enfer où mon coeur se plaît.''}
Baudelaire, Charles. \textit{As flores do mal} (trad.,
pref. e notas de Jamil Almansur Haddad), São Paulo, Difel, 1964, p.~217.} 
\end{verse}
\end{hedraquote}

“O Inferno em que meu coração se compraz”. Esse poema codifica uma das
bases do pensamento poético de \indiceBA{Baudelaire}{Charles} nas \textit{Flores do mal},
unindo repulsa e atração, concomitantes, como lemos também nestes
ensaios e nos \textit{Pequenos poemas em prosa}: como flagra neles a
cidade de \indice{Paris}, a metrópole. A força, a cor, a agitação, os contrastes
ferozes e o arrebatamento são também os motivos que o levam, neste
contínuo de sentido, dos caricaturistas às deformações de \indiceBA{Goya}{Francisco José de}, da
desproporção filosófica à arte plástica de \indiceAB{Eugène}{Delacroix}. Sobre ele
\indiceBA{Baudelaire}{Charles} escreve: “Eugène Delacroix era uma curiosa mistura de
ceticismo, polidez, dandismo, vontade ardente, astúcia, despotismo e,
enfim, uma espécie de bondade particular e de ternura moderada que
acompanha sempre o gênio”.

Delacroix é um dos pintores mais intensos da pintura francesa.
Contraposto à delicadeza de caixa de música dos quadros adoráveis de
\indice{Watteau}, que compunham uma sociedade do gosto superficial e requintado,
\indiceBA{Delacroix}{Eugène} visitará as terras do heroísmo e da revolução, e irá do
exótico e peregrino ao auto"-retrato de linhas dramáticas. O equivalente
disso para \indiceBA{Baudelaire}{Charles} em música é \indiceBA{Wagner}{Richard}. Não por acaso, \indiceBA{Baudelaire}{Charles},
que não era crítico musical tão pronto como crítico de arte, escreveu a
Wagner que passagens de sua música o faziam imaginar um vermelho
profundo. 

Podemos encontrar o equivalente visual da música trágica e poderosa de
\indiceBA{Wagner}{Richard} na pintura de \indiceBA{Delacroix}{Eugène}. Basta lembrar de \textit{Fantasia
árabe}, quadro inscrito no Salão de 1834, no qual o céu de batalha, cor
de terra e alaranjado, encontra uma equivalência visual de sua
intensidade no cavalo negro do árabe em primeiro plano, que se empina à
vista do tumulto. Sua pintura é construída um passo além daquela que,
dois séculos antes dele, se propunha como o contraste dramático, fosse
da escuridão com luzes fortes e focalizadas, fosse o aproveitamento
também dramático das diagonais no quadro.

A amizade do jovem poeta pelo pintor mais velho começa em 1845, segundo
o próprio \indiceBA{Baudelaire}{Charles} na carta"-artigo a \textit{L’Opinion Nationale}, de
novembro de 1863, que integra este nosso livro. E \indiceBA{Baudelaire}{Charles} afirma que
a grandeza de \indiceBA{Delacroix}{Eugène} está em pintar o invisível: certamente. Outro
pintor que menciona entre os grandes franceses é \indiceAB{Jacques"-Louis}{David},
que vem, aliás, muito a propósito para entendermos o que \indiceBA{Baudelaire}{Charles}
quer dizer com pintar o invisível. Se \indice{Watteau} (ou \indice{Fragonard}, pouco mais
tarde) é quase a delicadeza da miniatura, que mimetiza sua sociedade do
detalhe,\footnote{ Como escreve o historiador Piero Camporesi sobre o \index{Camporesi, Piero|nn}
gosto no século \textsc{xviii}: “A interiorização do prazer é destilada pela
miniaturização da paisagem e o apequenamento das coisas. O olho tem de
ser acariciado por coisas agradáveis, de medida justa mas tendentes à
bem temperada graciosidade”, em Piero Camporesi,
\textit{Hedonismo e exotismo: A arte de viver na Época das Luzes
}(trad. Gilson César Cardoso de Souza), São Paulo, Editora Unesp,
1996, p.~127.} David tem escopo maior; como Delacroix, ele também
pinta o invisível, mas um invisível de “fria” alegoria, de fixidez
estatuária, como vemos no famoso \textit{Juramento dos horácios}, de
1784, e não a velocidade e a paixão, como Delacroix, de quem \indiceBA{Baudelaire}{Charles}
sabiamente diz: “Ele fez isso --- observe"-o bem, senhor --- sem outros
meios além do contorno e da cor; ele o fez melhor do que ninguém; ele o
fez com a perfeição de um pintor consumado, com o rigor de um literato
sutil, com a eloquência de um músico apaixonado”. Nesse perpassar das
artes e sensações, seria preciso fixar o adjetivo “apaixonado”.

E essa descrição nos apresenta também uma lei que é tanto plástica
quanto poética, encontrável nas ideias de \indiceBA{Baudelaire}{Charles} também sob a forma
daquela mística infusão de sentidos do soneto “Correspondances”, das
\textit{Flores do mal}, quando escreve: 

\begin{hedraquote}
\begin{verse}
Como os longos ecos que de longe se confundem \\ 
Numa tenebrosa e profunda unidade, \\ 
Vasta como a noite e como a claridade, \\ 
Os perfumes, as cores e os sons se correspondem.\footnote{ \textit{``Comme de longs échos qui de loin se confondent /
Dans une ténébreuse et profonde unité, /
Vaste comme la nuit et comme la clarté, / 
Les parfums, les couleurs et les sons se répondent.''} 
Charles Baudelaire, \textit{As flores do mal}
(trad., intr. e notas Ivan Junqueira), Rio de Janeiro, Nova
Fronteira, 1985, pp.~114---115.}
\end{verse}
\end{hedraquote}

O que mencionei anteriormente da visão de \indiceBA{Baudelaire}{Charles} ao ouvir \indiceBA{Wagner}{Richard} também se
aplica ao artigo “Richard Wagner e \indice{Tannhäuser} em \indice{Paris}”, no qual
\indiceBA{Baudelaire}{Charles} dizia suas impressões sobre a apresentação musical e rebatia
críticas francesas ao compositor. Encontramos o soneto
“Correspondances” citado como exemplo do que queria dizer, associando
sons a cores, ou visões. E completa: “O que seria verdadeiramente
surpreendente é que o som \textit{não pudesse} sugerir a cor, que as
cores \textit{não pudessem} dar a ideia de uma melodia, e que o som e a
cor fossem impróprios para traduzir ideias”.\footnote{ Charles Baudelaire,
 \textit{Obras estéticas: filosofia da imaginação criadora}
(trad. Edison Darci Heldt), Petrópolis, Vozes, 1993, p.~162. 
O célebre soneto das vogais, de Rimbaud, deriva dessas ideias.} 	\index{Rimbaud|nn}

Podemos depreender disso o valor que \indiceBA{Baudelaire}{Charles} confere à
\textit{sinestesia}, essa mágica mistura e correspondência, algo que
ele encontrava na arte que mais lhe suscitava interesse, fosse a poesia
ou a pintura: elas estão imbuídas de qualidades sensórias
intercambiáveis. 

O entusiasmo de \indiceBA{Baudelaire}{Charles} pela arte de \indiceBA{Delacroix}{Eugène}, assim como os hábitos
clássicos de sua escrita, ou o apreço por \indiceBA{Gautier}{Théophile}, devem também
demonstrar aos leitores que a costumeira divisão periódica da
literatura em “movimentos” que se oporiam não é tão factual quanto
parece, pois os escritores elegem seus antecessores com muito mais
liberdade do que os manuais de literatura fazem acreditar: é possível
perceber como \indiceBA{Baudelaire}{Charles}, através de suas posições críticas, se
estabelece num contínuo em que interagem lemas clássicos, preceitos
“barrocos”, romantismo, parnasianismo, decadência, simbolismo,
“modernidade”. A fluidez do pensamento artístico dos grandes autores
normalmente não se fecha em meia dúzia de estilemas de fácil
identificação.

Essa é a essência do que, suponho, poderemos apreciar na mente invulgar
que teceu as considerações inteligentes e bem"-humoradas destes ensaios
a seguir. Boa leitura.
